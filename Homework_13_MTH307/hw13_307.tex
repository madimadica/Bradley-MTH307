\documentclass[12pt]{article}
%\usepackage[document]{ragged2e}
\usepackage{array, amssymb, amsthm, linguex, enumerate, amsmath, physics, enumitem, xcolor, graphicx, xparse}
\let\fg\undefined %remove linguex/siunitx naming clash
\usepackage[english]{babel}
\usepackage[letterpaper,top=2cm,bottom=2cm,left=3cm,right=3cm,marginparwidth=1.75cm]{geometry}
\usepackage[colorlinks=true, allcolors=blue]{hyperref}
\usepackage[group-separator={,}]{siunitx} %\num{12345} -> "12,345"
\usepackage{fancyhdr}
\usepackage{notomath}
\usepackage[T1]{fontenc}

%Number sets
\newcommand{\R}{\mathbb{R}}
\newcommand{\C}{\mathbb{C}}
\newcommand{\N}{\mathbb{N}}
\newcommand{\F}{\mathbb{F}}
\renewcommand{\Re}{\operatorname{Re}}
\renewcommand{\Im}{\operatorname{Im}}
\renewcommand{\L}[1]{\mathcal{L}\left({#1}\right)} %Linear Map

\newcommand{\pmp}{\,\pm\,} %add small extra space to \pm

\NewDocumentCommand{\ceil}{ s m }{% ceiling brackets
    \IfBooleanTF{#1}%
    {\lceil #2 \rceil}% starred: no-autosizing
    {\left\lceil #2 \right\rceil}% unstarred: autosizing
}

\NewDocumentCommand{\ceiling}{ s m }{% ceiling brackets
    \IfBooleanTF{#1}%
    {\lceil #2 \rceil}% starred: no-autosizing
    {\left\lceil #2 \right\rceil}% unstarred: autosizing
}

\NewDocumentCommand{\floor}{ s m }{% floor brackets
    \IfBooleanTF{#1}%
    {\lfloor #2 \rfloor}% starred: no-autosizing
    {\left\lfloor #2 \right\rfloor}% unstarred: autosizing
}

\NewDocumentCommand{\pars}{ s m }{% parenthesis
    \IfBooleanTF{#1}%
    {( #2 ) }% starred: no-autosizing
    {\left( #2 \right) }% unstarred: autosizing
}

\NewDocumentCommand{\inner}{ s m }{% inner product
    \IfBooleanTF{#1}%
    {\langle #2 \rangle}% starred: no-autosizing
    {\left\langle #2 \right\rangle}% unstarred: autosizing
}

\NewDocumentCommand{\innerconj}{ s m }{% inner product
    \IfBooleanTF{#1}%
    {\overline{\langle #2 \rangle}}% starred: no-autosizing
    {\overline{\left\langle #2 \right\rangle}}% unstarred: autosizing
}

\NewDocumentCommand{\brac}{ s m }{% brackets
    \IfBooleanTF{#1}%
    {[#2] }% starred: no-autosizing
    {\left[ #2 \right] }% unstarred: autosizing
}

%default latex bracket size naming
\newcommand{\biggbrac}[1]{\bigg[ {#1} \bigg] }
\newcommand{\bigbrac}[1]{\big[ {#1} \big] }
\newcommand{\Bigbrac}[1]{\Big[ {#1} \Big] }


\RenewDocumentCommand{\over}{ s m }{% fraction 1/arg
    \IfBooleanTF{#1}%
    {\dfrac{1}{#2}}% starred: dfrac
    {\frac{1}{#2}}% unstarred: normal frac
}

\NewDocumentCommand{\pover}{ s m }{% parenthesis around fraction (1/arg)
    \IfBooleanTF{#1}%
    {\left(\dfrac{1}{#2}\right)}% starred: dfrac
    {\left(\frac{1}{#2}\right)}% unstarred: normal frac
}

\NewDocumentCommand{\pfrac}{ s m m}{% parenthesis around fraction (arg1/arg2)
    \IfBooleanTF{#1}%
    {\left( \dfrac{{#2}}{{#3}} \right)}% starred: dfrac
    {\left( \frac{{#2}}{{#3}} \right)}% unstarred: normal frac
}


\newcommand{\Xbar}{\bar{X}}
\newcommand{\Ybar}{\bar{Y}}
\newcommand{\xbar}{\bar{x}}
\newcommand{\ybar}{\bar{y}}


\newcommand{\limn}{\lim_{n\to\infty}}

\newcommand{\gammaDist}[2]{\operatorname{Gamma} \left( {#1},{#2} \right)} %gamma distribution
\NewDocumentCommand{\normalDist}{s g g}{ %normal distibution
    \IfBooleanTF{#1} { % starred, no autosizing parenthesis
      \IfNoValueTF{#2}{ 
          N (\mu,\, \sigma^2 ) %\normalDist* "default" normal distribution N(\mu, \sigma^2)
        } {
            \IfNoValueTF{#3}{N (#2)}{} %\normalDist{arg} --> N(arg)
        }
      \IfNoValueTF{#3}{}{N ( #2, #3 )}  %\normalDist*{arg1}{arg2} --> N(arg1,arg2)
    }  % else (unstarred) autosize parenthesis
    {
        \IfNoValueTF{#2}{
            N \left(\mu,\, \sigma^2 \right) %\normalDist "default" normal distribution N(\mu, \sigma^2)
        } {
            \IfNoValueTF{#3}{N \left(#2\right)}{} %\normalDist{arg} --> N(arg)
        }
        \IfNoValueTF{#3}{}{N \left( #2, #3 \right)} %\normalDist{arg1}{arg2} --> N(arg1,arg2)
    }
}



%colors
\definecolor{ggreen}{RGB}{0, 127, 0}
\definecolor{dgray}{RGB}{63,63,63}
\definecolor{neonorange}{RGB}{255,47,0}
\definecolor{mygray}{rgb}{0.5,0.5,0.5}
\definecolor{eblue}{RGB}{0,74,127}
\newcommand{\red}[1]{\color{red}{#1}\color{black}}
\newcommand{\green}[1]{\color{ggreen}{#1}\color{black}}
\newcommand{\blue}[1]{\color{blue}{#1}\color{black}}
\newcommand{\setRed}{\color{red}}
\newcommand{\setBlack}{\color{black}}
\newcommand{\setBlue}{\color{blue}}
\newcommand{\setGreen}{\color{ggreen}}



\newcommand{\thru}[1]{{#1}_1, \dots, {#1}_n}
\newcommand{\sumThru}[1]{{#1}_1 + \cdots + {#1}_n}
\newcommand{\yn}{Y_1, \dots, Y_n} % Y_1, ..., Y_n
\newcommand{\xn}{X_1, \dots, X_n} % Y_1, ..., Y_n

%hats and tildes
\newcommand{\that}{\widehat{\theta}} % theta hat
\newcommand{\phat}{\widehat{p}} % p hat
\newcommand{\qhat}{\widehat{q}} % p hat
\newcommand{\psihat}{\widehat{\psi}} % psi hat
\newcommand{\Psihat}{\widehat{\Psi}} % Psi hat
\newcommand{\ptilde}{\widetilde{p}} % psi tilde
\newcommand{\Psitil}{\widetilde{\Psi}} % Psi tilde
\newcommand{\betah}{\widehat{\beta}} % beta hat

%2x2 matrix shortcuts
\newcommand{\detx}[4]{\begin{vmatrix}{#1} & {#2}\\{#3}&{#4}\end{vmatrix}} % 2x2 determinant
\newcommand{\bmat}[4]{\begin{bmatrix}{#1} & {#2}\\{#3}&{#4}\end{bmatrix}} % 2x2 matrix brackets
\renewcommand{\pmat}[4]{\begin{pmatrix}{#1} & {#2}\\{#3}&{#4}\end{pmatrix}} % 2x2 matrix parenthesis

%remove any enumerate/itemize indent temporarily
\makeatletter   %% <- make @ usable in macro names
\newcommand*\notab[1]{%
  \begingroup   %% <- limit scope of the following changes
    \par        %% <- start a new paragraph
    \@totalleftmargin=0pt \linewidth=\columnwidth
    %% ^^ let other commands know that the margins have been reset
    \parshape 0
    %% ^^ reset the margins
    #1\par      %% <- insert #1 and end this paragraph
  \endgroup
}
\makeatother    %% <- revert @


\newcommand{\dimrange}[1]{\operatorname{dim}\operatorname{range}{#1}} % dimrange
\newcommand{\dimnull}[1]{\operatorname{dim}\operatorname{null}{#1}} % dimnull
\newcommand{\range}[1]{\operatorname{range}{#1}} %range
\newcommand{\nullspace}{\operatorname{null}} %null

% polynomial notation
\NewDocumentCommand{\poly}{ s g g }{%
    \IfBooleanTF{#1} {
        \IfNoValueTF{#2} {
            \mathcal{P}(\mathbb{R})
        } {
            \mathcal{P}_{#2}(\mathbb{R})
        }
    } {
        \IfNoValueTF{#3} {
            {\mathcal{P}(#2)}
        } { %else
            {\mathcal{P}_{#2}(#3)}
        }
    }
}

\NewDocumentCommand{\bias}{ s m }{% bias(arg)
    \IfBooleanTF{#1}%
    {\operatorname{bias}(#2)}% starred: no autosizing
    {\operatorname{bias}\left(#2\right)}% unstarred: autosizing
}

\NewDocumentCommand{\MSE}{ s m }{% MSE(arg)
    \IfBooleanTF{#1}%
    {\operatorname{MSE}(#2)}% starred: no autosizing
    {\operatorname{MSE}\left(#2\right)}% unstarred: autosizing
}

\NewDocumentCommand{\Var}{ s m }{% variance with parenthesis V(arg)
    \IfBooleanTF{#1}%
    {\operatorname{Var}(#2)}% starred: no autosizing
    {\operatorname{Var}\left(#2\right)}% unstarred: autosizing
}

\NewDocumentCommand{\Varb}{ s m }{% variance with brackets V[arg]
    \IfBooleanTF{#1}%
    {\operatorname{Var}[\,#2\,]}% starred: no autosizing
    {\operatorname{Var}\left[\,#2\,\right]}% unstarred: has autosizing
}

\NewDocumentCommand{\Vb}{ s m }{% another renaming of variance with brackets V[arg]
    \IfBooleanTF{#1}%
    {\operatorname{Var}[\,#2\,]}% starred: no autosizing
    {\operatorname{Var}\left[\,#2\,\right]}% unstarred: has autosizing
}

\NewDocumentCommand{\E}{ s m }{% expectation with parenthesis E(arg)
    \IfBooleanTF{#1}%
    {\operatorname{E}(#2)}% starred: no autosizing
    {\operatorname{E}\left(#2\right)}% unstarred: has autosizing
}

\NewDocumentCommand{\Eb}{ s m }{% expectation with brackets E[arg]
    \IfBooleanTF{#1}%
    {\operatorname{E}[#2]}% starred: no autosizing
    {\operatorname{E}\left[#2\right]}% unstarred: has autosizing
}

\RenewDocumentCommand{\P}{ s m }{% probability with parenthesis Pr(arg)
    \IfBooleanTF{#1}%
    {\Pr (#2) }% starred: no autosizing
    {\Pr \left( #2 \right) }% unstarred: has autosizing
}

\NewDocumentCommand{\prob}{ s m }{% probability with parenthesis Pr(arg)
    \IfBooleanTF{#1}%
    {\Pr (#2) }% starred: no autosizing
    {\Pr \left( #2 \right) }% unstarred: has autosizing
}

\NewDocumentCommand{\eff}{ s m }{% efficiency with parenthesis eff(arg)
    \IfBooleanTF{#1}%
    {\operatorname{eff}(#2)}% starred: no autosizing
    {\operatorname{eff}\left(#2\right)}% unstarred: has autosizing
}

%vertical vector of up to 8 elements
\NewDocumentCommand\vvec{s m g g g g g g g}{%
    \IfBooleanTF{#1} {
        \begin{bmatrix}% if starred use brackets
            \IfNoValueTF{#2}{}{#2}
            \IfNoValueTF{#3}{}{\\#3}
            \IfNoValueTF{#4}{}{\\#4}
            \IfNoValueTF{#5}{}{\\#5}
            \IfNoValueTF{#6}{}{\\#6}
            \IfNoValueTF{#7}{}{\\#7}
            \IfNoValueTF{#8}{}{\\#8}
        \end{bmatrix}
    }  % else (unstarred) use parethesis
    {
        \begin{pmatrix}%
            \IfNoValueTF{#2}{}{#2}
            \IfNoValueTF{#3}{}{\\#3}
            \IfNoValueTF{#4}{}{\\#4}
            \IfNoValueTF{#5}{}{\\#5}
            \IfNoValueTF{#6}{}{\\#6}
            \IfNoValueTF{#7}{}{\\#7}
            \IfNoValueTF{#8}{}{\\#8}
        \end{pmatrix}
    }
}
\def\Cov{\operatorname{Cov}} %Covariance
\def\df{\text{df}} %degrees of freedom

\NewDocumentCommand{\example}{ s g }{% Example header
    \IfBooleanTF{#1}%
    {\vspace{0.1in}}% starred: 0.1in
    {\vspace{0.2in}}% unstarred: 0.2in
    \IfNoValueTF{#2} {\noindent\textbf{\color{eblue} Example: }}{\noindent\textbf{\color{eblue} Example (#2): }}
}
\NewDocumentCommand{\disc}{ s }{% Discussion header
    \IfBooleanTF{#1}%
    {\vspace{0.1in}\noindent\textbf{Discussion: } }% starred: 0.1in
    {\vspace{0.2in}\noindent\textbf{Discussion: } }% unstarred: 0.2in
}
\NewDocumentCommand{\defn}{ s }{% Definition header
    \IfBooleanTF{#1}%
    {\vspace{0.1in}\noindent\textbf{\color{neonorange} Definition: } }% starred: 0.1in
    {\vspace{0.2in}\noindent\textbf{\color{neonorange} Definition: } }% unstarred: 0.2in
}
\NewDocumentCommand{\reason}{ s }{% Reason header
    \IfBooleanTF{#1}%
    {\vspace{0.1in}\noindent\textbf{Reason:} }% starred: 0.1in
    {\vspace{0.2in}\noindent\textbf{Reason:} }% unstarred: 0.2in
}
\NewDocumentCommand{\recall}{ s }{% Recall header
    \IfBooleanTF{#1}%
    {\vspace{0.1in}\noindent\textit{Recall:} }% starred: 0.1in
    {\vspace{0.2in}\noindent\textit{Recall:} }% unstarred: 0.2in
}
\NewDocumentCommand{\remark}{ s }{% Remark header
    \IfBooleanTF{#1}%
    {\vspace{0.1in}\noindent\textit{Remark:} }% starred: 0.1in
    {\vspace{0.2in}\noindent\textit{Remark:} }% unstarred: 0.2in
}

\NewDocumentCommand{\soln}{ s }{% Remark header
    \IfBooleanTF{#1}%
    {\vspace{0.1in}\noindent\textbf{Solution: } }% starred: 0.1in
    {\vspace{0.2in}\noindent\textbf{Solution: } }% unstarred: 0.2in
}

\newcommand{\proj}[2]{\operatorname{proj}_{{#1}}{#2}} %projection
\newcommand{\wideand}{\qquad \text{and} \qquad}

\newcommand{\bu}[1]{\textbf{\underline{{#1}}} } %bold underline
\newcommand{\boldit}[1]{\textbf{\textit{{#1}}} } %bold italix

% put actual quotation marks "around something"
\newcommand{\say}[1]{\textquotedblleft{#1}\textquotedblright}

% max{arg} and min{arg}
\renewcommand{\max}[1]{\operatorname{max}\left\{ #1 \right\}}
\renewcommand{\min}[1]{\operatorname{min}\left\{ #1 \right\}}

%Create a new vspace line no indent
\newcommand{\nl}{\vspace{0.1in}\noindent}
\newcommand{\nnl}{\vspace{0.2in}\noindent}
\newcommand{\nnnl}{\vspace{0.3in}\noindent}
\textwidth=7.02in
\hoffset=-.45in
\begin{document}

\pagestyle{fancy}
\fancyhf{}
\fancyhead[RO]{Matthew Wilder} %header top right
\fancyhead[LO]{MTH 307 - Homework \#13} %header top left
\fancyfoot[CO]{Page \thepage} %page center bottom

\noindent MATH 307 \\
Assignment \#13 \\  % 6.A, 6.B, 6.C
Due Monday, May 2$^{\text{nd}}$, 2022

\nl For each problem, include the statement of the problem. Leave a blank line.  At the beginning of the next line, write \textbf{Solution} or \textbf{Proof} -- as appropriate.

\begin{enumerate}
  \item Define $T \in \mathcal{L}(\mathbf{C}^2)$ by $T(w,z) =  (0,w)$.  Find all generalized eigenvectors of $T$.

\soln* Using the standard basis, $\mathcal{M}(T) = \begin{bmatrix}
    0 & 0 \\ 1 & 0
\end{bmatrix}$. For eigenvalue $\lambda_1 = 0$ we have $(0, 1)$ as an eigenvector. Then for $\mathcal{M}(T^2) = \begin{bmatrix}
    0 & 0 \\ 0 & 0
\end{bmatrix}$ with eigenvalue $\lambda_2 = 0$, we have $(1, 0)$ as an eigenvector. Then all the generalized eigenvectors of $T$ are $\operatorname{span}(e_1, e_2)$.\vspace{1in}
  \item Define $T \in\mathcal{L}(\mathbf{C}^2)$ by $T(w,z) = (z,-w)$. Find the generalized eigenspaces corresponding to the distinct eigenvalues of $T$. (Note Example 5.8 is an analogous transformation.)

\soln* Using the standard basis, $\mathcal{M}(T) = \begin{bmatrix}
    0 & 1 \\ -1 & 0
\end{bmatrix}$. So since $\lambda^2 + 1 = 0$ we have $\pm i$ as the eigenvalues. For $\lambda = i$, the corresponding eigenvector is $(w, -wi)$. For $\lambda = -i$, the corresponding eigenvector is $(w,wi)$. 

\nl Hence, $G(i, T) \equiv \operatorname{span}\{(1, -i)\}$ and $G(-i, T) \equiv \operatorname{span}\{(1, i)\}$.\vspace{1in}
  \item Suppose $T \in \mathcal{L}(V)$ and $\alpha , \beta \in \mathbf{F}$ with $\alpha \ne \beta$.  Prove that $G(\alpha, T) \cap G(\beta,T) = \{ 0 \}$.

\begin{proof} $ $
    By theorem 8.13 (Linearly independent generalized eigenvectors), since we have distinct $\alpha$ and $\beta$ by hypothesis, then $G(\alpha, T)$ and $G(\beta, T)$ are also form linearly independent subspaces. Then the only thing left in common is the zero vector. Hence $G(\alpha, T) \cap G(\beta,T) = \{ 0 \}$.  
\end{proof}\newpage
  \item Suppose that $T \in \mathcal{L}(\mathbf{C}^3)$ is defined by $T(z_1,z_2,z_3) = (z_2,z_3,0)$. Prove that $T$ has no square root.  More precisely, prove that there does not exist $S \in \mathcal{L}(\mathbf{C}^3)$ such that $S^2 = T$.

\begin{proof} (By contradiction)\\
    Suppose there exists a square root $S$ such that $S^2 = T$. Since $S^2 = T$, with the matrix of $T$ with respect to the standard basis being
    $$\mathcal M (T) = \begin{bmatrix}
        0 & 1 & 0 \\
        0 & 0 & 1 \\
        0 & 0 & 0 
    \end{bmatrix},$$
    Then by commutivity $TS = (S^2)S = S(S^2) = ST$. Denote $S$ by some matrix 
    $$\mathcal{M}(S) := \begin{bmatrix}
        a & b & c \\ d & e & f \\ g & h & j
    \end{bmatrix}.$$
    So 
    $$TS = \begin{bmatrix}
        d & e & f \\ g & h & j \\ 0 & 0 & 0
    \end{bmatrix} \wideand ST = \begin{bmatrix}
        0 & a & b \\ 0 & d & e \\ 0 & g & h
    \end{bmatrix}.$$
    As such, $d = g = h = 0$ by commutivity. Similarly, $$a = e = j := x_1 \wideand b = f := x_2.$$
    Substituting back into $S$,
    $$S := \begin{bmatrix}
        x_1 & x_2 & c \\ 0 & x_1 & x_2\\ 0 & 0 & x_1
    \end{bmatrix} \wideand S^2 = \begin{bmatrix}
        {x_1}^2 & 2x_1x_2 & 2cx_1 + {x_2}^2 \\ 0 & {x_1}^2 & 2x_1x_2 \\ 0 & 0 & {x_1}^2
    \end{bmatrix} \underbrace{=}_{\text{assumption}} \begin{bmatrix}
        0 & 1 & 0 \\
        0 & 0 & 1 \\
        0 & 0 & 0 
    \end{bmatrix} = T.$$
    By our assumption, $S^2 = T$, so all entries must match. In particular, ${x_1}^2$ must equal 0. As such, $S^2_{1,2} = 2x_1x_2 = 0 \neq T_{1,2}$. Hence a contradiction. 
\end{proof}\newpage
  \item Suppose that $T \in \mathcal{L}(V)$ is not nilpotent.  Let $n = \dim V$.  Show that \\ $V = \text{null }T^{n-1} \oplus \text{range }T^{n-1}$.

\begin{proof}
    Since $T$ is not nilpotent we know $\operatorname{null}T^n \oplus \range T^n = V$ by theorem. We know that $\operatorname{null}T \neq V$ by the subset chaining. (Since if it equalled $V$ then it would have to be nilpotent (which its not by hypothesis).) We claim that $\operatorname{null} T^{n-1} = \operatorname{null} T^n$ implies that $\dim \range T^{n-1} = \dim \range T^n$. \\Suppose that $\operatorname{null} T^{n-1} \neq \operatorname{null} T^n$. Then $\{0\} = \operatorname{null}I \subset \operatorname{null}T \subset \cdots \subset \operatorname{null} T^{n-1} \subset \operatorname{null}T^n$, which would imply that $T$ is nilpotent (since the sequence continues till $\dim V$). Hence a contradiction to the not-nilpotent hypothesis. That is, we know \textit{at least} $\operatorname{null} T^{n-1} = \null T^n$, and by rank-nullity theorem, $\range T^{n-1} = \range T^n$. Hence we can substitute into the known equation with 
    $$\underbrace{\operatorname{null}T^n}_{= \; \operatorname{null}T^{n-1}} \oplus \underbrace{\range T^n}_{= \; \range T^{n-1}} = V$$
    to obtain 
    $$V = \text{null }T^{n-1} \oplus \text{range }T^{n-1}.$$
\end{proof}\vspace{.5in}
  \item Suppose $T \in \mathcal{L}(V)$. Suppose $S \in \mathcal{L}(V)$ is invertible.  Prove that $T$ and $S^{-1}TS$ have the same eigenvalues with the same multiplicities.

\begin{proof}
    Since $T$ and $S^{-1}TS$ are similar matrices by change of bases. Then the characteristic polynomials of similar matrices are the same. Hence the eigenvalues are the same (and the multiplicities also).
\end{proof}\vspace{.5in}
  \item Suppose $V$ is a complex vector space and $T \in \mathcal{L}(V)$.  Prove that $V$ has a basis consisting of eigenvectors of $T$ if and only if every generalized eigenvector of $T$ is an eigenvector of $T$.

\begin{proof} ($\Longrightarrow$)
    Since we have a basis of eigenvectors ($\dim V$ of them) and generalized eigenvectors are a superset of an already spanning set, then the set of generalized eigenvectors is the same as the set of eigenvectors. Note that the number of linearly independent generalized eigenvectors cannot exceed $\dim V$.
\end{proof}

\begin{proof} ($\Longleftarrow$)
    Since by our assumption all of the generalized eigenvectors are equivalent to the eigenvectors. Theorem 8.23 says that we have a basis of generalized eigenvectors (which are assumed to equal the eigenvectors), hence we have a basis of eigenvectors.
\end{proof}\vspace{2.5in}
  \item Define $N \in \mathcal{L}(\mathbf{F}^5)$ by
    \[
       N(x_1,x_2,x_3,x_4,x_5) = (2x_2,3x_3,-x_4,4x_5,0).
    \]
    Find a square root of $I+N$.

    \soln* The matrix with respect to the standard basis for $N$ is 
    $$\mathcal{M}(N) = \begin{bmatrix}
       0 & 2 & 0 & 0 & 0 \\
       0 & 0 & 3 & 0 & 0 \\
       0 & 0 & 0 & -1 & 0 \\
       0 & 0 & 0 & 0 & 4 \\
       0 & 0 & 0 & 0 & 0
    \end{bmatrix}
    \wideand \mathcal{M}(I+N) = \begin{bmatrix}
      1 & 2 & 0 & 0 & 0 \\
      0 & 1 & 3 & 0 & 0 \\
      0 & 0 & 1 & -1 & 0 \\
      0 & 0 & 0 & 1 & 4 \\
      0 & 0 & 0 & 0 & 1
   \end{bmatrix}.$$

   \noindent We are looking for $R^2 = I+N$ for some square root $R$. Since $N$ is nilpotent, applying Theorem 8.31 we have 
   \begin{align*}
      & (I + a_1N + a_2N^2 + \cdots + a_{m-1}N^{m-1})\\
      \times \; & (I + a_1N + a_2N^2 + \cdots + a_{m-1}N^{m-1})\\
      \hline\\
      =\;&I + (2a_1N) + \brac{2a_2N^2 + a_1^2N^2} + \brac{
         2a_3N^3 + 2a_1a_2N^3
      } + \brac {
         2a_4 N^4 + 2a_1 a_3 N^4 + {a_2}^2 N^4
      } + \underbrace{ 0 + \cdots}_{\text{nilpotent}} 
   \end{align*}
   To make the right hand side of this equal zero, we want $I + (2a_1 N) = I + N$, hence $a_1 = \dfrac{1}{2}$. 
   \\ Then for the second term we want $(2a_2 + {a_1}^2) = 0 = 2a_2 + \dfrac{1}{4}$ which implies $a_2 = - \dfrac{1}{8}$.
   \\For the third term we want $2a_3 + 2a_1a_2 = 0 = 2a_3 - \dfrac{1}{8}$. Hence $a_3 = \dfrac{1}{16}$.
   \\Finally for the fourth term, we want $2a_4 + 2a_1a_3 + {a_2}^2 = 0 = 2a_4 + \over*{16} + \over*{64}$ so $a_4 = - \dfrac{5}{128}$.

   \nl Then $\sqrt{I+N} = I + \over*{2} N - \over*{8} N^2 + \over*{16} N^3 - \dfrac{5}{128}N^4$.\newpage
  \item Suppose $\mathbf{F} = \mathbf{C}$ and $T \in \mathcal{L}(V)$.  Prove that there exists $D, N \in  \mathcal{L}(V)$ such that $T = D + N$, the operator $D$ is diagonalizable, $N$ is nilpotent, and $DN = ND$.

\begin{proof}
    By Theorem 8.29 there exists a basis of $V$ where $T$ is a block diagonal matrix of the form 
    $$T = \begin{bmatrix}
        A_1 & & 0 \\ & \ddots & \\ 0 & & A_m
    \end{bmatrix} \wideand A_j = \begin{bmatrix}
        \lambda_j & & \ast \\ & \ddots & \\ 0 & & \lambda_j
    \end{bmatrix}$$
    Then we can decompose $A_j$ into 
    $$D_j = \begin{bmatrix}
        \lambda_1 & & 0 \\ & \ddots & \\ 0 & & \lambda_j
    \end{bmatrix} \wideand N_j = \begin{bmatrix}
        0 & & \ast \\ & \ddots & \\ 0 & & 0
    \end{bmatrix}$$
    Then commutivity clearly holds for $D_j$ and $N_j$ since
    $$D_jN_j = \lambda_j I N_j = \lambda_j N_j = N_j (\lambda_j) = N_j (\lambda I) = N_j D_j;$$
    Then to form $D$ and $N$ we have 
    $$D = \begin{bmatrix}
        \begin{pmatrix}
            \lambda_1 & & 0 \\
            & \ddots & \\
            0 & & \lambda_1
        \end{pmatrix}
        & & \\
        &         \ddots & \\
        & &         \begin{pmatrix}
            \lambda_m & & 0 \\
            & \ddots & \\
            0 & & \lambda_m
        \end{pmatrix}
    \end{bmatrix} \wideand 
    N = \begin{bmatrix}
        \begin{pmatrix}
            0 & & \ast \\
            & \ddots & \\
            0 & & 0
        \end{pmatrix}
        & & \\
        &         \ddots & \\
        & &         \begin{pmatrix}
            0 & & \ast \\
            & \ddots & \\
            0 & & 0
        \end{pmatrix}
    \end{bmatrix}.$$
    Clearly $T = D + N$. And for every $j \in \{1, \dots, m\}$ we have $D_j N_j = N_j D_j$, then $DN = ND$.
\end{proof}
\end{enumerate}
\end{document} 