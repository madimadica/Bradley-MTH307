\documentclass[12pt]{article}
\usepackage[english]{babel}
\usepackage{array}
\usepackage{setspace}
\usepackage{graphicx}
\usepackage{sistyle} %\num{100000} for commas
\SIthousandsep{,}
\usepackage{fancyhdr}
\usepackage{listings} % For code listings, may break stuff
\usepackage{xcolor, diagbox, empheq, makecell, tcolorbox}
\usepackage[autostyle]{csquotes}
\usepackage{amssymb, amsthm, linguex, enumitem, amsmath}
\usepackage{tcolorbox} %dont know what this does
\usepackage[colorlinks=true, allcolors=blue]{hyperref}
\usepackage{lipsum}

\makeatletter   %% <- make @ usable in macro names
\newcommand*\notab[1]{%
  \begingroup   %% <- limit scope of the following changes
    \par        %% <- start a new paragraph
    \@totalleftmargin=0pt \linewidth=\columnwidth
    %% ^^ let other commands know that the margins have been reset
    \parshape 0
    %% ^^ reset the margins
    #1\par      %% <- insert #1 and end this paragraph
  \endgroup
}
\makeatother    %% <- revert @

\pagestyle{empty}

\textwidth 6.5in
\hoffset=-.65in
\textheight=9.5in
\voffset=-1.in

%Sets
\newcommand{\R}{\mathbb{R}}
\newcommand{\C}{\mathbb{C}}
\newcommand{\N}{\mathbb{N}}
\newcommand{\F}{\mathbb{F}}
\newcommand{\Cb}{\mathbf{C}}
\newcommand{\Fb}{\mathbf{F}}
\newcommand{\Rb}{\mathbf{R}}

%Misc
\newcommand{\pars}[1]{\left( {#1} \right) } %auto size parenthesis 
\newcommand{\brac}[1]{\left[ {#1} \right] } %auto size brackets around arg
\newcommand{\set}[1]{\left\{{#1}\right\}} %auto size curly braces around arg
\newcommand{\vbrac}[1]{\left\langle{#1}\right\rangle} %vector angle brackets

\newcommand{\ceiling}[1]{\left\lceil {#1} \right\rceil} %auto size ceiling around arg
\newcommand{\floor}[1]{\left\lfloor {#1} \right\rfloor} %auto size floor around arg

\newcommand{\limn}{\lim_{n\to\infty}} %limit as n approaches infinity
\newcommand{\thru}[1]{{#1}_1, \dots, {#1}_n}
\newcommand{\sumthru}[1]{{#1}_1 + \cdots + {#1}_n}
\renewcommand{\over}[1]{\frac{1}{{#1}}}
\newcommand{\pfrac}[2]{\left( \frac{{#1}}{{#2}} \right) } %auto size parenthesis over fraction 
\newcommand{\pover}[1]{\left( \frac{1}{{#1}} \right) } %auto size parenthesis over fraction

%Boolean Algebra
\newcommand{\OR}{\,\lor\,}
\newcommand{\AND}{\,\land\,}

%Probability and Statistics
\newcommand{\xbar}{\bar{X}}
\newcommand{\ybar}{\bar{Y}}
\newcommand{\yn}{Y_1, \dots, Y_n}
\newcommand{\yx}{X_1, \dots, X_n}
\newcommand{\normDist}{N\left(\mu, \sigma^2\right)} %default normal distribution
\newcommand{\gammaDist}[2]{\operatorname{Gamma} \left( {#1},{#2} \right)}
\newcommand{\norm}[1]{N \left( {#1} \right)}
\newcommand{\prob}[1]{P \left( {#1} \right) }
\newcommand{\E}[1]{E \left( {#1} \right) }
\newcommand{\Eb}[1]{E[ \,{#1}\, ]} %E bracket
\newcommand*{\V}[1]{V \left( {#1} \right) }
\newcommand{\Vb}[1]{V [ \,{#1}\, ] }
\newcommand{\that}{\hat{\theta}} %theta hat
\newcommand{\phat}{\hat{p}}
\newcommand{\psihat}{\hat{\psi}}
\newcommand{\Psihat}{\hat{\Psi}}

%Linear Algebra
\newcommand{\poly}[1]{\mathcal{P}_{#1}(\mathbf{R})} %polynomial up to degree (arg)
\newcommand{\pf}{\mathcal{P}(\mathbf{F})} %set of all polynomials
\newcommand{\Lc}{\mathcal{L}} %Set of linear maps
\newcommand{\vdub}[2]{\begin{pmatrix}{#1}\\{#2}\end{pmatrix}} %vertical vector parenthesis, 3 args
\newcommand{\vtrip}[3]{\begin{pmatrix}{#1}\\{#2}\\{#3}\end{pmatrix}} %vertical vector parenthesis, 3 args
\newcommand{\vquad}[4]{\begin{pmatrix}{#1}\\{#2}\\{#3}\\{#4}\end{pmatrix}} %vertical vector parenthesis, 4 args
\newcommand{\vquin}[5]{\begin{pmatrix}{#1}\\{#2}\\{#3}\\{#4}\\{#5}\end{pmatrix}} %vertical vector parenthesis, 5 args
\newcommand{\vvector}[3]{\begin{bmatrix}{#1}\\{#2}\\{#3}\end{bmatrix}} %vertical vector braces, 3 args
\newcommand{\dimn}[1]{\operatorname{dim}\,{#1}}
\newcommand{\rank}[1]{\operatorname{dim}\operatorname{range}{#1}}
\newcommand{\nullity}[1]{\operatorname{dim}\operatorname{null}{#1}}
\newcommand{\matx}[4]{\begin{bmatrix}{#1} & {#2}\\{#3}&{#4}\end{bmatrix}} % 2x2 matrix
\newcommand{\detx}[4]{\begin{vmatrix}{#1} & {#2}\\{#3}&{#4}\end{vmatrix}} % 2x2 determinant
\newcommand{\nf}{\infty}
%colors
\definecolor{ggreen}{RGB}{0, 127, 0}
\definecolor{dgray}{RGB}{63,63,63}
\definecolor{neonorange}{RGB}{255,47,0}
\definecolor{mygray}{rgb}{0.5,0.5,0.5}
\newcommand{\red}[1]{\color{red}{#1}\color{black}}
\newcommand{\grn}[1]{\color{ggreen}{#1}\color{black}}
\newcommand{\blu}[1]{\color{blue}{#1}\color{black}}
\newcommand{\redx}[1]{\color{red}\not{#1}\color{black}}

\newcommand{\say}[1]{\textquotedblleft{#1}\textquotedblright} %quote the "argument"
\newcommand*\widefbox[1]{\fbox{\hspace{2em}#1\hspace{2em}}}
\newtcolorbox{mybox}[1][]{colback=white, sharp corners, #1}

%Line break spacings
\newcommand{\nl}{\vspace{0.1in}\noindent}
\newcommand{\nnl}{\vspace{0.2in}\noindent}
\newcommand{\nnnl}{\vspace{0.3in}\noindent}

% Code snippets
\newcommand*{\code}{\fontfamily{qcr}\selectfont}
\lstset{
    backgroundcolor=\color{white},
    basicstyle=\footnotesize,
    breakatwhitespace=false,         % sets if automatic breaks should only happen at whitespace
    breaklines=true,                 % sets automatic line breaking
    captionpos=b,                    % sets the caption-position to bottom
    commentstyle=\color{dgray},    % comment style
    deletekeywords={...},            % if you want to delete keywords from the given language
    escapeinside={(*@}{@*)},          % if you want to add LaTeX within your code
    extendedchars=true,              % lets you use non-ASCII characters; for 8-bits encodings only, does not work with UTF-8
    firstnumber=1,                % start line enumeration with line 1
    frame=single,	                   % adds a frame around the code
    keepspaces=true,                 % keeps spaces in text, useful for keeping indentation of code (possibly needs columns=flexible)
    keywordstyle=\color{neonorange},       % keyword style
    language=C++,                 % the language of the code
    morekeywords={*,...},            % if you want to add more keywords to the set
    numbers=left,                    % where to put the line-numbers; possible values are (none, left, right)
    numbersep=5pt,                   % how far the line-numbers are from the code
    numberstyle=\tiny\color{mygray}, % the style that is used for the line-numbers
    rulecolor=\color{black},         % if not set, the frame-color may be changed on line-breaks within not-black text (e.g. comments (green here))
    showspaces=false,                % show spaces everywhere adding particular underscores; it overrides 'showstringspaces'
    showstringspaces=false,          % underline spaces within strings only
    showtabs=false,                  % show tabs within strings adding particular underscores
    stringstyle=\color{purple},     % string literal style
    tabsize=4,	                   % sets default tabsize to 4 spaces
}

\lstdefinestyle{cpp}{language=C++,
    morekeywords={cout, cin, Comparable, T},numbers=none
}
%Examples:
%{\code while}
%
%{\code \begin{lstlisting}[language=C++]
%sum1 = 0;
%for (i = 1; i <= n; i *= 2)
%    for (j = 1; j <= n; j++)
%        sum1++;
%\end{lstlisting}}







\begin{document}

\pagestyle{fancy}
\fancyhf{}
\fancyhead[RO]{Matthew Wilder} %header top right
\fancyhead[LO]{MTH 307 - Homework \#6} %header top left
\fancyfoot[CO]{Page \thepage} %page center bottom

\noindent MTH 307 - Spring 2022
\\Assignment \#6
\\Due: Friday, February 25, 2022 (4pm)\\

\nl For each problem, include the statement of the problem. Leave a blank line.  At the beginning of the next line, write \textbf{Solution} or \textbf{Proof} -- as appropriate.

\begin{enumerate}
    \item Label the following statements as being true or false.
Provide some justification from the text for your label.
\begin{enumerate}
    \item Every linear operator on an $n$-dimensional vector space has $n$ distinct eigenvalues.
    \begin{mybox}
        \textbf{Solution - False: } Some eigenvalues can have a multiplicity greater than 1, such as the identity matrix $I_2$ has the eigenvalues $\lambda = 1$ and $\lambda = 1$. Thus, it has a single \textit{distinct} eigenvalue $\lambda = 1$ with multiplicity 2.
    \end{mybox}



    \item If a linear operator on a vector space over $\mathbf{R}$ has one eigenvector, then it has an infinite number of eigenvectors.
    \begin{mybox}
        \textbf{Solution - True: } Let $v$ be an eigenvector, then any non-zero vector in $\operatorname{span}\{v\}$ is also an eigenvector. (All non-zero scalar multiples of $v$.) There are uncountably infinite non-zero scalars, $c$, in $\R$ or $\C$ such that $cv \in \operatorname{span}\{v\}$, therefore the statement is true. 
    \end{mybox}
%3
\newpage
    \item There exists a square matrix with no eigenvectors.
    Eigenvalues must be nonzero scalars.
    \begin{mybox}
        \textbf{Solution - True: } Example. Let $T = \brac{
            \begin{array}{cc}
              0 & -1  \\
              1 & 0 \\
            \end{array}}$ in the field $\R$. Then for some $x,y \in \R$ such that $x \neq 0 \text{ or } y \neq 0$, and for a vector  $v = \vbrac{x,y}$, $\,(T-\lambda I_2)v = \brac{ \begin{array}{cc}
            -\lambda & -1  \\
            1 & -\lambda \\
          \end{array}} \vdub{x}{y} = 0.$ Then we have the following system of equations:
          $$-\lambda x - y = 0 \hspace{1in} x - \lambda y = 0.$$
          The second equation can be rewritten as $x = \lambda y$ and substituting into the first we get
          \begin{align*}
            -\lambda(\lambda y) - y = 0
            \iff & \quad -\lambda^2 y - y = 0 \\
            \iff & \quad -\lambda^2 = 1\\
            \iff & \quad \lambda = \sqrt{-1} \not \in \R.
          \end{align*}
          Thus the first equation yields zero eigenvalues and thus zero eigenvectors. Next, the first equation can be rewritten as $y = -\lambda x$. Substituting that value into the second equation, we get
          \begin{align*}
            x=\lambda y
            \iff & \quad x = -\lambda^2 x \\
            \iff & \quad 1 = -\lambda^2\\
            \iff & \quad \lambda = \sqrt{-1} \not \in \R.
          \end{align*}
          Therefore neither equation yields an eigenvalue or eigenvector and hence we have an example of a square matrix with no eigenvectors.
    \end{mybox}


    \item Eigenvalues must be nonzero scalars.
    \begin{mybox}
        \textbf{Solution - False: } Counterexample. Let $T = \left(
            \begin{array}{cc}
              1 & 0  \\
              0 & 0 \\
            \end{array}
          \right)$ in the field $\R$. Then for some $x,y \in \R$ such that $x \neq 0 \text{ or } y \neq 0$, and for a vector  $v = \vbrac{x,y}$, $\,(T-\lambda I_2)v = \brac{ \begin{array}{cc}
            1 -\lambda & 0  \\
            0 & -\lambda \\
          \end{array}} \vdub{x}{y} = 0.$ Then we have the following system of equations:
          $$(1 -\lambda) x = 0 \hspace{1in} - \lambda y = 0.$$
          Looking at the second equation, the only way it is true is if $\lambda = 0$, therefore we have a contradiction that there exists no non-zero eigenvalue.
    \end{mybox}

    \newpage
    \item Any two eigenvectors are linearly independent.
    \begin{mybox}
        \textbf{Solution - False: } Taking the matrix setup from (d), we have that $\lambda = 0$ is an eigenvalue. Computing an eigenvector, we first have
        $$\matx{1}{0}{0}{0} \vdub{x}{y} = \lambda v = 0v.$$
        From here we can see that $\vdub{0}{1}$ is a valid eigenvector for $\lambda = 0$ since $Tv = 0v$ holds true. Additionally, $\vdub{0}{2}$ would also be an eigenvector (or any non-zero scalar). But these are linearly dependent, hence a contradiction.
    \end{mybox}



    \item The sum of two eigenvalues of a linear operator $T$ is also an eigenvalue of $T$.
    \begin{mybox}
        \textbf{Solution - False: } Consider the matrix $\matx{1}{0}{0}{2}.$ It has eigenvalues $\lambda = 1$ and $\lambda = 2$ by the triangular matrix diagonal property. But $2 + 1 = 3$ is not an eigenvalue since we can only have 2 eigenvalues and we already showed them to be 1 and 2.
    \end{mybox}

    \item Linear operators on infinite-dimensional vectors spaces never have eigenvalues.
    \begin{mybox}
        Define $T \in \Lc (\R^{\infty})$ by
        $$T(x_1, x_2, \dots) = (x_1, 0, \dots).$$
        Then $T$ has an eigenvalue $\lambda = 1$ with eigenspace of $\operatorname{span}(1, 0, \dots)$. Hence, a contradiction.
    \end{mybox}

    \item The sum of two eigenvectors of an operator $T$ is always an eigenvector of $T$.
    \begin{mybox}
        \textbf{Solution - False: } Using the same setup for $T$ as in part (d), we have $\lambda = 1$ and $\lambda = 2$ as the 2 distinct eigenvalues with $\vdub{1}{0}$ and $\vdub{0}{1}$ as respective eigenvectors. But the sum, $\vdub{1}{0} + \vdub{0}{1} = \vdub{1}{1}$ is not in either eigenspace. Hence a contradiction to the assumption.
    \end{mybox}

\end{enumerate}
    \newpage
    \item Consider the operator
$T = \left(
        \begin{array}{cc}
          0 & 0  \\
          0 & 1 \\
        \end{array}
      \right)$
acting on $\mathbf{R}^2$.  How many subspaces are there that are invariant under $T$?
\begin{mybox}
  \textbf{Solution: } First, the trival subspaces of $\operatorname{span}\{0\}$ and $\R^2$ hold true. Then using the triangular matrix property to get the eigenvalues $\lambda = 0$ and $\lambda = 1$, we can compute eigenvectors for them. 

  \nl For $\lambda = 0$, 
  $$\matx{1}{0}{0}{0} \vdub{x}{y} =  \vdub{1x}{0y} = 0v = 0 \quad \text{ implies }  \vdub{x}{y} \in \operatorname{span}(\vbrac{0,1}).$$
  For $\lambda = 1$, 
  $$\matx{1}{0}{0}{0} \vdub{x}{y} = \vdub{1x}{0y} = 1v = v \quad \text{ implies }  \vdub{x}{y} \in \operatorname{span}(\vbrac{1,0}).$$
  So there are 4 distinct invariant subspaces,
  $$\operatorname{span}(\vec{0}) \qquad \operatorname{span}(\vbrac{0,1}) \qquad \operatorname{span}(\vbrac{1,0}) \qquad \R^2$$
\end{mybox}
    
    \item If $U$ and $W$ are invariant subspaces for $T \in \mathcal{L}(V)$ then $U+W$ is invariant for $T$.
\begin{mybox}
\begin{proof}
    Let $u \in U$ and $w \in W$. Then $T(u + w) = T(u) + T(w)$ by linearity. 

    \nl By our assumption of invariance, $T(u) \in U$ and $T(w) \in W$. Therefore\\$T(u) + T(w) \in U + W$ by the definition of Sum of Subspaces. This implies $T(u + w) \in U + W$ by linearity. Hence, $U + W$ is invariant under $T$.
\end{proof}
\end{mybox}
    
    \item In $\mathbf{R}^2$, let $T$ be the reflection across the line $y=x$.
\begin{enumerate}
\item Write the matrix $A$ that represents $T$ relative to the standard basis.
\begin{mybox}
        \textbf{Solution:}
        $$T : \R^2 \to \R^2$$
        $$T(x,y) = (y,x)$$
        Computing the images of the standard basis,
        $$T(e_1) = (0,1) \qquad\text{and}\qquad T(e_2) = (1,0).$$
        Therefore,
        $$A = \matx{0}{1}{1}{0}.$$
\end{mybox}

\newpage
\item Determine two invariant subspaces $\mathcal{M}$ and $\mathcal{N}$ for $T$ such that $\mathbf{R}^2 = \mathcal{M} \oplus \mathcal{N}$ where neither $\mathcal{M}$ nor $\mathcal{N}$ is the zero subspace.
\begin{mybox}
\textbf{Solution: } Let $v \in \R^2$ denoted by $v = \vbrac{x,y}$ such that $v \neq \vec{0}$, then
\begin{align*}
        (A-\lambda I_2)v = \vec{0} &\iff \matx{-\lambda}{1}{1}{-\lambda} \vdub{x}{y} = \vdub{0}{0}\\
        &\iff -\lambda x + y = 0 \quad \text{and} \quad x - \lambda y = 0\\
        & \iff x = \lambda y  \quad \text{and} \quad  y = \lambda x\\
        & \iff x = \lambda^2 x  \quad \text{and} \quad y = \lambda^2 y\\
        & \iff \lambda = \pm 1.
\end{align*}
For $\lambda = 1$, $Av = 1v$ implies that $y=x$ and $x=y$, so we get an eigenvectors of $\vdub{1}{1}$.

\nl For $\lambda = -1$, $Av = -v$ implies that $y=-x$ and $x=-y$, so we get an eigenvectors of $\vdub{-1}{1}$.

\nl Therefore we can let $\mathcal{M} = \operatorname{span}\vdub{1}{1}$ and let $\mathcal{N} = \operatorname{span}\vdub{-1}{1}$.

\nl These are linearly independent and invariant under $T$ because $\mathcal{M} \cap \mathcal{N} = \vec{0}$. Therefore, $\R^2 = \mathcal{M} \oplus \mathcal{N}$.
\end{mybox}

\item Write a basis $\{ u_1, u_2 \}$ for $\mathbf{R}^2$ so that $\mathcal{M} = \text{span } (u_1)$ and  $\mathcal{N} = \text{span } (u_2)$.
\begin{mybox}
       \textbf{Solution:} Using the conclusion of part (b), we can let the basis be $$\left\{\vdub{1}{1}, \; \vdub{-1}{1}  \right\}.$$
\end{mybox}

\item Write the matrix $B$ that represents $T$ relative to the basis $\{ u_1, u_2 \}$.
\begin{mybox}
        \textbf{Solution: } 
        $$T\vdub{1}{1} = \vdub{1}{1} \qquad \text{and} \qquad T\vdub{-1}{1} = \vdub{1}{-1}$$
        Using these as the columns in matrix B,
        $$B = \matx{1}{1}{1}{-1}.$$
\end{mybox}
\end{enumerate}
    \newpage
    \item \begin{enumerate}
    \item Let $V = \mathbf{R}^2$. Find eigenvalues and eigenvectors for the linear operator $T$ defined by $T(x,y) = (2y , x)$.
    \begin{mybox}
        Let $A$ be a transformation matrix for T, then
        $$A = \brac{T(e_1) \quad T(e_2)} = \matx{0}{2}{1}{0}.$$

        \nl Let $v \in \R^2$ denoted by $v = \vbrac{x,y}$ such that $v \neq \vec{0}$, then
        \begin{align*}
                (A-\lambda I_2)v = \vec{0} &\iff \matx{-\lambda}{2}{1}{-\lambda} \vdub{x}{y} = \vdub{0}{0}\\
                &\iff -\lambda x + 2y = 0 \quad \text{and} \quad x - \lambda y = 0\\
                & \iff 2y = \lambda x  \quad \text{and} \quad x = \lambda y\\
                & \iff 2y = \lambda^2 y  \quad \text{and} \quad x = \frac{\lambda^2x}{2}\\
                & \iff 2 = \lambda^2  \quad \text{and} \quad 1 = \frac{\lambda^2}{2}\\
                & \iff \lambda = \pm \sqrt{2}.
        \end{align*}
        Therefore the eigenvalues are $\lambda = \sqrt{2}$ and $\lambda = -\sqrt{2}$. 

        \nnl Computing eigenvectors for $\lambda = \sqrt{2}$, 
        \begin{align*}
            Av = \lambda v & \iff \matx{0}{2}{1}{0} \vdub{x}{y} = \vdub{x\sqrt{2}}{y\sqrt{2}}\\
            &\iff 2y = x\sqrt{2} \quad \text{and} \quad x = y\sqrt{2}
        \end{align*}
        Which we can therefore conclude an eigenvector $\vdub{2}{\sqrt{2}}$ for $\lambda = \sqrt{2}$.

        \nl Computing eigenvectors for $\lambda = -\sqrt{2}$, 
        \begin{align*}
            Av = \lambda v & \iff \matx0210 \vdub{x}{y} = \vdub{-x\sqrt{2}}{-y\sqrt{2}}\\
            &\iff 2y = -x\sqrt{2} \quad \text{and} \quad x = -y\sqrt{2}
        \end{align*}
        Which we can therefore conclude an eigenvector $\vdub{2}{-\sqrt{2}}$ for $\lambda = -\sqrt{2}$.
    \end{mybox}










\newpage
    \item Let $V = \mathbf{R}^2$. Find eigenvalues and eigenvectors for the linear operator $T$ defined by $T(x,y) = (-2y , x)$.
    \begin{mybox}
        Let $A$ be a transformation matrix for T, then
        $$A = \brac{T(e_1) \quad T(e_2)} = \matx{0}{-2}{1}{0}.$$

        \nl Let $v \in \R^2$ denoted by $v = \vbrac{x,y}$ such that $v \neq \vec{0}$, then
        \begin{align*}
                (A-\lambda I_2)v = \vec{0} &\iff \matx{-\lambda}{-2}{1}{-\lambda} \vdub{x}{y} = \vdub{0}{0}\\
                &\iff -\lambda x - 2y = 0 \quad \text{and} \quad x - \lambda y = 0\\
                & \iff 2y = -x\lambda  \quad \text{and} \quad x = \lambda y\\
                & \iff 2y = -\lambda^2 y  \quad \text{and} \quad x = -\frac{\lambda^2x}{2}\\
                & \iff 2 = -\lambda^2  \quad \text{and} \quad 1 = -\frac{\lambda^2}{2}\\
                & \iff \lambda = -\pm \sqrt{2}\\
                & \iff \lambda = \pm \sqrt{2}.
        \end{align*}
        Therefore the eigenvalues are $\lambda = \sqrt{2}$ and $\lambda = -\sqrt{2}$. 

        \nnl Computing eigenvectors for $\lambda = \sqrt{2}$, 
        \begin{align*}
            Av = \lambda v & \iff \matx{0}{2}{1}{0} \vdub{x}{y} = \vdub{x\sqrt{2}}{y\sqrt{2}}\\
            &\iff 2y = x\sqrt{2} \quad \text{and} \quad x = y\sqrt{2}
        \end{align*}
        Which we can therefore conclude an eigenvector $\vdub{2}{\sqrt{2}}$ for $\lambda = \sqrt{2}$.

        \nl Computing eigenvectors for $\lambda = -\sqrt{2}$, 
        \begin{align*}
            Av = \lambda v & \iff \matx0210 \vdub{x}{y} = \vdub{-x\sqrt{2}}{-y\sqrt{2}}\\
            &\iff 2y = -x\sqrt{2} \quad \text{and} \quad x = -y\sqrt{2}
        \end{align*}
        Which we can therefore conclude an eigenvector $\vdub{2}{-\sqrt{2}}$ for $\lambda = -\sqrt{2}$.
    \end{mybox}

    \end{enumerate}
    
    \newpage
    \item Let $T \in \mathcal{L}(V)$ and $S \in \mathcal{L}(V)$ be invertible.
        \begin{enumerate}
        \item Show that $T$ and $S^{-1}TS$ have the same eigenvalues.
        \begin{mybox}
        \begin{proof}
                Assume that $\lambda$ is the eigenvalue of $T$. By the definition of eigenvalue, for some vector $v \in V$, we have $Tv = \lambda v$. And, again, the the definition of eigenvalue, we will show that $S^{-1}TSv = \lambda v$.
                \begin{align*}
                        S^{-1}TSv = \lambda v & \iff  S^{-1}T(SS^{-1})v = S^{-1} \lambda v & \text{Multiply } S^{-1}\\
                        &\iff S^{-1}Tv =  S^{-1} \lambda v  & \text{Identity, } SS^{-1} = I\\
                        & \iff S^{-1} \lambda v =  S^{-1} \lambda v & \text{Eigenvalue of } T\\
                        & \iff \lambda S^{-1}  v =  \lambda S^{-1}  v & \text{Linearity of } S\\
                        & \iff \lambda S^{-1}S  v =   \lambda S^{-1}Sv & \text{Multiply } S\\
                        &\iff \lambda v = \lambda v & \text{Identity}\\
                        &\iff \lambda v = Tv & \text{Substitute }Tv
                \end{align*} 
        \end{proof}
        \end{mybox}

        \item What is the relationship between the eigenvectors of $T$ and the eigenvectors of $S^{-1}TS$.
        \begin{mybox}
                We can manipulate the equation as follows,
                \begin{align*}
                        S^{-1}TSv = \lambda v & \iff SS^{-1}TSv = S \lambda v & \text{Multiply }S\\
                        &\iff TSv = \lambda S v & \text{Identity, linearity.}
                \end{align*} 
                Which, by the definition of eigenvector, $Tv = \lambda v$, if we substitute in $v$ with $Sv$ to the above manipulation, we get that $Sv$ is an \textit{eigenvector} of $T$.
        \end{mybox}
        \end{enumerate}
        
    \newpage
    \item Show that the operator $T \in \mathcal{L}(\mathbf{C}^\infty)$ defined by
        \[
        T(z_1,z_2, \ldots) = (0,z_1,z_2, \ldots)
        \]
        has no eigenvalues.
        \begin{mybox}
                \begin{proof}
                        Let $v \in \C^{\infty}$ such that $v\neq \vec0$. Denote $v$ by $v = (z_1,z_2, \ldots)$ for some $z_i \in \C$.  By the definition of an eigenvalue, we would have
                        \begin{align*}
                                Tv &= \lambda v & \text{Definition of eigenvalue.}\\
                                T(z_1, z_2, \dots)  &= \lambda(z_1, z_2, \dots)& \text{Substitute in for }v.\\
                                (0, z_1, \dots) &= (\lambda z_1, \lambda z_2, \dots) & \text{Definition of } T \text{ and distribution of }\lambda. 
                        \end{align*}
                        Which implies that $\lambda z_1 = 0$ and $z_i = \lambda z_{i+1}$ (for index $i \geq 1$). Therefore $\lambda = 0$ or $\lambda \neq 0$.

                        \nl Case 1: $\lambda = 0$. Means that $z_1 = 0$ and by $z_i = \lambda z_{i+1}$ we would have \\$0 = z_1 = z_2 = \dots$. This is a contradiction to the eigenvalue assumption that $v \neq \vec0$. 

                        \nnl Case 2: $\lambda \neq 0$. Means that $z_1 \neq 0$ and by $z_i = \lambda z_{i+1}$ we would have \\$0 = z_1 = z_2 = \dots$. This is a contradiction to the assumption that $z_1 \neq 0$. 
                \end{proof}
        \end{mybox}
    \newpage
    \item Suppose $T \in \mathcal{L}(V)$ and there exists nonzero vectors $v$ and $w$ so that
        \[
            Tv = 3w \hspace{.5in} \text{and} \hspace{.5in} Tw=3v.
        \]
        Prove that 3 or $-3$ is an eigenvalue of $T$.
\begin{mybox}
\begin{proof}
    Because $T$ is linear, for some $v,w \in V,\; T(v+w) = T(v) + T(w)$.

    \begin{align*}
        T(v+w) &= T(v) + T(w)\qquad \text{Linearity.}\\
        &= 3w + 3v \qquad \text{Given.}\\
        &= 3(w+v) \qquad \text{Factor out 3.}\\
        &= 3(v+w) \qquad \text{Commutativity on } V
    \end{align*}
    Therefore $3$ is an eigenvalue since $T(v+w) = 3(v+w)$.

    \nnl For the next eigenvalue, we will take a look at $T(v-w)$.
    \begin{align*}
        T(v-w) &= T(v) - T(w)\qquad \text{Linearity.}\\
        &= 3w - 3v \qquad\qquad \text{Given.}\\
        &= -3(-w+v) \qquad \text{Factor out 3.}\\
        &= 3(v-w) \qquad\qquad \text{Commutativity on } V
    \end{align*}
    Therefore $-3$ is an eigenvalue since $T(v-w) = -3(v-w)$.
\end{proof}
\end{mybox}
    
    \end{enumerate}
\end{document} 