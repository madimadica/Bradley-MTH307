The general operation of finding an antiderivative is not a linear map because of the ``$+C$'' which means that any function has infinitely many antiderivatives.  Let's define a linear map from $\mathcal{P}_3(\mathbf{R})$ to $\mathcal{P}_4(\mathbf{R})$ that avoids ambiguity.  Let $A(a_0 + a_1x + a_2x^2 + a_3 x^3) = a_0x + (a_1/2)x^2 + (a_2/3)x^3 + (a_3/4) x^4$.
        \begin{enumerate}
        \item Find the matrix of $A$ with respect to the standard bases for $\mathcal{P}_3(\mathbf{R})$ and $\mathcal{P}_4(\mathbf{R})$.\vspace{0.4in}
        \begin{mybox}
        \textbf{Solution: } Using the definition of antiderivatives, we need to write the antiderivative of each element in the standard basis of $\poly{3}$ as a linear combination of the standard basis of $\poly{4}$. Thus,
        \begin{align}
            A(x^0) &= \frac{x^1}{1} = 0(1) + 1(x) + 0(x^2) + 0(x^3) + 0(x^4).\notag\\
            A(x^1) &= \frac{x^2}{2} = 0(1) + 0(x) + \frac{1}{2}(x^2) + 0(x^3) + 0(x^4).\notag\\
            A(x^2) &= \frac{x^3}{3} = 0(1) + 0(x) + 0(x^2) + \frac{1}{3}(x^3) + 0(x^4).\notag\\
            A(x^3) &= \frac{x^4}{4} = 0(1) + 0(x) + 0(x^2) + 0(x^3) + \frac{1}{4}(x^4).\notag
        \end{align}
        Turning the coefficients on the right hand side of the four bases into column vectors, we have
        $$\vquin{0}{1}{0}{0}{0},\; \vquin{0}{0}{1/2}{0}{0},\; \vquin{0}{0}{0}{1/3}{0},\; \vquin{0}{0}{0}{0}{1/4}.$$
        Turning these into the columns of a 5x4 matrix, $A$, we get
        $$A = \begin{bmatrix}
            0 & 0 & 0 & 0\\
            1 & 0 & 0 & 0\\
            0 & 1/2 & 0 & 0\\
            0 & 0 & 1/3 & 0\\
            0 & 0 & 0 & 1/4\\
        \end{bmatrix}.
            $$

        \end{mybox}\newpage
        \item Find new bases for $\mathcal{P}_3(\mathbf{R})$ and $\mathcal{P}_4(\mathbf{R})$ so that the matrix for $A$ with respect to the new bases is
            \[
        \left(
          \begin{array}{cccc}
            1 & 0 & 0 & 0 \\
            0 & 1 & 0 & 0 \\
            0 & 0 & 1 & 0 \\
            0 & 0 & 0 & 1 \\
            0 & 0 & 0 & 0 \\
          \end{array}
        \right).
        \]\vspace{0.4in}
        \begin{mybox}
            \textbf{Solution: } We essentially need to reverse the methodology from part (a). We will fix our basis of $\poly{3}$ to be $\{1,x,x^2,x^3\}$ and the basis of $\poly{4}$ to be some $\{a,b,c,d,e\}$. Using its columns of $A$ as coefficients, we need
            \begin{align}
                A(1) &= \frac{x^1}{1} = 1(a) + 0(b) + 0(c) + 0(d) + 0(e).\notag\\
                A(x^1) &= \frac{x^2}{2} = 0(a) + 1(b) + 0(c) + 0(d) + 0(e).\notag\\
                A(x^2) &= \frac{x^3}{3} = 0(a) + 0(b) + 1(c) + 0(d) + 0(e).\notag\\
                A(x^3) &= \frac{x^4}{4} = 0(a) + 0(b) + 0(c) + 1(d) + 0(e).\notag
            \end{align}
            From this, we can see that
            \begin{align}
                a &= x^1,\notag\\
                b &= x^2/2,\notag\\
                c &= x^3/3,\text{ and}\notag\\
                d &= x^4/4.\notag
            \end{align}
            And our remaining term, $e$, in order to complete the basis of $\poly{4}$, must be a constant term. Therefore we'll let $e = \pi$.
            
            \nl Thus, we have a fixed and derived basis,
            \begin{align}
                \poly{3} &= \{1,\;x,\;x^2,\;x^3\} \quad \text{ and}\notag\\
                \poly{4} &= \left\{x,\;\frac{x^2}{2},\;\frac{x^3}{3},\;\frac{x^4}{4}, \;\pi\right\}.\notag
            \end{align}
        \end{mybox}
        \end{enumerate}