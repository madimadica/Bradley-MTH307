\documentclass[12pt]{article}
\usepackage{setspace}
\usepackage{graphicx}
\usepackage{amsmath}
\usepackage{amssymb}
\usepackage{fancyhdr}
\usepackage{setspace}
\usepackage{graphicx}
\usepackage{amsmath}
\usepackage{amssymb}
\usepackage{fancyhdr}
\usepackage{xcolor}
\usepackage{diagbox}
\usepackage{empheq}
\usepackage{makecell}
\usepackage[autostyle]{csquotes}
\usepackage{amssymb, amsthm, linguex, enumerate, amsmath}
\usepackage{amsmath}
\usepackage{graphicx}
\usepackage{enumitem}
\usepackage{xcolor}
\usepackage[colorlinks=true, allcolors=blue]{hyperref}
\usepackage{fancyhdr}
\usepackage{tcolorbox}
%\usepackage{mathabx}
\pagestyle{empty}
\usepackage{amssymb, amsthm, linguex, enumerate, amsmath}
\newcommand{\pf}{\mathcal{P}(\mathbf{F})}
\newcommand{\R}{\mathbb{R}}
\newcommand{\C}{\mathbb{C}}
\newcommand{\Cb}{\mathbb{C}}
\newcommand{\F}{\mathbb{F}}
\newcommand{\Fb}{\mathbf{F}}
\newcommand{\Rb}{\mathbf{R}}
\newcommand{\x}{x \in \mathbb{R}}
\newcommand{\nat}{n \in \mathbb{N}}
\newcommand{\dx}{\frac{d}{dx}}
\newcommand{\dxof}[1]{\frac{d}{dx} \left( {#1} \right) }
\renewcommand{\vector}[1]{\left\langle{#1}\right\rangle}
\newcommand{\pars}[1]{\left( {#1} \right) }
\newcommand{\brac}[1]{\left[ {#1} \right] }
\newcommand{\limit}[3]{\lim_{{#1}\to {#2}} {#3}}
\newcommand{\xo}{x_0}
\newcommand{\iid}{independent identically distributed }
\newcommand{\dble}{differentiable }
\newcommand{\xbar}{\bar{X}}
\newcommand{\ybar}{\bar{Y}}
\newcommand{\OR}{\,\lor\,}
\newcommand{\AND}{\,\land\,}
\newcommand{\rv}{random variable }
\newcommand{\distn}{distribution }
\newcommand{\cont}{continuous }
\newcommand{\then}{\,\Rightarrow\,}
\renewcommand{\emptyset}{\{\,\}}
\newcommand{\disc}{discontinuous }
\newcommand{\forx}{\qquad \text{for all } x}
\newcommand{\cbi}{closed bounded interval }
\newcommand{\seq}[1]{\{{#1}\}}
\newcommand{\limn}{\lim_{n\to\infty}}
\renewcommand{\f}[1]{f^{({#1})}}
\renewcommand{\sup}[1]{\text{sup}\left\{ {#1} \right\}}
\renewcommand{\inf}[1]{\text{inf}\left\{ {#1} \right\}}
\newcommand{\N}[2]{N \left( {#1},{#2} \right)}
\newcommand{\gammaDist}[2]{\gamma \left( {#1},{#2} \right)}
\newcommand{\Ndef}{N\left(\mu, \sigma^2\right)} %default normal
\newcommand{\thru}[1]{{#1}_1, \dots, {#1}_n}
\newcommand{\yn}{Y_1, \dots, Y_n}
\renewcommand{\max}[1]{\text{max}\left\{ {#1} \right\}}
\renewcommand{\min}[1]{\text{min}\left\{ {#1} \right\}}
\renewcommand{\over}[1]{\frac{1}{{#1}}}
\newcommand{\poly}[1]{\mathcal{P}_{#1}(\mathbf{R})}
\newcommand{\vtrip}[3]{\begin{pmatrix}{#1}\\{#2}\\{#3}\end{pmatrix}}
\newcommand{\vquad}[4]{\begin{pmatrix}{#1}\\{#2}\\{#3}\\{#4}\end{pmatrix}}
\newcommand{\vquin}[5]{\begin{pmatrix}{#1}\\{#2}\\{#3}\\{#4}\\{#5}\end{pmatrix}}
\usepackage{setspace}
\usepackage{graphicx}
\usepackage{amsmath}
\usepackage{amssymb}
\newcommand*\widefbox[1]{\fbox{\hspace{2em}#1\hspace{2em}}}
%\usepackage{mathabx}
\newtcolorbox{mybox}[1][]{colback=white, sharp corners, #1}


%Create a new vspace line no indent
\newcommand{\nl}{\vspace{0.1in}\noindent}
\newcommand{\nnl}{\vspace{0.2in}\noindent}

%Remove hyphens
\tolerance=1
\emergencystretch=\maxdimen
\hyphenpenalty=10000
\hbadness=10000

%Page dimentions
\textwidth 6.5in
\hoffset=-.65in
\textheight=9.5in
\voffset=-1.in
\newcommand{\say}[1]{\textquotedblleft{#1}\textquotedblright}
\newcommand{\T}[2]{T\big({#1}\big)({#2})}
\newcommand{\func}[2]{\big({#1}\big)({#2})}
\newcommand{\dimn}[1]{\operatorname{dim}\,{#1}}
\newcommand{\rank}[1]{\operatorname{dim}\operatorname{range}{#1}}
\newcommand{\nullity}[1]{\operatorname{dim}\operatorname{null}{#1}}
\begin{document}

\pagestyle{fancy}
\fancyhf{}
\fancyhead[RO]{Matthew Wilder}
\fancyhead[LO]{Assignment 4 - MTH 307}
\fancyfoot[CO]{Page \thepage}

\noindent \textbf{Matthew Wilder}\\MATH 307 - Spring 2022 \\
Assignment \#4 \\
Due Friday, 02-11-22, 16:00 CST \\

For each Problem, include the statement of the problem. Leave a blank line.  At the beginning of the next line, write \textbf{Solution} or \textbf{Proof} -- as appropriate.

\begin{enumerate}
    \item Let $V$ be a finite-dimensional vector space and let $A,B,C,D \in \mathcal{L}(V)$.  Assume that $A+B$ and $A-B$ are invertible.  Show that there exist $X,Y$ so that
    \begin{align*}
    AX + BY &= C \\
    BX + AY &= D.
    \end{align*}
\begin{proof}
    Adding the two equations together, and using the property that $(A+B)^{-1}$ exists, we get
    \begin{align*}
        & AX + BY + BX + AY = C + D\\
        \Longleftrightarrow \;& A(X+Y) + B(X + Y) = C + D\\
        \Longleftrightarrow \;& (A+B)(X+Y) = C + D\\
        \Longleftrightarrow \;& (A+B)^{-1}(A+B)(X+Y) = (A+B)^{-1}(C+D)\\
        \Longleftrightarrow \;& X+Y = (A+B)^{-1}(C+D).
    \end{align*}
    Subtracting the two equations from each other and using the property that \\$(A-B)^{-1}$ exists, we get
    \begin{align*}
        & AX + BY - BX - AY = C - D\\
        \Longleftrightarrow \;& AX - AY - BX + BY = C - D\\
        \Longleftrightarrow \;& AX - AY - (BX - BY) = C - D\\
        \Longleftrightarrow \;& A(X-Y) - B(X-Y) = C - D\\
        \Longleftrightarrow \;& (A-B)(X-Y) = C - D\\
        \Longleftrightarrow \;& (A-B)^{-1}(A-B)(X-Y) = (A-B)^{-1}(C-D)\\
        \Longleftrightarrow \;& X-Y = (A-B)^{-1}(C-D).
    \end{align*}
    Adding the results from each system together,
    \begin{align*}
        & X+Y + X-Y= (A+B)^{-1}(C+D) + (A-B)^{-1}(C-D)\\
        \Longleftrightarrow \;& 2X  = (A+B)^{-1}(C+D) + (A-B)^{-1}(C-D)\\
        \Longleftrightarrow \;& X  = \frac{(A+B)^{-1}(C+D) + (A-B)^{-1}(C-D)}{2}.
    \end{align*}
    From our first system of equations,
    \begin{align*}
        & X + Y = (A+B)^{-1}(C+D)\\
        \Longleftrightarrow \;& Y = (A+B)^{-1}(C+D) - X\\
        \Longleftrightarrow \;& Y = (A+B)^{-1}(C+D) - \frac{(A+B)^{-1}(C+D) + (A-B)^{-1}(C-D)}{2}\\
        \Longleftrightarrow \;& Y = \frac{2(A+B)^{-1}(C+D)}{2} + \frac{-(A+B)^{-1}(C+D) - (A-B)^{-1}(C-D)}{2}\\
        \Longleftrightarrow \;& Y = \frac{2(A+B)^{-1}(C+D) -(A+B)^{-1}(C+D) - (A-B)^{-1}(C-D)}{2}\\
        \Longleftrightarrow \;& Y = \frac{(A+B)^{-1}(C+D) - (A-B)^{-1}(C-D)}{2}.
    \end{align*}
    Meanwhile, back at the ranch, we can write $C$ as
    \begin{align*}
        C &=\brac{\frac{(C+D) +(C-D)}{2}}\\
        &= \brac{\frac{(A+B)(A+B)^{-1}(C+D) + (A-B)(A-B)^{-1}(C-D)}{2}}\\
        &= \brac{\frac{A(A+B)^{-1}(C+D) + B(A+B)^{-1}(C+D) + A(A-B)^{-1}(C-D) - B(A-B)^{-1}(C-D)}{2}}\\
        &= \brac{\frac{A(A+B)^{-1}(C+D) + A(A-B)^{-1}(C-D) + B(A+B)^{-1}(C+D) - B(A-B)^{-1}(C-D)}{2}}\\
        &= A\brac{\frac{(A+B)^{-1}(C+D) + (A-B)^{-1}(C-D)}{2}} + B\brac{\frac{(A+B)^{-1}(C+D) - (A-B)^{-1}(C-D)}{2}}\\
        &= AX + BY.
    \end{align*}
    Smilarly, we can write $D$ as
    \begin{align*}
        D &= \brac{\frac{(C+D) - (C-D) }{2}}\\
        &= \brac{\frac{(A+B)(A+B)^{-1}(C+D) - (A-B)(A-B)^{-1}(C-D) }{2}}\\
        &= \brac{\frac{(A+B)(A+B)^{-1}(C+D) + (B-A)(A-B)^{-1}(C-D) }{2}}\\
        &= \brac{\frac{B(A+B)^{-1}(C+D) + A (A+B)^{-1}(C+D) + B(A-B)^{-1}(C-D)  -A (A-B)^{-1}(C-D)}{2}}\\
        &= \brac{\frac{B(A+B)^{-1}(C+D) + B(A-B)^{-1}(C-D) + A (A+B)^{-1}(C+D) -A (A-B)^{-1}(C-D)}{2}}\\
        &= B \brac{\frac{(A+B)^{-1}(C+D) + (A-B)^{-1}(C-D)}{2}} + A \brac{\frac{(A+B)^{-1}(C+D) - (A-B)^{-1}(C-D)}{2}}\\
        &= BX + AY.
    \end{align*}
    Therefore $AX + BY = C$ and $BX + AY = D$ have been shown.
\end{proof}
    \newpage
    \item Show that if $A \in \mathcal{L}(V)$ satisfying $A^2 -A+I=0$, then $A$ is invertible.
\begin{mybox}
\begin{proof}
\begin{align*}
    A^2 -A+I=0
    \iff \;& I = A - A^2\\
    \iff \;& I = A(I-A)\\
    \iff \;& I = (I-A)A
\end{align*}
Therefore $A$ is invertible and $A^{-1} = I-A$.
\end{proof}
\end{mybox}
    \vspace{1.2in}
    \item Assume $V$ is a finite-dimensional vector space with $S,T,U \in \mathcal{L}(V)$.  Show that if $STU=I$ then $T$ is invertible and $T^{-1}= US$
\begin{mybox}
\begin{proof} Utilizing that $AA^{-1} = I \iff A^{-1}A = I$, it can be shown directly that
    $$STU = I \iff S(TU) = I \iff (TU)S = I \iff T(US) = I \iff T^{-1} = US.$$
\end{proof}
\end{mybox}
    \vspace{1.5in}
    \item Let $T : V \to V$ be a linear map with a one-dimensional range.  Prove that $T^2 = cT$ for some scalar $c$.  (This means that $T(Tv) = cTv$ for all $v \in V$.)\vspace{0.5in}
\begin{mybox}
\begin{proof}
    Fix $v \in V$.
    
    \nl If $\rank{T} = 1$ then there is only one element in the basis of the range. Therefore, there exists some element $w \in V$ such that $\operatorname{range}T = \operatorname{span}w$. Then,
    \[Tv \in \operatorname{range}T = \operatorname{span}w = \{\alpha w : \alpha \in \mathbb{F}\}.\tag{4.1}\]
    By (4.1), there exists scalars $\beta$ and $\gamma$ in $\mathbb{F}$ such that $Tv = \beta w$ and $Tw = \gamma w$. Therefore,
    \begin{align}
        T(Tv) &= T(\beta w) & \text{By }Tv = \beta w\text{.}\notag\\
        &= \beta Tw & \text{By homogeneity of }T\text{.}\notag\\
        &= \beta \gamma w & \text{By }Tw = \gamma w\text{.}\notag\\
        &= \gamma \beta w & \text{By commutativity over }\mathbb{F}\text{.}\notag\\
        &= \gamma Tv. & \text{By }Tv = \beta w\text{.}\notag
    \end{align}
    Letting $c = \gamma$ it has been shown that $T(Tv) = cTv$ (for some scalar $c$) when $T : V \to V$ has a one dimensional range.
\end{proof}
\end{mybox}
    
    \newpage
    \item Let $D \in \mathcal{L}(\mathcal{P}_3(\mathbf{R}), \mathcal{P}_2(\mathbf{R}))$ denote the differentiation map $Dp = p'$.  Example 3.34 gives the matrix of $D$ with respect to the usual bases for $\mathcal{P}_3(\mathbf{R})$ and $\mathcal{P}_2(\mathbf{R})$.  \\
        Find two new bases for $\mathcal{P}_3(\mathbf{R})$ and $\mathcal{P}_2(\mathbf{R})$ so that the matrix for $D$ with respect to these bases is
        \[
        \left(
          \begin{array}{cccc}
            1 & 0 & 0 & 0 \\
            0 & 1 & 0 & 0 \\
            0 & 0 & 1 & 0 \\
          \end{array}
        \right).
        \]\vspace{.4in}
        \begin{mybox}
            \textbf{Solution: } Fix the basis of $\poly{2}$ to be $\{1, x, x^2\}$. Pulling from the columns of the given matrix as coefficients for with this basis, we can find the preimage from the definitions of derivitives. Thus,
            \begin{align}
                1(1) + 0(x) + 0(x^2) &= 1 = \dxof{x} = D(x). \notag\\
                0(1) + 1(x) + 0(x^2) &= x = \dxof{\frac{x^2}{2}} = D\pars{\frac{x^2}{2}} .\notag\\
                0(1) + 0(x) + 1(x^2) &= x^2 = \dxof{\frac{x^3}{3}} = D\pars{\frac{x^3}{3}}.\notag\\
                0(1) + 0(x) + 0(x^2) &= 0 = \dxof{1} = D(1).\notag 
            \end{align} 
            Therefore our ordered basis for $\poly{3} = (x,\;\frac{x^2}{2},\;\frac{x^3}{3},\;1)$.
        \end{mybox}
    
    \newpage
    \item The general operation of finding an antiderivative is not a linear map because of the ``$+C$'' which means that any function has infinitely many antiderivatives.  Let's define a linear map from $\mathcal{P}_3(\mathbf{R})$ to $\mathcal{P}_4(\mathbf{R})$ that avoids ambiguity.  Let $A(a_0 + a_1x + a_2x^2 + a_3 x^3) = a_0x + (a_1/2)x^2 + (a_2/3)x^3 + (a_3/4) x^4$.
        \begin{enumerate}
        \item Find the matrix of $A$ with respect to the standard bases for $\mathcal{P}_3(\mathbf{R})$ and $\mathcal{P}_4(\mathbf{R})$.\vspace{0.4in}
        \begin{mybox}
        \textbf{Solution: } Using the definition of antiderivatives, we need to write the antiderivative of each element in the standard basis of $\poly{3}$ as a linear combination of the standard basis of $\poly{4}$. Thus,
        \begin{align}
            A(x^0) &= \frac{x^1}{1} = 0(1) + 1(x) + 0(x^2) + 0(x^3) + 0(x^4).\notag\\
            A(x^1) &= \frac{x^2}{2} = 0(1) + 0(x) + \frac{1}{2}(x^2) + 0(x^3) + 0(x^4).\notag\\
            A(x^2) &= \frac{x^3}{3} = 0(1) + 0(x) + 0(x^2) + \frac{1}{3}(x^3) + 0(x^4).\notag\\
            A(x^3) &= \frac{x^4}{4} = 0(1) + 0(x) + 0(x^2) + 0(x^3) + \frac{1}{4}(x^4).\notag
        \end{align}
        Turning the coefficients on the right hand side of the four bases into column vectors, we have
        $$\vquin{0}{1}{0}{0}{0},\; \vquin{0}{0}{1/2}{0}{0},\; \vquin{0}{0}{0}{1/3}{0},\; \vquin{0}{0}{0}{0}{1/4}.$$
        Turning these into the columns of a 5x4 matrix, $A$, we get
        $$A = \begin{bmatrix}
            0 & 0 & 0 & 0\\
            1 & 0 & 0 & 0\\
            0 & 1/2 & 0 & 0\\
            0 & 0 & 1/3 & 0\\
            0 & 0 & 0 & 1/4\\
        \end{bmatrix}.
            $$

        \end{mybox}\newpage
        \item Find new bases for $\mathcal{P}_3(\mathbf{R})$ and $\mathcal{P}_4(\mathbf{R})$ so that the matrix for $A$ with respect to the new bases is
            \[
        \left(
          \begin{array}{cccc}
            1 & 0 & 0 & 0 \\
            0 & 1 & 0 & 0 \\
            0 & 0 & 1 & 0 \\
            0 & 0 & 0 & 1 \\
            0 & 0 & 0 & 0 \\
          \end{array}
        \right).
        \]\vspace{0.4in}
        \begin{mybox}
            \textbf{Solution: } We essentially need to reverse the methodology from part (a). We will fix our basis of $\poly{3}$ to be $\{1,x,x^2,x^3\}$ and the basis of $\poly{4}$ to be some $\{a,b,c,d,e\}$. Using its columns of $A$ as coefficients, we need
            \begin{align}
                A(1) &= \frac{x^1}{1} = 1(a) + 0(b) + 0(c) + 0(d) + 0(e).\notag\\
                A(x^1) &= \frac{x^2}{2} = 0(a) + 1(b) + 0(c) + 0(d) + 0(e).\notag\\
                A(x^2) &= \frac{x^3}{3} = 0(a) + 0(b) + 1(c) + 0(d) + 0(e).\notag\\
                A(x^3) &= \frac{x^4}{4} = 0(a) + 0(b) + 0(c) + 1(d) + 0(e).\notag
            \end{align}
            From this, we can see that
            \begin{align}
                a &= x^1,\notag\\
                b &= x^2/2,\notag\\
                c &= x^3/3,\text{ and}\notag\\
                d &= x^4/4.\notag
            \end{align}
            And our remaining term, $e$, in order to complete the basis of $\poly{4}$, must be a constant term. Therefore we'll let $e = \pi$.
            
            \nl Thus, we have a fixed and derived basis,
            \begin{align}
                \poly{3} &= \{1,\;x,\;x^2,\;x^3\} \quad \text{ and}\notag\\
                \poly{4} &= \left\{x,\;\frac{x^2}{2},\;\frac{x^3}{3},\;\frac{x^4}{4}, \;\pi\right\}.\notag
            \end{align}
        \end{mybox}
        \end{enumerate}
    
    \end{enumerate}
\end{document} 