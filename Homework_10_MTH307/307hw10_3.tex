Suppose that $T$ is a normal operator on $V$.  Suppose also that $v, w \in V$ satisfy the equations
    \[
        \|v\| = \|w\| = 2, \hspace{.25in} Tv=3v, \hspace{.25in} Tw=4w.
    \]
    Show that $\|T(v+w)\| = 10$.

    \nnl \textbf{Solution: }

    \nl By our hypotheses, $v \in E(3,T)$ and $w \in E(2,T)$. By proposition 7.22 (since $T$ is normal and 3 and 4 are distict eigenvalues), then $v$ and $w$ are orthogonal. Hence,
    \begin{align*}
        \norm{T(v+w)} &= \sqrt{\norm{T(v+w)}^2} \tag{Algebra} \\
        &= \sqrt{ \norm{Tv + Tw}^2} \tag{Linearity}\\
        &= \sqrt{\norm{3v + 4w}^2} \tag{$Tv = \lambda v$ substitution (hypotheses)}\\
        &= \sqrt{\norm{3v}^2 + \norm{4w}^2} \tag{Pythagorean Theorem}\\
        &= \sqrt{ 3^2 \norm v^2 + 4^2 \norm w^2} \tag{Properties of norms}\\
        &= \sqrt{ 9 \norm v^2 + 16 \norm w^2} \tag{Simplify}\\
        &= \sqrt{ 9 \cdot 2^2 + 16 \cdot 2^2} \tag{Substitution (hypotheses)}\\
        &= \sqrt{ 36 + 64 } \tag{Simplify}\\
        &= 10 \tag{Simplify}
    \end{align*}
    Therefore, under the hypotheses, $\norm{T(v+w)} = 10$.