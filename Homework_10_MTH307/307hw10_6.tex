(CST) Suppose $\mathbf{F} = \mathbf{C}$ and $T \in \mathcal{L}(V)$. Prove that $T$ is normal if and only if all pairs of eigenvectors corresponding to distinct eigenvalues of $T$ are orthogonal and
    \[
        V = E(\lambda_1,T) \oplus \cdots \oplus E(\lambda_m,T)
    \]
    where $\lambda_1 , \ldots , \lambda_m$ denote the distinct eigenvalues of $T$.

    \begin{proof} ($\Longrightarrow$)

        \nl Since $T$ is normal, by the Complex Spectral Theorem there exists an orthonormal basis of $V$ consisting of eigenvectors. The eigenvectors corresponding to distinct eigenvalues is a subset of that orthonormal basis, and hence are all orthonormal, and thus orthogonal.
        
        \nl By proposition 5.38 (page 156) the sum of eigenspaces for distinct eigenvalues is a direct sum. Hence, $V =  E(\lambda_1,T) \oplus \cdots \oplus E(\lambda_m,T)$.

    \end{proof}
\vspace{.5in}
    \begin{proof} ($\Longleftarrow$)

        \nl For every index $i \in [1,m]$ we can form an orthonormal basis of $E(\lambda_i,T)$. Since this basis is formed from vectors inside an eigenspace, then are themselves eigenvectors. Since $V$ is a direct sum of these eigenspaces, of which are orthonormal, we have an orthnormal basis of eigenvectors. Thus, by the Complex Spectral Theorem $[(b) \implies (a)]$, $T$ is normal.

    \end{proof}