\documentclass[12pt]{article}
\usepackage[english]{babel}
\usepackage{array}
\usepackage{setspace}
\usepackage{graphicx}
\usepackage{sistyle} %\num{100000} for commas
\SIthousandsep{,}
\usepackage{fancyhdr}
\usepackage{listings} % For code listings, may break stuff
\usepackage{xcolor, diagbox, empheq, makecell, tcolorbox}
\usepackage[autostyle]{csquotes}
\usepackage{amssymb, amsthm, linguex, enumitem, amsmath}
\usepackage{tcolorbox} %dont know what this does
\usepackage[colorlinks=true, allcolors=blue]{hyperref}
\usepackage{lipsum}
\usepackage{physics}
\usepackage{notomath}
\usepackage[T1]{fontenc}
\usepackage{multicol}
\makeatletter   %% <- make @ usable in macro names
\newcommand*\notab[1]{%
  \begingroup   %% <- limit scope of the following changes
    \par        %% <- start a new paragraph
    \@totalleftmargin=0pt \linewidth=\columnwidth
    %% ^^ let other commands know that the margins have been reset
    \parshape 0
    %% ^^ reset the margins
    #1\par      %% <- insert #1 and end this paragraph
  \endgroup
}
\makeatother    %% <- revert @

\pagestyle{empty}

\textwidth 6.9in
\hoffset= -.75in
\textheight=9.5in
\voffset=-1.in

%Sets
\newcommand{\R}{\mathbb{R}}
\newcommand{\C}{\mathbb{C}}
\newcommand{\N}{\mathbb{N}}
\newcommand{\F}{\mathbb{F}}
\newcommand{\Cb}{\mathbf{C}}
\newcommand{\Fb}{\mathbf{F}}
\newcommand{\Rb}{\mathbf{R}}

\renewcommand{\Re}{\operatorname{Re}}
\renewcommand{\Im}{\operatorname{Im}}
%Misc

\newcommand{\pars}[1]{\left( {#1} \right) } %auto size parenthesis 
\newcommand{\brac}[1]{\left[ {#1} \right] } %auto size brackets around arg
\newcommand{\Brac}[1]{\bigg[ {#1} \bigg] } %auto size brackets around arg
\newcommand{\set}[1]{\left\{{#1}\right\}} %auto size curly braces around arg
\newcommand{\vbrac}[1]{\left\langle{#1}\right\rangle} %vector angle brackets
\newcommand{\conj}[1]{\overline{{#1}}}
\newcommand{\vconj}[1]{\overline{\vbrac{{#1}}}}
\newcommand{\ceiling}[1]{\left\lceil {#1} \right\rceil} %auto size ceiling around arg
\newcommand{\floor}[1]{\left\lfloor {#1} \right\rfloor} %auto size floor around arg


\newcommand{\inner}[1]{\left\langle{#1}\right\rangle} %auto size angle brackets<>
\newcommand{\innerc}[1]{\overline{\left\langle{#1}\right\rangle}} %auto size angle brackets<>
\renewcommand{\norm}[1]{\left\| {#1} \right\|} % norm: ||v||
\renewcommand{\abs}[1]{\left| {#1} \right|} % absolute value |v|
\newcommand{\ceil}[1]{\left\lceil {#1} \right\rceil} %auto size ceiling

\newcommand{\limn}{\lim_{n\to\infty}} %limit as n approaches infinity
\newcommand{\thru}[1]{{#1}_1, \dots, {#1}_n}
\newcommand{\sumthru}[1]{{#1}_1 + \cdots + {#1}_n}
\renewcommand{\over}[1]{\frac{1}{{#1}}}
\newcommand{\pfrac}[2]{\left( \frac{{#1}}{{#2}} \right) } %auto size parenthesis over fraction 
\newcommand{\pover}[1]{\left( \frac{1}{{#1}} \right) } %auto size parenthesis over fraction

%Boolean Algebra
\newcommand{\OR}{\,\lor\,}
\newcommand{\AND}{\,\land\,}

%Probability and Statistics
\newcommand{\xbar}{\bar{X}}
\newcommand{\ybar}{\bar{Y}}
\newcommand{\yn}{Y_1, \dots, Y_n}
\newcommand{\yx}{X_1, \dots, X_n}
\newcommand{\normDist}{N\left(\mu, \sigma^2\right)} %default normal distribution
\newcommand{\gammaDist}[2]{\operatorname{Gamma} \left( {#1},{#2} \right)}
\newcommand{\prob}[1]{P \left( {#1} \right) }
\newcommand{\E}[1]{E \left( {#1} \right) }
\newcommand{\Eb}[1]{E[ \,{#1}\, ]} %E bracket
\newcommand*{\V}[1]{V \left( {#1} \right) }
\newcommand{\Vb}[1]{V [ \,{#1}\, ] }
\newcommand{\that}{\hat{\theta}} %theta hat
\newcommand{\phat}{\hat{p}}
\newcommand{\psihat}{\hat{\psi}}
\newcommand{\Psihat}{\hat{\Psi}}

%Linear Algebra
\newcommand{\spanset}[1]{\operatorname{span}\left\{{#1} \right\}} % \spanset{v}  is  span{v}
\newcommand{\range}[1]{\operatorname{range}{#1}} %range
\renewcommand{\null}{\operatorname{null}} %null
\newcommand{\dimrange}[1]{\operatorname{dim}\operatorname{range}{#1}} % dimrange
\newcommand{\dimnull}[1]{\operatorname{dim}\operatorname{null}{#1}} % dimnull
\newcommand{\mat}[4]{\begin{pmatrix}{#1} & {#2}\\{#3}&{#4}\end{pmatrix}} % 2x2 matrix
\newcommand{\Mat}[9]{\begin{bmatrix}{#1} & {#2} & {#3}\\{#4}&{#5}&{#6}\\{#7}&{#8}&{#9}\end{bmatrix}} % 3x3 matrix
\newcommand{\vdouble}[2]{\begin{pmatrix}{#1}\\{#2}\end{pmatrix}} % 2 high vertical vector
\newcommand{\vtriple}[3]{\begin{pmatrix}{#1}\\{#2}\\{#3}\end{pmatrix}} %vertical vector parenthesis, 3 args
\renewcommand{\L}[1]{\mathcal{L}\left({#1}\right)} %Set of linear maps
\newcommand{\poly}[2]{\mathcal{P}_{#1}({#2})} %polynomial up to degree (arg)
\newcommand{\pf}{\mathcal{P}(\mathbf{F})} %set of all polynomials
\newcommand{\vfive}[5]{\begin{pmatrix}{#1}\\{#2}\\{#3}\\{#4}\\{#5}\end{pmatrix}} %vertical vector parenthesis, 5 args
\newcommand{\vfour}[4]{\begin{bmatrix}{#1}\\{#2}\\{#3}\\{#4}\end{bmatrix}} %vertical vector parenthesis, 5 args
\newcommand{\detx}[4]{\begin{vmatrix}{#1} & {#2}\\{#3}&{#4}\end{vmatrix}} % 2x2 determinant

\renewcommand{\ev}[1]{\vec{\mathbf{e_{#1}}}}

\newcommand{\bu}{\vec{\mathbf{u}}}
\newcommand{\bv}{\vec{\mathbf{v}}}
\newcommand{\bw}{\vec{\mathbf{w}}}
\newcommand{\bzero}{\vec{\mathbf{0}}}

\newcommand{\per}[1]{{#1}^{\perp}}

%colors
\definecolor{ggreen}{RGB}{0, 127, 0}
\definecolor{dgray}{RGB}{63,63,63}
\definecolor{neonorange}{RGB}{255,47,0}
\definecolor{mygray}{rgb}{0.5,0.5,0.5}
\newcommand{\red}[1]{\color{red}{#1}\color{black}}
\newcommand{\grn}[1]{\color{ggreen}{#1}\color{black}}
\newcommand{\blu}[1]{\color{blue}{#1}\color{black}}
\newcommand{\redx}[1]{\color{red}\not{#1}\color{black}}

\newcommand{\prt}[1]{ \sqrt{{#1}} }
\newcommand{\port}[1]{\left( \frac{1}{\sqrt{{#1}}} \right)}

\newcommand{\say}[1]{\textquotedblleft{#1}\textquotedblright} %quote the "argument"
\newcommand*\widefbox[1]{\fbox{\hspace{2em}#1\hspace{2em}}}
\newtcolorbox{mybox}[1][]{colback=white, sharp corners, #1}

%Line break spacings
\newcommand{\nl}{\vspace{0.1in}\noindent}
\newcommand{\nnl}{\vspace{0.2in}\noindent}
\newcommand{\nnnl}{\vspace{0.3in}\noindent}

% Code snippets
\newcommand*{\code}{\fontfamily{qcr}\selectfont}
\lstset{
    backgroundcolor=\color{white},
    basicstyle=\footnotesize,
    breakatwhitespace=false,         % sets if automatic breaks should only happen at whitespace
    breaklines=true,                 % sets automatic line breaking
    captionpos=b,                    % sets the caption-position to bottom
    commentstyle=\color{dgray},    % comment style
    deletekeywords={...},            % if you want to delete keywords from the given language
    escapeinside={(*@}{@*)},          % if you want to add LaTeX within your code
    extendedchars=true,              % lets you use non-ASCII characters; for 8-bits encodings only, does not work with UTF-8
    firstnumber=1,                % start line enumeration with line 1
    frame=single,	                   % adds a frame around the code
    keepspaces=true,                 % keeps spaces in text, useful for keeping indentation of code (possibly needs columns=flexible)
    keywordstyle=\color{neonorange},       % keyword style
    language=C++,                 % the language of the code
    morekeywords={*,...},            % if you want to add more keywords to the set
    numbers=left,                    % where to put the line-numbers; possible values are (none, left, right)
    numbersep=5pt,                   % how far the line-numbers are from the code
    numberstyle=\tiny\color{mygray}, % the style that is used for the line-numbers
    rulecolor=\color{black},         % if not set, the frame-color may be changed on line-breaks within not-black text (e.g. comments (green here))
    showspaces=false,                % show spaces everywhere adding particular underscores; it overrides 'showstringspaces'
    showstringspaces=false,          % underline spaces within strings only
    showtabs=false,                  % show tabs within strings adding particular underscores
    stringstyle=\color{purple},     % string literal style
    tabsize=4,	                   % sets default tabsize to 4 spaces
}

\lstdefinestyle{cpp}{language=C++,
    morekeywords={cout, cin, Comparable, T},numbers=none
}
%Examples:
%{\code while}
%
%{\code \begin{lstlisting}[language=C++]
%sum1 = 0;
%for (i = 1; i <= n; i *= 2)
%    for (j = 1; j <= n; j++)
%        sum1++;
%\end{lstlisting}}



\begin{document}

\pagestyle{fancy}
\fancyhf{}
\fancyhead[RO]{Matthew Wilder} %header top right
\fancyhead[LO]{MTH 307 - Homework \#10} %header top left
\fancyfoot[CO]{Page \thepage} %page center bottom

\noindent MATH 307 \\
Assignment \#10 \\  % 6.A, 6.B, 6.C
Due Friday, April 1$^{\text{st}}$, 2022

\nl For each problem, include the statement of the problem. Leave a blank line.  At the beginning of the next line, write \textbf{Solution} or \textbf{Proof} -- as appropriate.

\begin{enumerate}
  \item Suppose that $T$ is a normal operator on $V$ and that 3 and 4 are eigenvalues of $T$.  Prove that there exists a vector $v \in V$ such that $\|v\| = \sqrt{2}$ and $\|Tv\| = 5$. 
\begin{proof} \,\\
Let $v_1, v_2 \in V$ be the eigenvectors corresponding to the eigenvalues 3 and 4 respectively. Then, by proposition 7.22 (since $T$ is normal and 3 and 4 are distinct eigenvalues), $v_1$ and $v_2$ are orthogonal. Let $e_1, e_2 \in V$ be eigenvectors of $T$ such that $e_1 := \dfrac{v_1}{\norm{v_1}} \in E(3, T)$ and $e_2 := \dfrac{v_2}{\norm{v_2}} \in E(4, T)$. Then, $\set{e_1, e_2}$ is a set of orthonormal eigenvectors of $T$ in $V$.

\nl Let $v \in V$ denote the linear combination $v = e_1 + e_2$. Applying the Pythagorean Theorem (since $e_1$ and $e_2$ are orthogonal), we get that
\begin{align*}
    \norm{v} &= \sqrt{\norm{v}^2} \tag{Algebra}\\
        &= \sqrt{\norm{e_1 + e_2}^2} \tag{Substitution}\\
        &= \sqrt{\norm{e_1}^2 + \norm{e_2}^2} \tag{Pythagorean Theorem}\\
        &= \sqrt{1 + 1} \tag{normalized vectors}\\
        &= \sqrt{2}.
\end{align*}
Therefore, there exists a vector $v \in V$ such that $\norm{v} = \sqrt{2}$.

\nl We will now show that $\norm{Tv} = 5$. Using the previous $v, e_1, e_2 \in V$, then
\begin{align*}
    \norm{Tv} &= \sqrt{\norm{Tv}^2} \tag{Algebra}\\
    &= \sqrt{\norm{Te_1 + Te_2}^2} \tag{Substitution and linearity}\\
    &= \sqrt{\norm{Te_1}^2 + \norm{Te_2}^2} \tag{Pythagorean Theorem}\\
    &= \sqrt{ \norm{3 e_1}^2 + \norm{4 e_2}^2} \tag{$Tv = \lambda v$}\\
    &= \sqrt{ \pars{\abs3 \norm{e_1}}^2 + \pars{\abs4 \norm{e_2}}^2 } \tag{Property of norms}\\
    &= \sqrt{ 9 \norm{e_1}^2 + 16 \norm{e_2}^2} \tag{Simplify}\\
    &= \sqrt{ 9 \sqrt{1^2} + 16 \sqrt{1^2}} \tag{Normalized vectors}\\
    &= 5\tag{Simplify} \\
\end{align*}
Hence, for $v \in V$ such that $v = e_1+e_2$, then $\norm{Tv} = 5$. Therefore, there exists a vector $v \in V$ such that $\norm v = \sqrt2$ and $\norm{Tv} = 5$.
\end{proof}
  \item \begin{enumerate}
    \item Suppose that $T$ is a self-adjoint operator on a finite-dimensional inner product space and that 2 and 3 are the only eigenvalues of $T$.  Prove that $T^2-5T+6I=0$.
    \begin{proof}$\;$\\
        Because self-adjoint implies normal, then regardless of if $\F = \R$ or $\F = \C$, $V$ has an orthonormal basis consisting of eigenvectors of $T$ of which has a diagonal matrix representation by either Spectral Theorem. Hence by direct proof, 
        \begin{align*}
            T^2 - 5T + 6I &= (T-2I)(T-3I)\\
            &= \brac{ \mqty(\dmat[]{2}{\ddots}{2}{3}{\ddots}{3}) - 2I} \cdot \brac{\mqty(\dmat[]{2}{\ddots}{2}{3}{\ddots}{3}) - 3I}\\
            &= \mqty(\dmat[]{0}{\ddots}{0}{1}{\ddots}{1}) \cdot \mqty(\dmat[]{-1}{\ddots}{-1}{0}{\ddots}{0})\\
            &= \mqty(\dmat[]{0 \cdot -1}{\ddots}{0 \cdot -1}{1 \cdot 0}{\ddots}{1\cdot 0})\\
            &= \mqty(\dmat{0}{\ddots}{0})\\
            &= 0
        \end{align*}
        Therefore $T^2-5T+6I=0$ (the zero transformation) for all $v \in V$  if $2$ and $3$ are the only eigenvalues.
    \end{proof}
    \newpage
    \item Give an example of an operator $T \in \mathcal{L}(\mathbf{C}^3)$ such that 2 and 3 are the only eigenvalues of $T$ and $T^2-5T+6I \ne 0$.

    \nnl \textbf{Solution: }
        
        \nl Let the transformation matrix of operator $T$ be be denoted by 
        $$\mathcal M(T) := \begin{bmatrix} 2 & \pi & 0 \\ 0 & 2 & 0 \\ 0 & 0 & 3 \end{bmatrix}.$$
        Then, it is clear by properties of upper-triangular matrices that $2$ (with multiplicity 2), and 3 are the eigenvalues of $T$. Additionally, $T^2 = \displaystyle \begin{bmatrix} 4 & 4\pi & 0 \\ 0 & 4 & 0 \\ 0 & 0 & 9 \end{bmatrix}$, and $5T$ and $6I$ are obvious.
        Hence, 
        \begin{align*}
            T^2 - 5T + 6I &= \begin{bmatrix} 4 & 4\pi & 0 \\ 0 & 4 & 0 \\ 0 & 0 & 9 \end{bmatrix} - \begin{bmatrix} 10 & 5\pi & 0 \\ 0 & 10 & 0 \\ 0 & 0 & 15 \end{bmatrix} + \mqty[\dmat[0]{6}{6}{6}]\\
            &= \begin{bmatrix}
                0 & -\pi & 0\\
                0 & 0 & 0\\
                0 & 0 & 0 
            \end{bmatrix}
        \end{align*}
        Which certainly is not the zero transformation.
    \end{enumerate}\newpage
  \item Suppose that $T$ is a normal operator on $V$.  Suppose also that $v, w \in V$ satisfy the equations
    \[
        \|v\| = \|w\| = 2, \hspace{.25in} Tv=3v, \hspace{.25in} Tw=4w.
    \]
    Show that $\|T(v+w)\| = 10$.

    \nnl \textbf{Solution: }

    \nl By our hypotheses, $v \in E(3,T)$ and $w \in E(2,T)$. By proposition 7.22 (since $T$ is normal and 3 and 4 are distict eigenvalues), then $v$ and $w$ are orthogonal. Hence,
    \begin{align*}
        \norm{T(v+w)} &= \sqrt{\norm{T(v+w)}^2} \tag{Algebra} \\
        &= \sqrt{ \norm{Tv + Tw}^2} \tag{Linearity}\\
        &= \sqrt{\norm{3v + 4w}^2} \tag{$Tv = \lambda v$ substitution (hypotheses)}\\
        &= \sqrt{\norm{3v}^2 + \norm{4w}^2} \tag{Pythagorean Theorem}\\
        &= \sqrt{ 3^2 \norm v^2 + 4^2 \norm w^2} \tag{Properties of norms}\\
        &= \sqrt{ 9 \norm v^2 + 16 \norm w^2} \tag{Simplify}\\
        &= \sqrt{ 9 \cdot 2^2 + 16 \cdot 2^2} \tag{Substitution (hypotheses)}\\
        &= \sqrt{ 36 + 64 } \tag{Simplify}\\
        &= 10 \tag{Simplify}
    \end{align*}
    Therefore, under the hypotheses, $\norm{T(v+w)} = 10$.\vspace{.75in}
  \item Suppose $T \in \mathcal{L}(V)$ is normal.  Prove that $\text{range }T = \text{range }T^*$.
\begin{proof} $\;$

    \nl Because $T$ is normal, by proposition 7.20, $\norm{Tv} = \norm{T^*v}$ for all $v$. Since $Tw = 0$ for any vector $w \in \null T$ and $Tv = T^*v$, then $w \in \null T^*$. Similarly, it can be shown that for $T^*u = 0$ for $u \in \null T^*$ and $T^*v = Tv$, then $u \in \null T$. Hence $\null T = \null T^*$. 
    
    \nl Then, using the table of properties regarding the null space and range of $T$ and $T^*$ (7.7),
    \begin{align*}
        \range T &= \pars{\null T^*}^{\perp} \tag{7.7 d}\\
         &= \pars{\null T}^{\perp} \tag{$\null T = \null T^*$}
         \\ &= \range T^* \tag{7.7 b}
    \end{align*}
    Therefore $\range T = \range T^*$ for a normal operator $T \in \L{V}$.

\end{proof}
  \item Consider the statement: If $T \in \mathcal{L}(V)$ and there exists an orthonormal basis $e_1, \ldots , e_n$ of $V$ such that $\|Te_j\| = \|T^*e_j\|$ for each $j$, then $T$ is normal.  Show that a counterexample to the statement is given by the matrix
    $T = \left(
  \begin{array}{cc}
    0 & 1  \\
    -1 & 2  \\
  \end{array}
\right)$ with respect to the standard basis in $\mathbf{R}^2$.

\nnl \textbf{Solution:}

\nl We have $\mathcal M(T^*) = \mat{0}{-1}{1}{2}$. Then,

\nl For $e_1$, 
$$\norm{Te_1} = \norm{ \mat{0}{1}{-1}{2} \vdouble10 } = \norm{ \vdouble{0}{-1} } = 1,$$
and
$$\norm{T^*e_1} = \norm{ \mat{0}{-1}{1}{2} \vdouble10 } = \norm{\vdouble01} = 1.$$
So $\norm{Te_1} = \norm{T^*e_1}$ holds. As for $e_2$,

$$\norm{Te_2} = \norm{ \mat{0}{1}{-1}{2} \vdouble01 } = \norm{ \vdouble12 } = 5,$$
and
$$\norm{T^*e_1} = \norm{ \mat{0}{-1}{1}{2} \vdouble01 } = \norm{\vdouble{-1}{2}} = 5.$$
So $\norm{Te_2} = \norm{T^*e_2}$ holds, and hence the hypothesis $\norm{Te_j} = \norm{T^*e_j}$ holds for each $j$. The conclusion then states that $T$ is normal, that is $TT^* = T^*T$. Checking this,
$$TT^* = \mat{0}{1}{-1}{2} \cdot \mat{0}{-1}{1}{2} = \mat{1}{-2}{-2}{5},$$
and 
$$T^*T = \mat{0}{-1}{1}{2} \cdot \mat{0}{1}{-1}{2} = \mat1225.$$
Since $TT^* \neq T^*T$, then $T$ is \textbf{not} normal. Hence, a counterexample. \newpage
  \item (CST) Suppose $\mathbf{F} = \mathbf{C}$ and $T \in \mathcal{L}(V)$. Prove that $T$ is normal if and only if all pairs of eigenvectors corresponding to distinct eigenvalues of $T$ are orthogonal and
    \[
        V = E(\lambda_1,T) \oplus \cdots \oplus E(\lambda_m,T)
    \]
    where $\lambda_1 , \ldots , \lambda_m$ denote the distinct eigenvalues of $T$.

    \begin{proof} ($\Longrightarrow$)

        \nl Since $T$ is normal, by the Complex Spectral Theorem there exists an orthonormal basis of $V$ consisting of eigenvectors. The eigenvectors corresponding to distinct eigenvalues is a subset of that orthonormal basis, and hence are all orthonormal, and thus orthogonal.
        
        \nl By proposition 5.38 (page 156) the sum of eigenspaces for distinct eigenvalues is a direct sum. Hence, $V =  E(\lambda_1,T) \oplus \cdots \oplus E(\lambda_m,T)$.

    \end{proof}
\vspace{.5in}
    \begin{proof} ($\Longleftarrow$)

        \nl For every index $i \in [1,m]$ we can form an orthonormal basis of $E(\lambda_i,T)$. Since this basis is formed from vectors inside an eigenspace, then are themselves eigenvectors. Since $V$ is a direct sum of these eigenspaces, of which are orthonormal, we have an orthnormal basis of eigenvectors. Thus, by the Complex Spectral Theorem $[(b) \implies (a)]$, $T$ is normal.

    \end{proof} \newpage
  \item (CST) Prove that a normal operator on a complex inner product space is self-adjoint if and only if all its eigenvalues are real.
\begin{proof} ($\Longrightarrow$)

    \nl By the Complex Spectral Theorem, a normal operator can be expressed as a diagonal matrix consisting of the eigenvalues. Let
    $$\mathcal M(T) =  \mqty[\dmat{\lambda_1}{\ddots}{\lambda_n}]$$
    for eigenvalues $\lambda_i$. Then by definition of $T^*$ (conjugate transpose),
    $$\mathcal M(T^*) = \mqty[\dmat{\bar{\lambda}_1}{\ddots}{\bar{\lambda}_n}].$$
    By the ($\Longrightarrow$) hypothesis, $T = T^*$, which implies $\lambda_1 = \bar{\lambda}_1, \dots, \lambda_n = \bar{\lambda}_n$, which is only true if $\lambda_i \in \R$. Hence if $T$ is self-adjoint, then all of its eigenvalues are real.
\end{proof}
\vspace{1in}
\begin{proof} ($\Longleftarrow$)

    \nl By the reasoning of the forward direction, we have that 
    $$\mathcal M(T) =  \mqty[\dmat{\lambda_1}{\ddots}{\lambda_n}] \qquad \text{and} \qquad \mathcal M(T^*) = \mqty[\dmat{\bar{\lambda}_1}{\ddots}{\bar{\lambda}_n}].$$
    Under the ($\Longleftarrow$) hypothesis, we assume $\lambda_i \in \R$, therefore $\bar{\lambda}_i = \lambda_i$ for every index $i \in [1, n]$. Hence,
    $$M(T^*) = \mqty[\dmat{\bar{\lambda}_1}{\ddots}{\bar{\lambda}_n}] =  \mqty[\dmat{\lambda_1}{\ddots}{\lambda_n}] = \mathcal M(T),$$
    which is the definition of self-adjoint.

\end{proof} \newpage
  \item (CST) Suppose $V$ is a complex inner product space. Prove that every normal operator on $V$ has a square root. (An operator $R \in \mathcal{L}(V)$ is called a \emph{square root} of $T$ if $R^2 = T$.)

\begin{proof}
    Let $T$ be an arbitrary normal operator on $V$. Then By the Complex Spectral Theorem, since $T$ is normal, there exists an orthonormal basis for $V$ consisting of $T$'s eigenvectors, which can further be used to create a diagonal matrix. Hence, the transformation matrix of $T$ can be expressed as:
    $$\mathcal M(T) = \mqty[\dmat{\lambda_1}{\ddots}{\lambda_n}]$$
    with $\lambda_i$ denoting an eigenvalue of $T$ and $\dim V = n$.

    \nl Now, let the transformation matrix of $R$ be defined as
    $$\mathcal M(R) = \mqty[\dmat{\sqrt{\lambda_1}}{\ddots}{\sqrt{\lambda_n}}].$$
    Then,
    $$\mathcal M(R^2) = \mqty[\dmat{\sqrt{\lambda_1}}{\ddots}{\sqrt{\lambda_n}}] \cdot \mqty[\dmat{\sqrt{\lambda_1}}{\ddots}{\sqrt{\lambda_n}}] = \mqty[\dmat{\lambda_1}{\ddots}{\lambda_n}] = \mathcal M(T).$$
    Hence, $R$ is the square root of $T$. Note that the square root of any complex number is indeed a complex number (and its opposite) (this is left as an exercise for a MTH 403 student).
\end{proof}
\end{enumerate}
\end{document} 