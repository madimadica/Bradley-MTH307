\documentclass[12pt]{article}
%\usepackage[document]{ragged2e}
\usepackage{array, amssymb, amsthm, linguex, enumerate, amsmath, physics, enumitem, xcolor, graphicx, xparse}
\let\fg\undefined %remove linguex/siunitx naming clash
\usepackage[english]{babel}
\usepackage[letterpaper,top=2cm,bottom=2cm,left=3cm,right=3cm,marginparwidth=1.75cm]{geometry}
\usepackage[colorlinks=true, allcolors=blue]{hyperref}
\usepackage[group-separator={,}]{siunitx} %\num{12345} -> "12,345"
\usepackage{fancyhdr}
\usepackage{notomath}
\usepackage[T1]{fontenc}

%Number sets
\newcommand{\R}{\mathbb{R}}
\newcommand{\C}{\mathbb{C}}
\newcommand{\N}{\mathbb{N}}
\newcommand{\F}{\mathbb{F}}
\renewcommand{\Re}{\operatorname{Re}}
\renewcommand{\Im}{\operatorname{Im}}
\renewcommand{\L}[1]{\mathcal{L}\left({#1}\right)} %Linear Map

\newcommand{\pmp}{\,\pm\,} %add small extra space to \pm

\NewDocumentCommand{\ceil}{ s m }{% ceiling brackets
    \IfBooleanTF{#1}%
    {\lceil #2 \rceil}% starred: no-autosizing
    {\left\lceil #2 \right\rceil}% unstarred: autosizing
}

\NewDocumentCommand{\ceiling}{ s m }{% ceiling brackets
    \IfBooleanTF{#1}%
    {\lceil #2 \rceil}% starred: no-autosizing
    {\left\lceil #2 \right\rceil}% unstarred: autosizing
}

\NewDocumentCommand{\floor}{ s m }{% floor brackets
    \IfBooleanTF{#1}%
    {\lfloor #2 \rfloor}% starred: no-autosizing
    {\left\lfloor #2 \right\rfloor}% unstarred: autosizing
}

\NewDocumentCommand{\pars}{ s m }{% parenthesis
    \IfBooleanTF{#1}%
    {( #2 ) }% starred: no-autosizing
    {\left( #2 \right) }% unstarred: autosizing
}

\NewDocumentCommand{\inner}{ s m }{% inner product
    \IfBooleanTF{#1}%
    {\langle #2 \rangle}% starred: no-autosizing
    {\left\langle #2 \right\rangle}% unstarred: autosizing
}

\NewDocumentCommand{\brac}{ s m }{% brackets
    \IfBooleanTF{#1}%
    {[#2] }% starred: no-autosizing
    {\left[ #2 \right] }% unstarred: autosizing
}

%default latex bracket size naming
\newcommand{\biggbrac}[1]{\bigg[ {#1} \bigg] }
\newcommand{\bigbrac}[1]{\big[ {#1} \big] }
\newcommand{\Bigbrac}[1]{\Big[ {#1} \Big] }


\RenewDocumentCommand{\over}{ s m }{% fraction 1/arg
    \IfBooleanTF{#1}%
    {\dfrac{1}{#2}}% starred: dfrac
    {\frac{1}{#2}}% unstarred: normal frac
}

\NewDocumentCommand{\pover}{ s m }{% parenthesis around fraction (1/arg)
    \IfBooleanTF{#1}%
    {\left(\dfrac{1}{#2}\right)}% starred: dfrac
    {\left(\frac{1}{#2}\right)}% unstarred: normal frac
}

\NewDocumentCommand{\pfrac}{ s m m}{% parenthesis around fraction (arg1/arg2)
    \IfBooleanTF{#1}%
    {\left( \dfrac{{#2}}{{#3}} \right)}% starred: dfrac
    {\left( \frac{{#2}}{{#3}} \right)}% unstarred: normal frac
}


\newcommand{\Xbar}{\bar{X}}
\newcommand{\Ybar}{\bar{Y}}
\newcommand{\xbar}{\bar{x}}
\newcommand{\ybar}{\bar{y}}


\newcommand{\limn}{\lim_{n\to\infty}}

\newcommand{\gammaDist}[2]{\operatorname{Gamma} \left( {#1},{#2} \right)} %gamma distribution
\NewDocumentCommand{\normalDist}{s g g}{ %normal distibution
    \IfBooleanTF{#1} { % starred, no autosizing parenthesis
      \IfNoValueTF{#2}{ 
          N (\mu,\, \sigma^2 ) %\normalDist* "default" normal distribution N(\mu, \sigma^2)
        } {
            \IfNoValueTF{#3}{N (#2)}{} %\normalDist{arg} --> N(arg)
        }
      \IfNoValueTF{#3}{}{N ( #2, #3 )}  %\normalDist*{arg1}{arg2} --> N(arg1,arg2)
    }  % else (unstarred) autosize parenthesis
    {
        \IfNoValueTF{#2}{
            N \left(\mu,\, \sigma^2 \right) %\normalDist "default" normal distribution N(\mu, \sigma^2)
        } {
            \IfNoValueTF{#3}{N \left(#2\right)}{} %\normalDist{arg} --> N(arg)
        }
        \IfNoValueTF{#3}{}{N \left( #2, #3 \right)} %\normalDist{arg1}{arg2} --> N(arg1,arg2)
    }
}



%colors
\definecolor{ggreen}{RGB}{0, 127, 0}
\definecolor{dgray}{RGB}{63,63,63}
\definecolor{neonorange}{RGB}{255,47,0}
\definecolor{mygray}{rgb}{0.5,0.5,0.5}
\definecolor{eblue}{RGB}{0,74,127}
\newcommand{\red}[1]{\color{red}{#1}\color{black}}
\newcommand{\green}[1]{\color{ggreen}{#1}\color{black}}
\newcommand{\blue}[1]{\color{blue}{#1}\color{black}}
\newcommand{\setRed}{\color{red}}
\newcommand{\setBlack}{\color{black}}
\newcommand{\setBlue}{\color{blue}}
\newcommand{\setGreen}{\color{ggreen}}



\newcommand{\thru}[1]{{#1}_1, \dots, {#1}_n}
\newcommand{\sumThru}[1]{{#1}_1 + \cdots + {#1}_n}
\newcommand{\yn}{Y_1, \dots, Y_n} % Y_1, ..., Y_n
\newcommand{\xn}{X_1, \dots, X_n} % Y_1, ..., Y_n

%hats and tildes
\newcommand{\that}{\widehat{\theta}} % theta hat
\newcommand{\phat}{\widehat{p}} % p hat
\newcommand{\qhat}{\widehat{q}} % p hat
\newcommand{\psihat}{\widehat{\psi}} % psi hat
\newcommand{\Psihat}{\widehat{\Psi}} % Psi hat
\newcommand{\ptilde}{\widetilde{p}} % psi tilde
\newcommand{\Psitil}{\widetilde{\Psi}} % Psi tilde
\newcommand{\betah}{\widehat{\beta}} % beta hat

%2x2 matrix shortcuts
\newcommand{\detx}[4]{\begin{vmatrix}{#1} & {#2}\\{#3}&{#4}\end{vmatrix}} % 2x2 determinant
\newcommand{\bmat}[4]{\begin{bmatrix}{#1} & {#2}\\{#3}&{#4}\end{bmatrix}} % 2x2 matrix brackets
\renewcommand{\pmat}[4]{\begin{pmatrix}{#1} & {#2}\\{#3}&{#4}\end{pmatrix}} % 2x2 matrix parenthesis

%remove any enumerate/itemize indent temporarily
\makeatletter   %% <- make @ usable in macro names
\newcommand*\notab[1]{%
  \begingroup   %% <- limit scope of the following changes
    \par        %% <- start a new paragraph
    \@totalleftmargin=0pt \linewidth=\columnwidth
    %% ^^ let other commands know that the margins have been reset
    \parshape 0
    %% ^^ reset the margins
    #1\par      %% <- insert #1 and end this paragraph
  \endgroup
}
\makeatother    %% <- revert @


\newcommand{\dimrange}[1]{\operatorname{dim}\operatorname{range}{#1}} % dimrange
\newcommand{\dimnull}[1]{\operatorname{dim}\operatorname{null}{#1}} % dimnull
\newcommand{\range}[1]{\operatorname{range}{#1}} %range
\newcommand{\nullspace}{\operatorname{null}} %null

% polynomial notation
\NewDocumentCommand{\poly}{ s g g }{%
    \IfBooleanTF{#1} {
        \IfNoValueTF{#2} {
            \mathcal{P}(\mathbb{R})
        } {
            \mathcal{P}_{#2}(\mathbb{R})
        }
    } {
        \IfNoValueTF{#3} {
            {\mathcal{P}(#2)}
        } { %else
            {\mathcal{P}_{#2}(#3)}
        }
    }
}

\NewDocumentCommand{\bias}{ s m }{% bias(arg)
    \IfBooleanTF{#1}%
    {\operatorname{bias}(#2)}% starred: no autosizing
    {\operatorname{bias}\left(#2\right)}% unstarred: autosizing
}

\NewDocumentCommand{\MSE}{ s m }{% MSE(arg)
    \IfBooleanTF{#1}%
    {\operatorname{MSE}(#2)}% starred: no autosizing
    {\operatorname{MSE}\left(#2\right)}% unstarred: autosizing
}

\NewDocumentCommand{\Var}{ s m }{% variance with parenthesis V(arg)
    \IfBooleanTF{#1}%
    {\operatorname{Var}(#2)}% starred: no autosizing
    {\operatorname{Var}\left(#2\right)}% unstarred: autosizing
}

\NewDocumentCommand{\Varb}{ s m }{% variance with brackets V[arg]
    \IfBooleanTF{#1}%
    {\operatorname{Var}[\,#2\,]}% starred: no autosizing
    {\operatorname{Var}\left[\,#2\,\right]}% unstarred: has autosizing
}

\NewDocumentCommand{\Vb}{ s m }{% another renaming of variance with brackets V[arg]
    \IfBooleanTF{#1}%
    {\operatorname{Var}[\,#2\,]}% starred: no autosizing
    {\operatorname{Var}\left[\,#2\,\right]}% unstarred: has autosizing
}

\NewDocumentCommand{\E}{ s m }{% expectation with parenthesis E(arg)
    \IfBooleanTF{#1}%
    {\operatorname{E}(#2)}% starred: no autosizing
    {\operatorname{E}\left(#2\right)}% unstarred: has autosizing
}

\NewDocumentCommand{\Eb}{ s m }{% expectation with brackets E[arg]
    \IfBooleanTF{#1}%
    {\operatorname{E}[#2]}% starred: no autosizing
    {\operatorname{E}\left[#2\right]}% unstarred: has autosizing
}

\RenewDocumentCommand{\P}{ s m }{% probability with parenthesis Pr(arg)
    \IfBooleanTF{#1}%
    {\Pr (#2) }% starred: no autosizing
    {\Pr \left( #2 \right) }% unstarred: has autosizing
}

\NewDocumentCommand{\prob}{ s m }{% probability with parenthesis Pr(arg)
    \IfBooleanTF{#1}%
    {\Pr (#2) }% starred: no autosizing
    {\Pr \left( #2 \right) }% unstarred: has autosizing
}

\NewDocumentCommand{\eff}{ s m }{% efficiency with parenthesis eff(arg)
    \IfBooleanTF{#1}%
    {\operatorname{eff}(#2)}% starred: no autosizing
    {\operatorname{eff}\left(#2\right)}% unstarred: has autosizing
}

%vertical vector of up to 8 elements
\NewDocumentCommand\vvec{s m g g g g g g g}{%
    \IfBooleanTF{#1} {
        \begin{bmatrix}% if starred use brackets
            \IfNoValueTF{#2}{}{#2}
            \IfNoValueTF{#3}{}{\\#3}
            \IfNoValueTF{#4}{}{\\#4}
            \IfNoValueTF{#5}{}{\\#5}
            \IfNoValueTF{#6}{}{\\#6}
            \IfNoValueTF{#7}{}{\\#7}
            \IfNoValueTF{#8}{}{\\#8}
        \end{bmatrix}
    }  % else (unstarred) use parethesis
    {
        \begin{pmatrix}%
            \IfNoValueTF{#2}{}{#2}
            \IfNoValueTF{#3}{}{\\#3}
            \IfNoValueTF{#4}{}{\\#4}
            \IfNoValueTF{#5}{}{\\#5}
            \IfNoValueTF{#6}{}{\\#6}
            \IfNoValueTF{#7}{}{\\#7}
            \IfNoValueTF{#8}{}{\\#8}
        \end{pmatrix}
    }
}
\def\Cov{\operatorname{Cov}} %Covariance
\def\df{\text{df}} %degrees of freedom

\NewDocumentCommand{\example}{ s g }{% Example header
    \IfBooleanTF{#1}%
    {\vspace{0.1in}}% starred: 0.1in
    {\vspace{0.2in}}% unstarred: 0.2in
    \IfNoValueTF{#2} {\noindent\textbf{\color{eblue} Example: }}{\noindent\textbf{\color{eblue} Example (#2): }}
}
\NewDocumentCommand{\disc}{ s }{% Discussion header
    \IfBooleanTF{#1}%
    {\vspace{0.1in}\noindent\textbf{Discussion: } }% starred: 0.1in
    {\vspace{0.2in}\noindent\textbf{Discussion: } }% unstarred: 0.2in
}
\NewDocumentCommand{\defn}{ s }{% Definition header
    \IfBooleanTF{#1}%
    {\vspace{0.1in}\noindent\textbf{\color{neonorange} Definition: } }% starred: 0.1in
    {\vspace{0.2in}\noindent\textbf{\color{neonorange} Definition: } }% unstarred: 0.2in
}
\NewDocumentCommand{\reason}{ s }{% Reason header
    \IfBooleanTF{#1}%
    {\vspace{0.1in}\noindent\textbf{Reason:} }% starred: 0.1in
    {\vspace{0.2in}\noindent\textbf{Reason:} }% unstarred: 0.2in
}
\NewDocumentCommand{\recall}{ s }{% Recall header
    \IfBooleanTF{#1}%
    {\vspace{0.1in}\noindent\textit{Recall:} }% starred: 0.1in
    {\vspace{0.2in}\noindent\textit{Recall:} }% unstarred: 0.2in
}
\NewDocumentCommand{\remark}{ s }{% Remark header
    \IfBooleanTF{#1}%
    {\vspace{0.1in}\noindent\textit{Remark:} }% starred: 0.1in
    {\vspace{0.2in}\noindent\textit{Remark:} }% unstarred: 0.2in
}

\NewDocumentCommand{\soln}{ s }{% Remark header
    \IfBooleanTF{#1}%
    {\vspace{0.1in}\noindent\textbf{Solution: } }% starred: 0.1in
    {\vspace{0.2in}\noindent\textbf{Solution: } }% unstarred: 0.2in
}

\newcommand{\proj}[2]{\operatorname{proj}_{{#1}}{#2}} %projection
\newcommand{\wideand}{\qquad \text{and} \qquad}

\newcommand{\bu}[1]{\textbf{\underline{{#1}}} } %bold underline
\newcommand{\boldit}[1]{\textbf{\textit{{#1}}} } %bold italix

% put actual quotation marks "around something"
\newcommand{\say}[1]{\textquotedblleft{#1}\textquotedblright}

% max{arg} and min{arg}
\renewcommand{\max}[1]{\operatorname{max}\left\{ #1 \right\}}
\renewcommand{\min}[1]{\operatorname{min}\left\{ #1 \right\}}

%Create a new vspace line no indent
\newcommand{\nl}{\vspace{0.1in}\noindent}
\newcommand{\nnl}{\vspace{0.2in}\noindent}
\newcommand{\nnnl}{\vspace{0.3in}\noindent}
\textwidth=7.02in
\hoffset=-.45in
\begin{document}

\pagestyle{fancy}
\fancyhf{}
\fancyhead[RO]{Matthew Wilder} %header top right
\fancyhead[LO]{MTH 307 - Homework \#11} %header top left
\fancyfoot[CO]{Page \thepage} %page center bottom

\noindent MATH 307 \\
Assignment \#11 \\  % 6.A, 6.B, 6.C
Due Friday, April 15$^{\text{th}}$, 2022

\nl For each problem, include the statement of the problem. Leave a blank line.  At the beginning of the next line, write \textbf{Solution} or \textbf{Proof} -- as appropriate.

\begin{enumerate}
  \item \begin{enumerate}[label=\alph*.)]
    \item Show that
        $A=\left( \begin{array} {c c c} 1 & 1 & 1 \\ 1 & 1 & 1 \\1 & 1 & 1 \end{array} \right)$ is positive.
\notab{ 
        \nl \begin{proof} Let $v$ be an arbitrary vector defined as $v = \vvec{x}{y}{z}$. Computing $\inner{Av,v}$, we obtain 
        \begin{align*}
            \inner{Av,v} &= \inner {\begin{bmatrix}
                1 & 1 & 1\\
                1 & 1 & 1\\
                1 & 1 & 1
            \end{bmatrix}
            \vvec{x}{y}{z}, \vvec{x}{y}{z} }\\
            &= \inner{\vvec{x+y+z}{x+y+z}{x+y+z}, \vvec{x,y,z}}\\
            &= x^2 + y^2 + z^2 + 2xy + 2xz + 2yz\\
            &= (x+y+z)^2\\
            &\geq 0
        \end{align*}
        Therefore $A$ is positive.
    \end{proof}
}\vspace{.3in}
    \item Find all $\alpha$ such that
        $A=\left( \begin{array} {c c c} \alpha & 1 & 1 \\ 1 & 0 & 0 \\1 & 0 & 0 \end{array} \right)$ is positive.

\notab{
        \soln* Let $v$ be an arbitrary vector defined as $v = \vvec{x}{y}{z}$. Computing $\inner{Av,v}$, we obtain 
        \begin{align*}
            \inner{Av,v} &= \inner {\begin{bmatrix}
                \alpha & 1 & 1\\
                1 & 0 & 0\\
                1 & 0 & 0
            \end{bmatrix}
            \vvec{x}{y}{z}, \vvec{x}{y}{z} }\\
            &= \inner{\vvec{\alpha x + y + z}{x}{x}, \vvec{x,y,z}}\\
            &= \alpha^2 + 2xy + 2xz
        \end{align*}
        Fix  $x = 1$ and $z = 0$, then $\inner{Av,v} = \alpha + 2y$. We will show that it is always possibly to make this negative. Hence,
         $$\inner{Av,v} = \alpha + 2y < 0 \iff 2y < -\alpha \iff y < -\frac{\alpha}{2}.$$
         Therefore we choose $\displaystyle y = -\frac{\alpha}{2} + \varepsilon$ for some $\varepsilon > 0$. Choosing $\varepsilon = 1$, then $v =\displaystyle \pars{1,\; -\frac{\alpha}{2} - 1, 0}$. Then 
         \begin{align*}
             \inner{Av, v} &= \alpha (1)^2 + 2(1)\pars{-\frac{\alpha}{2} - 1} + 2(1)(0)\\
             &= \alpha - \frac{2\alpha}{2} - 2\\
             &= -2
             \\ &< 0.
         \end{align*}
         Therefore, for any given $\alpha$, we can choose a $v$ such that $\inner{Av,v} < 0$ for that $v$. Hence, for every $\alpha$, $A$ is not a positive matrix. In other words, there exists no $\alpha$ such that $A$ is positive.
}\vspace{.3in}
    \item Show that even though all its entries are positive, the matrix
        $A=\left( \begin{array} {c c c} 2 & 2  \\ 2 & 1  \end{array} \right)$
        is not positive.

        \notab{
            \soln* Let $\displaystyle v := \vvec{0.1}{-0.1}$. Then
            \begin{align*}
                \inner{Av,v} &= \inner {\begin{bmatrix}
                    2 & 2 \\ 2 & 1
                \end{bmatrix}
                \vvec{0.1}{-0.1},\; \vvec{0.1}{-0.1}}\\ 
                &= \inner{ \vvec{0}{0.1},\; \vvec{0.1}{-0.1}}\\
                &= 0\cdot 0.1 + 0.1 \cdot (-0.1)\\
                &= -0.01.
            \end{align*}
            Therefore $A$ is not positive by counterexample.
    }\vspace{1in}
    \item Find an example of a positive matrix some of whose entries are negative.

\notab{
    \soln* Let $\displaystyle A := \begin{bmatrix}
        1 & -1 \\ -1 & 1
    \end{bmatrix}$ and $v = \displaystyle \vvec{x}{y}$ for some $x$ and $y$. Then
    \begin{align*}
        \inner{Av,v} &= \inner {\begin{bmatrix}
            1 & -1 \\ -1 & 1
        \end{bmatrix}
        \vvec{x}{y},\; \vvec{x}{y}}\\ 
        &= \inner{ \vvec{x-y}{-x+y},\; \vvec{x}{y}}\\
        &= x^2 -2xy + y^2\\
        &= (x-y)^2\\
        &\geq 0
    \end{align*}
    Since $\inner{Av,v} \geq 0$ for all $v = \vvec{x}{y}$ then $A$ is positive, and it has a negative entry.
}
\end{enumerate}
  \item If $T$ is a positive and invertible operator, is $T^{-1}$ positive?

\notab {\begin{proof}
    By the hypothesis that $T$ is positive, $T$ is also self-adjoint. Hence, by the Spectral Theorem there exists a diagonal matrix of eigenvalues. By the properties of positive operators, all eigenvalues are nonnegative. Since $T$ is invertible by the hypothesis, the eigenvalues cannot be zero, and hence are positive. Therefore 
    $$\mathcal{M}(T) = \begin{bmatrix}
        \lambda_1 & & 0 \\ & \ddots & \\ 0 & & \lambda_n
    \end{bmatrix} \wideand \mathcal{M}(T^{-1}) = \begin{bmatrix}
        \over{\lambda_1} & & 0 \\ & \ddots & \\ 0 & & \over{\lambda_n}
    \end{bmatrix}.$$
    The eigenvalues of $T^{-1}$ are again non-negative (strictly positive) and $T^{-1}$ is clearly self adjoint. Therefore by property $b \implies a$ of positive operators, $T^{-1}$ is positive. This also holds for complex values since the eigenvalues are nonnegative by definition of $T$ positive.
\end{proof}}\vspace{0.75in}
  \item Consider the three statements:
    \begin{enumerate}
        \item $T$ is self-adjoint
        \item $T$ is an isometry
        \item $T^2=I$ (such a $T$ is called an \emph{involution})
    \end{enumerate}
    Prove that if an operator has any two of the properties, then it has the third one as well.

    \notab{
\begin{proof} (a $\land$ b $\implies$ c)\\
    Suppose $T$ is a self-adjoint isometry. Then $T^*T = I$ by properties of an isometry. Then using self-adjoint, $I = T^*T = TT = T^2$. 
\end{proof} \vspace{.25in}
\begin{proof} (a $\land$ c $\implies$ b)\\
    Suppose $T$ is self-adjoint and $T^2 = I$. Then $I = T^2 = TT = T^*T$. Therefore it has been shown since $T^*T = I$ is an equivalent condition of an isometry.
\end{proof}\vspace{.25in}
\begin{proof} (b $\land$ c $\implies$ a)\\
    Suppose $T$ is an isometry and $T^2 = I$. Then $T^*T = I$ by properties of an isometry. And since $I = T^2$ by hypothesis, then $T^*T = T^2 = TT$. Multiplying by $T$ on the right, we have $$T^*T = TT \iff T^*TT = TTT \iff T^*T^2 = TT^2 \iff T^*I = TI \iff T^* = T.$$ Therefore $T$ is self-adjoint.  
\end{proof} }
  \item Prove or give a counterexample:  If $T \in \mathcal{L}(V)$ and there exists an orthonormal basis $e_1, \ldots , e_n$ of  $V$ such that $\|Te_i\| = 1$ for each $e_i$, then $T$ is an isometry.

\notab{
\soln* Let $T = \begin{bmatrix}
    1 & 1 \\ 0 & 0
\end{bmatrix}$. Then $\norm{Te_1} = \norm{\vvec{1}{0}} = 1$ and $\norm{Te_2} = \norm{\vvec{1}{0}} = 1$. As such, the hypothesis are fulfilled but $T^2 = \begin{bmatrix}
    1 & 1 \\ 0 & 0
\end{bmatrix} \begin{bmatrix}
    1 & 1 \\ 0 & 0
\end{bmatrix} = \begin{bmatrix}
    1 & 1 \\ 0 & 0
\end{bmatrix} \neq I_2$. Hence $T$ is not an isometry and the assumption is false by counterexample.}\vspace{1.5in}
  \item Suppose $T \in \mathcal{L}(V)$.  Prove that there exists an isometry $S \in \mathcal{L}(V)$ such that
    \[
        T = \sqrt{TT^*}\;S.
    \]

    \notab{
\begin{proof}$\,$\\
    By Polar Decomposition, there exists an isometry $S_1 \in \L{V}$ such that $T^* = S_1\sqrt{(T^*)^*T^*} = S_1 \sqrt{TT^*}$. Taking the adjoint of both sides, 
    \begin{align*}
        & T^* = S_1 \sqrt{TT^*} \\
        \iff & (T^*)^* = \pars{S_1 \sqrt{TT^*}}^* \tag{adjoint both sides}\\
        \iff & T = \pars{\sqrt{TT^*}}^*{S_1}^* \tag{distribution of adjoint}
    \end{align*}
    Because $T^*T$ is a positive operator for any $T \in \L{V}$, then by property (b) of positive operators, $T^*T$ is self-adjoint. Similarly, the square root of a positive operator is self-adjoint so $\sqrt{T^*T}$ is self-adjoint. Therefore, $T = \pars{\sqrt{TT^*}}^*{S_1}^* = \sqrt{TT^*} {S_1}^*$. Since $S_1$ is an isometry, then ${S_1}^*$ is also an isometry by property (g). Therefore, there exists some isometry $S$ such that $T = \sqrt{TT^*}S$.
\end{proof} 
} \newpage
  \item Find the singular values of the differentiation operator $D \in \mathcal{L}(\mathcal{P}_2(\mathbf{R}))$ defined by $Dp = p'$, where the inner product is $\langle p , q \rangle = \int_{-1}^1 p(x)q(x)\;dx$.\\
    Remark: It might be helpful to compute the matrix for $D$ with respect to the basis $1, x, x^2$ to find eigenvalues (easy) and then compute the matrix for $D$ again using an \emph{orthonormal basis} for $\mathcal{P}_2(\mathbf{R})$ to compute the singular values.  Use some technology for the integrations.
    
    \notab{
\soln Using the orthonormal basis of $\poly{2}$ from Axler's Example 6.33, then $$\displaystyle \mathcal B = \pars{\sqrt{\frac{1}{2}}, \sqrt{\frac32}x, \sqrt{\frac{45}{8}}\pars{x^2-\over*{3}}}.$$ Applying the operator to this basis, $\displaystyle \dv{x}\pars{\sqrt{\over{2}}} = 0$ and  a change of basis on 0 is 0, hence $D(e_1) = 0$. Next, $\displaystyle  \dv{x}\pars{\sqrt{\frac{3}{2}}x} = \sqrt{\frac{3}{2}}$. For a change in basis we have $\displaystyle  a\sqrt{\frac{1}{2}} = \sqrt{\frac{3}{2}}$ which implies $a = \sqrt{3}$. Therefore $\displaystyle  D(e_2) = \pars{\sqrt{3}, 0, 0}$. Finally $\displaystyle  \dv{x} \pars{ \sqrt{\frac{45}{8}} \pars{x^2 - \over{3}} } = \sqrt{\frac{45}{2}}x$. Changing this basis, $\displaystyle  a \sqrt{\frac{3}{2}}x = \sqrt{\frac{45}{2}}x$ implies that $a = \sqrt{15}$. Thus $D(e_3) = \pars{0, \sqrt{15}, 0}$. Therefore, the transformation matrix with respect to an orthonormal basis is
$$\mathcal{M}(D) = \begin{bmatrix}
    0 & \sqrt{3} & 0 \\ 0 & 0 & \sqrt{15} \\ 0 & 0 & 0
\end{bmatrix} \wideand \mathcal{M}(D^*) = \begin{bmatrix}
    0 & 0 & 0 \\ \sqrt{3} & 0 & 0\\ 0 & \sqrt{15} & 0
\end{bmatrix}.$$
And hence,
$$M(D^*D) = \begin{bmatrix}
    0 & 0 & 0 \\ 0 & 3 & 0 \\ 0 & 0 & 15
\end{bmatrix}.$$
By properties of upper triangular matrices, the eigenvalues of $D^*D$ are 0, 3, and 15. Therefore the singular values are $\sqrt{15}, \sqrt{3}, 0$ by proposition 7.52 (nonnegative square roots). 
} \newpage
  \item Define $T \in \mathcal{L}(\mathbf{F}^3)$ by $T(z_1,z_2,z_3) = ( 4z_2 , 5z_3 , z_1 )$. Find (explicitly) an isometry $S \in \mathcal{L}(\mathbf{F}^3)$ such that $T = S\;\sqrt{T^*T}$.

\notab{
    \soln* It is clear that the matrix of $T$ with respect to the standard basis is 
    $$\mathcal M(T) = \begin{bmatrix}
        0 & 4 & 0 \\ 0 & 0 & 5\\ 1 & 0 & 0
    \end{bmatrix} \wideand \mathcal M(T^*) = \begin{bmatrix}
        0 & 0 & 1 \\ 4 & 0 & 0 \\ 0 & 5 & 0
    \end{bmatrix}.$$
    Further,
    $$T^*T = \begin{bmatrix}
        1 & 0 & 0 \\ 0 & 16 & 0 \\ 0 & 0 & 25 
    \end{bmatrix} \wideand \sqrt{T^*T} = \begin{bmatrix}
        1 & 0 & 0 \\ 0 & 4 & 0 \\ 0 & 0 & 5 
    \end{bmatrix}.$$
    By polar decomposition we know that $T = S\sqrt{T^*T}$, so multiplying on the right by $\pars{\sqrt{T^*T}}^{-1}$ yields $T \pars{\sqrt{T^*T}}^{-1} = S$. Since $T^*T$ is diagonal, the inverse is the inverse of the diagonal entries, hence
    $$\pars{\sqrt{T^*T}}^{-1} = \begin{bmatrix}
        1 & 0 & 0 \\ 0 & \frac{1}{4} & 0 \\ 0 & 0 & \frac15 
    \end{bmatrix}.$$
    Now, computing $S$ explicity, we have
    \begin{align*}
        S &= T \pars{\sqrt{T^*T}}^{-1}\\
        &= \begin{bmatrix}
            0 & 4 & 0 \\ 0 & 0 & 5\\ 1 & 0 & 0
        \end{bmatrix}\begin{bmatrix}
            1 & 0 & 0 \\ 0 & \frac{1}{4} & 0 \\ 0 & 0 & \frac15 
        \end{bmatrix}\\
        &= \begin{bmatrix}
            0 & 1 & 0 \\ 0 & 0 & 1 \\ 1 & 0 & 0
        \end{bmatrix}.
    \end{align*}
    To show that this is an isometry, we need $\norm{Sv} = \norm{v}$. Which for $v = (v_1, v_2, v_3)$, we have $S(v_1, v_2, v_3) = (v_2, v_3, v_1)$. Clearly $\norm{(v_1, v_2, v_3)} = \sqrt{{v_1}^2 + {v_2}^2 + {v_3}^2} =  \norm{(v_2, v_3, v_1)}$. Hence $S$ is an isometry.
    } \newpage
  \item Suppose $T \in \mathcal{L}(V)$ is self-adjoint.  Prove that the singular values of $T$ equal the absolute values of the eigenvalues of $T$, repeated appropriately.

\notab{
    \begin{proof}$\,$\\
        Since $T$ is a self-adjoint operator, under the Spectral Theorem there exists a diagonal matrix consisting of the eigenvalues for $T$. Hence
        $$\mathcal M(T) = \begin{bmatrix}
            \lambda_1 & & \\ & \ddots & \\ & & \lambda_n
        \end{bmatrix} \wideand \mathcal M(T^*) = \begin{bmatrix}
            \bar{\lambda_1} & & \\ & \ddots & \\ & & \bar{\lambda_n}
        \end{bmatrix}$$
        therefore 
        $$\mathcal M (T^*T) = \begin{bmatrix}
            \bar{\lambda_1}\lambda_1 & & \\ & \ddots & \\ & & \bar{\lambda_n}\lambda_n
        \end{bmatrix} = \begin{bmatrix}
            \abs{\lambda_1}^2 & & \\ & \ddots & \\ & & \abs{\lambda_n}^2
        \end{bmatrix}.$$
        By proposition 7.52 the singular values of $T$ are the square roots of the eigenvalues of $T^*T$, which are clearly $\abs{\lambda_i}^2$. So $\sqrt{\abs{\lambda_i}^2} = \abs{\lambda_i}$ are the singular values. Therefore for each eigenvalue $\lambda_i$ of $T$, there is a corresponding singular value $\abs{\lambda_i}$ of $T$.
    \end{proof}
}
\end{enumerate}
\end{document} 