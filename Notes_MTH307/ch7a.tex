\defnn{adjoint,} $T^*$
\\Suppose $T \in \L{V,W}$. The \boldit{adjoint} of $T$ is the function $T^* : W \to V$ such that
$$\inner{Tv,w} = \inner{v,T^*w}$$
for every $v \in V$ and every $w \in W$.

\header{The adjoint is a linear map}\\
If $T \in \L{V,W}$, then $T^* \in \L{W,V}$.

\properties{Properties of the adjoint}
\begin{enumerate}[label=(\alph*)]
    \item $(S+T)^*=S^* + T^* \fa S,T \in \L{V,W}$
    \item $\pars{\lambda T}^* = \bar{\lambda}T^* \fa \lambda \in F \text{ and } T \in \L{V,W}$
    \item $\pars{T^*}^* = T \fa T \in \L{V,W}$
    \item $I^* = I$, where $I$ is the identity operator on $V$
    \item $(ST)^* = T^*S^* \fa T \in \L{V,W}$ and $S \in \L{W,U}$ (here $U$ is an inner product space over $\F$).
\end{enumerate}

\header{Null space and range of $T^*$}
\begin{enumerate}[label=(\alph*)]
    \item $\null T^* = \per{(\range T)}$
    \item $\range T^* = \per{(\null T)}$
    \item $\null T = \per{(\range T^*)}$
    \item $\range T = \per{(\null T^*)}$
\end{enumerate}

\defn{conjugate transpose}
\\The \boldit{conjugate transpose} of an $m$-by-$n$ matrix is the $n$-by-$m$ matrix obtained by transposing the matrix and taking the complex conjugate of each entry.

\header{The matrix of $T^*$}
\\Let $T \in \L{V,W}$. Suppose $\thru{e}$ is an orthonormal basis of $V$ and $f_1, \dots, f_m$ is an orthonormal basis of $W$. Then
$$\mathcal{M}\bigl( T^*, (f_1, \dots, f_m), (\thru e) \bigr)$$
is the conjugate transpose of 
$$\mathcal{M}\bigl( T, (\thru e), (f_1, \dots, f_m) \bigr)$$

\defn{self-adjoint}
\\An operator $T \in \L{V}$ is called \boldit{self-adjoint} if $T = T^*$. In other words, $T \in \L{V}$ is self-adjoint if and only if
$$\inner{Tv,w} = \inner{v,Tw}$$
for all $v,w \in V$.

\header{Eigenvalues of self-adjoint operators are real}\\
Every eigenvalue of a self-adjoint operator is real

\header{Over $\C$, $Tv$ is orthogonal to $v$ for all $v$ only for the 0 operator}\\
Suppose $V$ is a complex inner product space and $T \in \L V.$ Suppose $$\inner{Tv,v}=0$$ for all $v \in V$. Then $T =0$.

\header{Over $\C$, $\inner{Tv,v}$ is real for all $v$ only for self-adjoint operators}\\
Suppose $V$ is a complex inner product space and $T \in \L V.$ Then $T$ is self-adjoint if and only if 
$$\inner{Tv,v} \in \R$$
for every $v \in V$.

\header{If $T = T^*$ and $\inner{Tv,v} = 0$ for all $v$, then $T = 0$}\\
Suppose $T$ is a self-adjoint operator on $V$ such that
$$\inner{Tv, v} = 0$$
for all $v \in V$. Then $T = 0$.

\defn{normal}
\\An operator on an inner product space is called \boldit{normal} if it commutes with its adjoint. In other words, $T \in \L V$ is normal if $$TT^* = T^* T.$$

\header{$T$ is normal if and only if $\norm{Tv} = \norm{T^*v}$ for all $v$}\\
An operator $T \in \L V$ is normal if and only if $$\norm{Tv} = \norm{T^*v}$$ for all $v \in V$.

\header{For $T$ normal, $T$ and $T^*$ have the same eigenvectors}\\
Suppose $T \in \L V$ is normal and $v \in V$ is an eigenvector of $T$ with eigenvalue $\lambda$. Then $v$ is also an eigenvector of $T^*$ with eigenvalue $\bar{\lambda}$.

\header{Orthogonal eigenvectors for normal operators}\\
Suppose $T \in \L V$ is normal. Then the eigenvectors of $T$ corresponding to distinct eigenvalues are orthogonal.