\documentclass[12pt]{article}
\usepackage[english]{babel}
\usepackage{array}
\usepackage{setspace}
\usepackage{graphicx}
\usepackage{sistyle} %\num{100000} for commas
\SIthousandsep{,}
\usepackage{fancyhdr}
\usepackage{listings} % For code listings, may break stuff
\usepackage{xcolor, diagbox, empheq, makecell, tcolorbox}
\usepackage[autostyle]{csquotes}
\usepackage{amssymb, amsthm, linguex, enumitem, amsmath}
\usepackage{tcolorbox} %dont know what this does
\usepackage[colorlinks=true, allcolors=blue]{hyperref}
\usepackage{lipsum}
%\usepackage{times}
%\usepackage{txfonts}
%\usepackage{mathptmx}
%\usepackage{fourier}

%\usepackage{libertine}
%\usepackage{libertinust1math}
%\usepackage[T1]{fontenc}

\usepackage{notomath}
\usepackage[T1]{fontenc}

%\usepackage{cmbright}
%\usepackage[T1]{fontenc}


\makeatletter   %% <- make @ usable in macro names
\newcommand*\notab[1]{%
  \begingroup   %% <- limit scope of the following changes
    \par        %% <- start a new paragraph
    \@totalleftmargin=0pt \linewidth=\columnwidth
    %% ^^ let other commands know that the margins have been reset
    \parshape 0
    %% ^^ reset the margins
    #1\par      %% <- insert #1 and end this paragraph
  \endgroup
}
\makeatother    %% <- revert @

\pagestyle{empty}

\textwidth 6.5in
\hoffset=-.65in
\textheight=9.5in
\voffset=-1.in

%Sets
\newcommand{\R}{\mathbb{R}}
\newcommand{\C}{\mathbb{C}}
\newcommand{\N}{\mathbb{N}}
\newcommand{\F}{\mathbb{F}}
\newcommand{\Cb}{\mathbf{C}}
\newcommand{\Fb}{\mathbf{F}}
\newcommand{\Rb}{\mathbf{R}}

%Misc

\newcommand{\pars}[1]{\left( {#1} \right) } %auto size parenthesis 
\newcommand{\brac}[1]{\left[ {#1} \right] } %auto size brackets around arg
\newcommand{\Brac}[1]{\bigg[ {#1} \bigg] } %auto size brackets around arg
\newcommand{\set}[1]{\left\{{#1}\right\}} %auto size curly braces around arg
\newcommand{\vbrac}[1]{\left\langle{#1}\right\rangle} %vector angle brackets
\newcommand{\conj}[1]{\overline{{#1}}}

\newcommand{\vconj}[1]{\overline{\vbrac{{#1}}}}
\newcommand{\ceiling}[1]{\left\lceil {#1} \right\rceil} %auto size ceiling around arg
\newcommand{\floor}[1]{\left\lfloor {#1} \right\rfloor} %auto size floor around arg


\newcommand{\inner}[1]{\left\langle{#1}\right\rangle} %auto size angle brackets<>
\newcommand{\innerc}[1]{\overline{\left\langle{#1}\right\rangle}} %auto size angle brackets<>
\newcommand{\norm}[1]{\left\| {#1} \right\|} % norm: ||v||
\newcommand{\abs}[1]{\left| {#1} \right|} % absolute value |v|
\newcommand{\ceil}[1]{\left\lceil {#1} \right\rceil} %auto size ceiling

\newcommand{\limn}{\lim_{n\to\infty}} %limit as n approaches infinity
\newcommand{\thru}[1]{{#1}_1, \dots, {#1}_n}
\newcommand{\sumthru}[1]{{#1}_1 + \cdots + {#1}_n}
\renewcommand{\over}[1]{\frac{1}{{#1}}}
\newcommand{\pfrac}[2]{\left( \frac{{#1}}{{#2}} \right) } %auto size parenthesis over fraction 
\newcommand{\pover}[1]{\left( \frac{1}{{#1}} \right) } %auto size parenthesis over fraction

%Boolean Algebra
\newcommand{\OR}{\,\lor\,}
\newcommand{\AND}{\,\land\,}

%Probability and Statistics
\newcommand{\xbar}{\bar{X}}
\newcommand{\ybar}{\bar{Y}}
\newcommand{\yn}{Y_1, \dots, Y_n}
\newcommand{\yx}{X_1, \dots, X_n}
\newcommand{\normDist}{N\left(\mu, \sigma^2\right)} %default normal distribution
\newcommand{\gammaDist}[2]{\operatorname{Gamma} \left( {#1},{#2} \right)}
\newcommand{\prob}[1]{P \left( {#1} \right) }
\newcommand{\E}[1]{E \left( {#1} \right) }
\newcommand{\Eb}[1]{E[ \,{#1}\, ]} %E bracket
\newcommand*{\V}[1]{V \left( {#1} \right) }
\newcommand{\Vb}[1]{V [ \,{#1}\, ] }
\newcommand{\that}{\hat{\theta}} %theta hat
\newcommand{\phat}{\hat{p}}
\newcommand{\psihat}{\hat{\psi}}
\newcommand{\Psihat}{\hat{\Psi}}

%Linear Algebra
\newcommand{\spanset}[1]{\operatorname{span}\left\{{#1} \right\}} % \spanset{v}  is  span{v}
\newcommand{\range}[1]{\operatorname{range}{#1}} %range
\renewcommand{\null}{\operatorname{null}} %null
\newcommand{\dimrange}[1]{\operatorname{dim}\operatorname{range}{#1}} % dimrange
\newcommand{\dimnull}[1]{\operatorname{dim}\operatorname{null}{#1}} % dimnull
\newcommand{\mat}[4]{\begin{bmatrix}{#1} & {#2}\\{#3}&{#4}\end{bmatrix}} % 2x2 matrix
\newcommand{\Mat}[9]{\begin{bmatrix}{#1} & {#2} & {#3}\\{#4}&{#5}&{#6}\\{#7}&{#8}&{#9}\end{bmatrix}} % 3x3 matrix
\newcommand{\vdouble}[2]{\begin{pmatrix}{#1}\\{#2}\end{pmatrix}} % 2 high vertical vector
\newcommand{\vtriple}[3]{\begin{pmatrix}{#1}\\{#2}\\{#3}\end{pmatrix}} %vertical vector parenthesis, 3 args
\renewcommand{\L}[1]{\mathcal{L}\left({#1}\right)} %Set of linear maps
\newcommand{\poly}[2]{\mathcal{P}_{#1}({#2})} %polynomial up to degree (arg)
\newcommand{\pf}{\mathcal{P}(\mathbf{F})} %set of all polynomials
\newcommand{\vfive}[5]{\begin{pmatrix}{#1}\\{#2}\\{#3}\\{#4}\\{#5}\end{pmatrix}} %vertical vector parenthesis, 5 args
\newcommand{\vfour}[4]{\begin{bmatrix}{#1}\\{#2}\\{#3}\\{#4}\end{bmatrix}} %vertical vector parenthesis, 5 args
\newcommand{\detx}[4]{\begin{vmatrix}{#1} & {#2}\\{#3}&{#4}\end{vmatrix}} % 2x2 determinant

\newcommand{\ev}[1]{\vec{\mathbf{e_{#1}}}}

\newcommand{\bu}{\vec{\mathbf{u}}}
\newcommand{\bv}{\vec{\mathbf{v}}}
\newcommand{\bw}{\vec{\mathbf{w}}}
\newcommand{\bzero}{\vec{\mathbf{0}}}

%colors
\definecolor{ggreen}{RGB}{0, 127, 0}
\definecolor{dgray}{RGB}{63,63,63}
\definecolor{neonorange}{RGB}{255,47,0}
\definecolor{eblue}{RGB}{0,74,127}
\definecolor{mygray}{rgb}{0.5,0.5,0.5}
\newcommand{\red}[1]{\color{red}{#1}\color{black}}
\newcommand{\grn}[1]{\color{ggreen}{#1}\color{black}}
\newcommand{\blu}[1]{\color{blue}{#1}\color{black}}
\newcommand{\redx}[1]{\color{red}\not{#1}\color{black}}

\newcommand{\prt}[1]{ \sqrt{{#1}} }
\newcommand{\port}[1]{\left( \frac{1}{\sqrt{{#1}}} \right)}

\newcommand{\say}[1]{\textquotedblleft{#1}\textquotedblright} %quote the "argument"
\newcommand*\widefbox[1]{\fbox{\hspace{2em}#1\hspace{2em}}}
\newtcolorbox{mybox}[1][]{colback=white, sharp corners, #1}

%Line break spacings
\newcommand{\nl}{\vspace{0.1in}\noindent}
\newcommand{\nnl}{\vspace{0.2in}\noindent}
\newcommand{\nnnl}{\vspace{0.3in}\noindent}

\newcommand{\defn}[1]{\nnnl {\color{neonorange}\textbf{Definition:}} \textbf{\textit{{#1}}}}
\newcommand{\defnn}[1]{ {\color{neonorange}\textbf{Definition:}} \textbf{\textit{{#1}}}}
\newcommand{\properties}[1]{\nnnl {\color{eblue}\textbf{#1}} }

\newcommand{\header}[1]{\nnl {\color{eblue}\textbf{#1}} }

% Code snippets
\newcommand*{\code}{\fontfamily{qcr}\selectfont}
\lstset{
    backgroundcolor=\color{white},
    basicstyle=\footnotesize,
    breakatwhitespace=false,         % sets if automatic breaks should only happen at whitespace
    breaklines=true,                 % sets automatic line breaking
    captionpos=b,                    % sets the caption-position to bottom
    commentstyle=\color{dgray},    % comment style
    deletekeywords={...},            % if you want to delete keywords from the given language
    escapeinside={(*@}{@*)},          % if you want to add LaTeX within your code
    extendedchars=true,              % lets you use non-ASCII characters; for 8-bits encodings only, does not work with UTF-8
    firstnumber=1,                % start line enumeration with line 1
    frame=single,	                   % adds a frame around the code
    keepspaces=true,                 % keeps spaces in text, useful for keeping indentation of code (possibly needs columns=flexible)
    keywordstyle=\color{neonorange},       % keyword style
    language=C++,                 % the language of the code
    morekeywords={*,...},            % if you want to add more keywords to the set
    numbers=left,                    % where to put the line-numbers; possible values are (none, left, right)
    numbersep=5pt,                   % how far the line-numbers are from the code
    numberstyle=\tiny\color{mygray}, % the style that is used for the line-numbers
    rulecolor=\color{black},         % if not set, the frame-color may be changed on line-breaks within not-black text (e.g. comments (green here))
    showspaces=false,                % show spaces everywhere adding particular underscores; it overrides 'showstringspaces'
    showstringspaces=false,          % underline spaces within strings only
    showtabs=false,                  % show tabs within strings adding particular underscores
    stringstyle=\color{purple},     % string literal style
    tabsize=4,	                   % sets default tabsize to 4 spaces
}

\lstdefinestyle{cpp}{language=C++,
    morekeywords={cout, cin, Comparable, T},numbers=none
}

\newcommand{\FA}{\;\, \forall \;\,}
\newcommand{\fa}{\text{ for all }}
\newcommand{\fe}{\text{ for every }}
\renewcommand{\Re}{\operatorname{Re}}
\renewcommand{\Im}{\operatorname{Im}}
\newcommand{\boldit}[1]{\textit{\textbf{{#1}}}}

\begin{document}

\pagestyle{fancy}
\fancyhf{}
\fancyhead[RO]{Matthew Wilder} %header top right
\fancyhead[LO]{MTH 307 - Notes} %header top left
\fancyfoot[CO]{Page \thepage} %page center bottom


\newcommand{\per}[1]{{#1}^{\perp}}
\section*{5.A Invariant Subspaces} \defn{invariant subspace}\\
Suppose $T \in \L{V}$. A subspace $U$ of $V$ is called \boldit{invariant} under $T$ if $u \in U$ implies $Tu \in U$.

\defn{eigenvalue}\\
Suppose $T \in \L{V}$. A number $\lambda \in \F$ is called an \boldit{eigenvalue} of $T$ if there exists $v \in V$ such that $v \neq 0$ and $Tv = \lambda v$.

\header{Equivalent conditions to be an eigenvalue}\\
Suppose $V$ is finite-dimensional, $T \in \L{V}$, and $\lambda \in \F$. Then TFAE:
\begin{enumerate}[label=(\alph*)]
    \item $\lambda$ is an eigenvalue of $T$.
    \item $T - \lambda I$ is not injective.
    \item $T - \lambda I$ is not surjective.
    \item $T - \lambda I$ is not invertible.
\end{enumerate}

\defn{eigenvector}\\
Suppose $T \in \L{V}$ and $\lambda \in \F$ is an eigenvalue of $T$. A vector $v \in V$ is called an \boldit{eigenvector} of $T$ corresponding to $\lambda$ if $v \neq 0$ and $Tv = \lambda v$.

\header{Linearly independent eigenvectors}\\
Let $T \in \L{V}$. Suppose $\lambda_1, \dots, \lambda_m$ are distinct eigenvalues of $T$ and $v_1, \dots, v_m$ are corresponding eigenvectors. Then $v_1, \dots, v_m$ is linearly independent.

\header{Number of eigenvalues}\\
Suppose $V$ is finite-dimensional. Then each operator on $V$ has at most $\dim V$ distinct eigenvalues.

\defn{$T|_U$ \textbf{and} $T/U$}\\
Suppose $T \in \L{V}$ and $U$ is a subspace of $V$ invariant under $T$. 
\begin{itemize}
    \item The \boldit{restriction operation} $T|_U \in \L{U}$ is defined by 
    $$T|_U(u) = Tu \; \forall \; u \in U.$$
    \item The \boldit{quotient operator} $T/U \in \L{V/U}$ is defined by
    $$(T/U)(v+U) = Tv + U \; \forall \; v \in V.$$
\end{itemize}
\section*{5.B Eigenvectors and Upper-Triangular Matrices} \defn{$T^m$}\\
Suppose $T \in \L{V}$ and $m$ is a positive integer.
\begin{itemize}
    \item $T^m$ is defined by 
    $$T^m = \underbrace{T \cdots T.}_{m \text{ times}}$$
    \item $T^0$ is defined to be the identity operator $I$ on $V$.
    \item If $T$ is invertible with inverse $T^{-1}$, then $T^{-m}$ is defined by 
    $$T^{-m} = (T^{-1})^m.$$
\end{itemize}

\defn{$p(T)$}\\
Suppose $T \in \L V$ and $p \in \mathcal P (\F)$ is a polynomial given by 
$$p(z) = a_0 + a_1z + a_2z^2 + \cdots + a_m z^m$$
for $z \in \F$. Then $p(T)$ is the operator defined by 
$$p(T) = a_0I + a_1T + a_2T^2 + \cdots + a_mT^m.$$

\defn{product of polynomials}
\\ If $p, q \in \mathcal P (\F)$, then $pq \in \mathcal P (\F)$ is the polynomial defined by
$$(pq)(z) = p(z)q(z)$$
for $z \in \F$.
\section*{6.A Inner Products and Norms} \defn{norm}\\The length of a vector $x$ is called the \textbf{norm} of $x$ denoted $\norm{x}$ The norm is non-linear. 
$$\norm{x} = \sqrt{x_1^2 + \cdots + x_n^2}.$$


\defn{dot product} \\ For $x,y \in \R^n$, the \textbf{\textit{dot product}} of $x$ and $y$, denoted $x \cdot y$ (or $x.y$) is defined by
$$x \cdot y = x_1y_1 + \cdots + x_ny_n,$$
where $x = (x_1, \dots , x_n)$ and $y = (y_1, \dots , y_n)$.

\nl Note that the dot product of two vectors in $\R^n$ is a scalar (number (i.e. 1, 7, 5.3, $\pi$)), not another vector.
$$x \cdot x = \norm{x}^2 \FA x \in \R^n$$

\properties{dot product properties} For a dot product on $\R^n$,
\begin{enumerate}[label=(\alph*)]
    \item $x \cdot x \geq 0 \FA x \in \R^n$
    \item $x \cdot x = 0 \iff x = 0$
    \item For a fixed $y \in \R^n$, the map from $\R^n \to \R$ that sends $x \in \R^n$ to $x \cdot y$ is linear
    \item $x \cdot y = y \cdot x \FA x,y \in \R^n \quad \text{(commutative)}$
\end{enumerate}

\defn{$\Re z,\; \Im z$}\\Suppose $z = a + bi$ where $a, b \in \R$.
\begin{itemize}
    \item The \boldit{real part} of $z$, denoted $\Re z$, is defined by $\Re z = a$.
    \item The \boldit{imaginary part} of $z$, denoted $\Im z$, is defined by $\Im z = b$.
\end{itemize}

\nl Thus for every $z \in \C$, we have
$$z = \Re z + (\Im z) i.$$

\defn{complex conjugate}\\Suppose $z \in \C$. The \boldit{complex conjugate} of $z \in \C$, denoted $\bar{z}$, is defined by
$$\bar{z} = \Re z - (\Im z)i.$$

\defn{absolute value}\\Suppose $z \in \C$. The \boldit{absolute value} of $z \in \C$, denoted $\abs{z}$, is defined by $$\abs{z} = \sqrt{(\Re z)^2 + (\Im z)^2}.$$

\properties{Complex properties}
\begin{itemize}
    \item sum of $z$ and $\bar{z}$ $$z + \bar{z} = 2 \Re z$$
    \item difference of $z$ and $\bar{z}$ $$z - \bar{z} = 2(\Im z)i$$
    \item product of $z$ and $\bar{z}$ $$z  \bar{z} = \abs{z}^2$$
    \item additivity and multiplicativity of complex conjugate $$\overline{w+z} = \bar{w} + \bar{z} \qquad \text{and} \qquad \overline{wz} = \bar{w} \bar{z}$$
    \item conjugate of a conjugate $$\overline{\bar{z}} = z$$
    \item real and imaginary parts are bounded by $\abs{z}$ $$\abs{\Re z} \leq \abs{z} \qquad \text{and} \qquad \abs{\Im z} \leq \abs{z}$$
    \item absolute value of a complex conjugate $$\abs{\bar{z}} = \bar{z}$$
    \item multiplicativity of absolute value $$\abs{wz} = \abs{w} \abs{z}$$
    \item Triangle Inequality $$\abs{w + z} \leq \abs{w} + \abs{z}$$
\end{itemize}
    

\nl Recall that for $\lambda = a + bi$ where $a,b \in \R$, then

\begin{itemize}
    \item The absolute value of $\lambda$, denoted $\abs{\lambda}$, is defined as $$\abs{\lambda} = \sqrt{a^2 + b^2}.$$
    \item The complex conjugate of $\lambda$, denoted $\bar{\lambda}$, is defined as $$\bar{\lambda} = a - bi.$$
    \item The absolute value squared is defined as $$\abs{\lambda}^2 = \lambda \bar{\lambda}$$
\end{itemize}

\nl For $z = (\thru{z}) \in \C^n$, we define the norm of $z$ by
$$\norm{z} = \sqrt{\abs{z_1}^2 + \cdots + \abs{z_n}^2}.$$

\nl These absolute values are needed because we want $\norm{z}$ to be a nonnegative scalar. Note that 
$$\norm{z}^2 = z_1 \overline{z_1} + \cdots + z_n \overline{z_n}.$$

\nl We want to think of $\norm{z}^2$ as an inner product of $z$ with itself, like we did in $\R^n$. The equation above suggests that for $w = (\thru{w}) \in \C^n$, the inner product of $w$ with $z$ should be
$$\inner{w,z} = w_1 \overline{z_1} + \cdots + w_n \overline{z_n}$$
$$\inner{z,w} = z_1 \overline{w_1} + \cdots + z_n \overline{w_n}$$

\nl If $\lambda \in \C$ then the notation $\lambda \geq 0$ means that $\lambda$ is a non-negative real number.

\nl We use the common notation $\inner{u,v}$ with angle brackets tod enote an inner product. Parenthesis can also be used, but then $(u,v)$ becomes confusing.

\defn{inner product}
\\An \boldit{inner product} on $V$ is a function that takes each ordered pair $(u,v)$ of elements of $V$ to a number $\inner{u,v} \in \F$ and has the following properties:

\nl \textbf{positivity: } 

$\inner{v,v} \geq 0 \FA v \in V$

\nl \textbf{definiteness: }

$\inner{v,v} = 0 \iff v = 0$

\nl \textbf{additivity in the first slot: }

$\inner{u+v, w} = \inner{u,w} + \inner{v,w} \FA u,v,w \in V$

\nl \textbf{homogeneity in the first slot: }

$\inner{\lambda u, v} = \lambda \inner{u,v} \FA \lambda \in \F \text{ and } u,v \in V$

\nl \textbf{conjugate symmetry: }

$\inner{u,v} = \conj{\inner{v,u}} \FA u,v\in V$

\nnnl Note: $x = \bar{x} \FA x \in \R$

\nl \textbf{Examples:}

\nl A \boldit{Euclidean inner product} on $\F^n$ is defined by
$$\inner{(\thru{w}),\; (\thru{z})} = w_1 \overline{z_1} + \cdots + w_n \overline{z_n}.$$

\nnl If $c_1,\dots,c_n$ are all positive numbers, then an inner product can be defined on $\F^n$ by
$$\inner{(\thru{w}),\; (\thru{z})} = c_1 w_1 \overline{z_1} + \cdots + c_n w_n \overline{z_n}.$$

\nnl An inner product can be defined on the vector space of continuous real-valued functions on the interval $[-1,\,1]$ by
$$\inner{f,g} = \int_{-1}^1 f(x)g(x)\,dx.$$

\nnl An inner product can be defined on $\mathcal{P}(\R)$ by 
$$\inner{p,q} = \int_0^{\infty} p(x)q(x)e^{-x}\,dx$$

\defn{inner product space}\\
An \boldit{inner product space} is a vector space $V$ along with an inner product on $V$.

\nl Unless otherwise stated, you can assume the inner product on $\F^n$ is defined as $$\inner{w,z} = w_1 \overline{z_1} + \cdots + w_n \overline{z_n}$$

\nnl \textbf{Notation}: For the rest of this chapter, $V$ denotes an inner product space over $\F$.

\properties{Basic properties of an inner product}
\begin{enumerate}[label=(\alph*)]
    \item For each fixed $u \in V$, the function that takes $v$ to $\inner{v,u}$ is a linear map from $V$ to $\F$.
    \item $\inner{0,u} = 0 \FA u \in V$
    \item $\inner{u,0} = 0 \FA u \in V$
    \item $\inner{u, v+w} = \inner{u,v} + \inner{u,w} \FA u,v,w \in V$
    \item $\inner{u,\lambda v} = \bar{\lambda} \inner{u,v} \FA \lambda \in \F \text{ and } u,v\in V$
    \item $\inner{u,v} = 0 \iff \inner{v,u} = 0 \FA u,v \in V$
\end{enumerate}

\defn{norm}
\\For $v \in V$, the \boldit{norm} of $v$, denoted $\norm{v}$, is defined by
$$\norm{v} = \sqrt{\inner{v,v}}.$$

\nl In other words, $\norm{v}^2 = \inner{v,v}$.

\properties{Basic properties of the norm}\\
Suppose $v \in V$. Then 
\begin{enumerate}[label=(\alph*)]
    \item $\norm{v} = 0 \iff v = 0$.
    \item $\norm{\lambda v} = \abs{\lambda}\norm{v} \FA \lambda \in \F$.
\end{enumerate}

\nl \textbf{Examples:}

\begin{enumerate}[label=(\alph*)]
    \item If $(\thru{z}) \in \F^n$ (with the usual Euclidean inner product), then
    $$\norm{(\thru{z})} = \sqrt{\abs{z_1}^2 + \cdots + \abs{z_n}^2}.$$
    \item In the vector space of continuous real-valued functions on $[1,\,-1]$ (with the inner product defined as $\inner{f,g} = \int_{-1}^1 f(x)g(x)\,dx$), then
    $$\norm{f} = \displaystyle \sqrt{\int_{-1}^1 \Bigl(f(x)\Bigr)^2\,dx}.$$
\end{enumerate}

\defn{orthogonal}
\\Two vectors $u,v \in V$ are called \boldit{orthogonal} if $\inner{u,v} = 0$.

\nl Instead of saying \say{$u$ and $v$ are orthogonal}, we sometimes say \say{$u$ is orthogonal to $v$}.

\nl For two vectors $u,v \in \R^2$, then $$\inner{u,v} = \norm{u}\norm{v}\cos \theta,$$
where $\theta$ is the angle between the two vectors. If $\theta = 90^{\circ} = \frac{\pi}{2} \text{ radians}$ then $\cos \theta = 0$. That is, if $u$ and $v$ are perpendicular the inner product is 0.

\header{Orthogonality and 0}
\begin{enumerate}[label=(\alph*)]
    \item 0 is orthogonal to every vector in $V$.
    \item 0 is the only vector in $V$ that is orthogonal to itself.
\end{enumerate}

\header{Pythagorean Theorem} \\ If $u$ and $v$ are orthogonal vectors in $V$. Then
$$\norm{u + v}^2 = \norm{u}^2 + \norm{v}^2$$
The converse is true if $V$ is a real inner product space. That is, if $\norm{u + v}^2 = \norm{u}^2 + \norm{v}^2$ then $u$ and $v$ are orthogonal.

\header{An orthogonal decomposition}
\\Suppose $u,v \in V$, with $v \neq 0$. Set $c = \dfrac{\inner{u,v}}{\norm v^2}$ and $w = u - \dfrac{\inner{u,v}}{\norm v^2}v$. Then 
$$\inner{w,v} = 0 \qquad \text{ and } \qquad u = cv + w.$$

\header{Cauchy-Schwarz Inequality}\\
Suppose $u,v\in V$. Then
$$\abs{\inner{u,v}} \leq \norm u \norm v.$$
Further, this is an \textit{equality} if and only if $u$ and $v$ are scalar multiples of one another. 

\header{Triangle Inequality}\\
Suppose $u,v \in V$. Then 
$$\norm{u+v}\leq \norm u + \norm v.$$
This is an equality if and only if $u,v$ are a nonnegative multiples of the other.

\header{Parallelogram Equality}
\\Suppose $u,v \in V$. Then $$\norm{u+v}^2 + \norm{u-v}^2 = 2 \pars{\norm u^2 + \norm v^2}.$$

\section*{6.B Orthonormal Bases} \defnn{orthonormal}
\begin{itemize}
    \item A lost of vectors is valled \boldit{orthonormal} if each vector in the list has norm 1 and is orthogonal to all the other vectors in the list.
    \item In other words, a list $e_1, \dots, e_m$ of vectors in $V$ is orthonormal if
    $$\inner{e_j, e_k} = \begin{cases} 1 & \text{if } j = k,\\0 & \text{if } j \neq j. \end{cases}$$
\end{itemize}

\nl Example, the standard basis in $\F^n$. 

\header{The norm of an orthonormal linear combination}\\
If $e_1, \dots, e_m$ is an orthonormal list of vectors in $V$, then 
$$\norm{a_1e_e + \cdots + a_m e_m}^2 = \abs{a_1}^2 + \cdots + \abs{a_m}^2$$
for all $a_1,\dots, a_m \in \F.$

\header{Corollary: An orthonormal list is linearly independent}\\
Every orthonormal list of vectors is linearly independent.

\defn{orthonormal basis}\\
An \boldit{orthonormal basis} of $V$ is an orthonormal list of vectors in $V$ that is also a basis of $V$.

\nl For example, the standard basis is an orthonormal basis of $\F^n$.

\header{An orthonormal list of the right length is an orthonormal basis}
\\Every orthonormal list of vectors in V with length $\dim V$ is an orthonormal basis of $V$.

\header{Writing a vector as linear combination of orthonormal basis}\\
Suppose $\thru{e}$ is an orthonormal basis of $V$ and $v \in V$. Then
$$v = \inner{v,e_1}e_1 + \cdots + \inner{v,e_n}e_n$$
and
$$\norm v^2 = \abs{\inner{v,e_1}}^2 + \cdots + \abs{\inner{v,e_n}}^2.$$

\header{Gram-Schmidt Procedure}\\
Suppose $v_1, \dots, v_m$ is a linearly independent list of vectors in $V$. Let $e_1 = \dfrac{v_1}{\norm{v_1}}$. \\For $j = 2, \dots, m$, define $e_j$ inductively by
$$e_j = \frac{v_j - \inner{v_j,e_1}e_1 - \cdots - \inner{v_j, e_{j-1}}e_{j-1}}{\norm{v_j - \inner{v_j,e_1}e_1 - \cdots - \inner{v_j, e_{j-1}}e_{j-1}}}.$$
Then $e_1, \dots, e_m$ is an orthonormal list of vectors in $V$ such that
$$\operatorname{span}(v_1,\dots,v_j) = \operatorname{span}(e_1,\dots,e_j)$$
for $j = 1,\dots, m$.

\header{Existance of orthonormal basis}\\
Every finite-dimensional inner product space has an orthonormal basis.

\header{Corollary: Orthonormal list extends to orthonormal basis}\\
Suppose $V$ is finite-dimensional. Then every orthonormal list of vectors in $V$ can be extended to an orthonormal basis of $V$. 

\header{Upper-triangular matrix with respect to orthonormal basis}\\
Suppose $T \in \L{V}$. If $T$ has an upper-triangular matrix with respect to some basis of $V$, then $T$ has an upper-triangular matrix with respect to some orthonormal basis of $V$.

\header{Schur's Theorem}\\
Suppose $V$ is a finite-dimensional complex vector space and $T \in \L{V}$. Then $T$ has an upper-triangular matrix with respect to some orthonormal basis of $V$.

\nnl Because linear maps into a scalar field play a special role, we give them a special name.

\defn{linear functional}
\\A \boldit{linear functional} on $V$ is a linear map from $V$ to $\F$. In other words, a linear function is an element of $\L{V,\F}.$

\header{Riesz Representation Theorem}\\
Suppose $V$ is finite-dimensional and $\varphi$ is a linear functional on $V$. Then there is a unique vector $u \in V$ such that
$$\varphi(v) = \inner{v,u} \FA v \in V,$$
where
$$u = \overline{\varphi(e_1)}e_1 + \cdots + \overline{\varphi(e_n)}e_n.$$
\section*{6.C Orthogonal Complements and Minimization Problems} 
\defnn{orthogonal complement,} $\per U$
\\If $U$ is a subset of $V$, then the \boldit{orthogonal complement} of $U$, denoted $\per U$, is the set of all vectors in $V$ that are orthogonal to every vector in $U$:
$$\per U = \set{v \in V : \inner{v,u} = 0 \fe u \in U}.$$
\noindent For example, if $U$ is a line in $\R^3$ then $\per U$ is the plain containing $0$ that is perpendicular to $U$. If $U$ was a plane in $\R^3$ then $\per U$ is a line passing through $0$ that is perpendicular to $U$.

\properties{Basic properties of orthogonal complement}
\begin{enumerate}[label=(\alph*)]
    \item If $U$ is a subset of $V$, then $\per U$ is a subspace of $V$. 
    \item $\per{\set{0}} = V$.
    \item $\per V = \set 0$.
    \item If $U$ is a subset of $V$, then $U \cap \per U \subseteq \set0$.
    \item If $U$ and $W$ are subsets of $V$ and $U \subseteq W$, then $\per W \subseteq \per U$. 
\end{enumerate}

\header{Direct sum of a subspace and its orthogonal complement}\\
Suppose $U$ is a finite-dimensional subspace of $V$. Then
$$V = U \oplus \per U$$

\header{Dimension of the orthogonal complement}\\
Suppose $V$ is finite-dimensional and $U$ is a subspace of $V$. Then 
$$\dim \per U = \dim V - \dim U.$$

\header{The orthogonal complement of the orthogonal complement}\\
Suppose $U$ is a finite-dimensional subspace of $V$. Then 
$$U = \per{\pars{\per U}}.$$

\defn{orthogonal projection,} $P_U$
\\Suppose $U$ is a finite-dimensional subspace of $V$. The \boldit{orthogonal projection} of $V$ onto $U$ is the operator $P_U \in \L{V}$ defined as follows:
\begin{center}For $v\in V$, write $v = u + w$, where $u \in U$ and $w \in \per U$. Then $P_Uv = u.$\end{center}

\properties{Properties of the orthogonal projection $P_U$}\\
Suppose $U$ is a finite-dimensional subspace of $V$ and $v\in V$. Then
\begin{enumerate}[label=(\alph*)]
    \item $P_U \in \L{V}$
    \item $P_U u = u \FA u \in U$
    \item $P_Uw = 0 \FA w \in \per U$
    \item $\range P_U = U$
    \item $\null P_U = \per U$
    \item $v - P_Uv \in \per U$
    \item ${P_U}^2 = P_U$
    \item $\norm{P_Uv} \leq \norm v$
    \item for every orthonormal basis $e_1, \dots, e_m$ of $U$,
    $$P_Uv = \inner{v,e_1}e_1 + \cdots + \inner{v,e_m}e_m.$$
\end{enumerate}

\header{Minimizing the distance to a subspace}\\
Suppose $U$ is a finite-dimensional subspace of $V$, $v \in V$, and $u \in U$. Then 
$$\norm{v - P_Uv} \leq \norm{v-u}.$$
Furthurmore, the inequality is an equality if and only if $u = P_Uv$.
\section*{7.A Self-Adjoint and Normal Operators} \defnn{adjoint,} $T^*$
\\Suppose $T \in \L{V,W}$. The \boldit{adjoint} of $T$ is the function $T^* : W \to V$ such that
$$\inner{Tv,w} = \inner{v,T^*w}$$
for every $v \in V$ and every $w \in W$.

\header{The adjoint is a linear map}\\
If $T \in \L{V,W}$, then $T^* \in \L{W,V}$.

\properties{Properties of the adjoint}
\begin{enumerate}[label=(\alph*)]
    \item $(S+T)^*=S^* + T^* \fa S,T \in \L{V,W}$
    \item $\pars{\lambda T}^* = \bar{\lambda}T^* \fa \lambda \in F \text{ and } T \in \L{V,W}$
    \item $\pars{T^*}^* = T \fa T \in \L{V,W}$
    \item $I^* = I$, where $I$ is the identity operator on $V$
    \item $(ST)^* = T^*S^* \fa T \in \L{V,W}$ and $S \in \L{W,U}$ (here $U$ is an inner product space over $\F$).
\end{enumerate}

\header{Null space and range of $T^*$}
\begin{enumerate}[label=(\alph*)]
    \item $\null T^* = \per{(\range T)}$
    \item $\range T^* = \per{(\null T)}$
    \item $\null T = \per{(\range T^*)}$
    \item $\range T = \per{(\null T^*)}$
\end{enumerate}

\defn{conjugate transpose}
\\The \boldit{conjugate transpose} of an $m$-by-$n$ matrix is the $n$-by-$m$ matrix obtained by transposing the matrix and taking the complex conjugate of each entry.

\header{The matrix of $T^*$}
\\Let $T \in \L{V,W}$. Suppose $\thru{e}$ is an orthonormal basis of $V$ and $f_1, \dots, f_m$ is an orthonormal basis of $W$. Then
$$\mathcal{M}\bigl( T^*, (f_1, \dots, f_m), (\thru e) \bigr)$$
is the conjugate transpose of 
$$\mathcal{M}\bigl( T, (\thru e), (f_1, \dots, f_m) \bigr)$$

\defn{self-adjoint}
\\An operator $T \in \L{V}$ is called \boldit{self-adjoint} if $T = T^*$. In other words, $T \in \L{V}$ is self-adjoint if and only if
$$\inner{Tv,w} = \inner{v,Tw}$$
for all $v,w \in V$.

\header{Eigenvalues of self-adjoint operators are real}\\
Every eigenvalue of a self-adjoint operator is real

\header{Over $\C$, $Tv$ is orthogonal to $v$ for all $v$ only for the 0 operator}\\
Suppose $V$ is a complex inner product space and $T \in \L V.$ Suppose $$\inner{Tv,v}=0$$ for all $v \in V$. Then $T =0$.

\header{Over $\C$, $\inner{Tv,v}$ is real for all $v$ only for self-adjoint operators}\\
Suppose $V$ is a complex inner product space and $T \in \L V.$ Then $T$ is self-adjoint if and only if 
$$\inner{Tv,v} \in \R$$
for every $v \in V$.

\header{If $T = T^*$ and $\inner{Tv,v} = 0$ for all $v$, then $T = 0$}\\
Suppose $T$ is a self-adjoint operator on $V$ such that
$$\inner{Tv, v} = 0$$
for all $v \in V$. Then $T = 0$.

\defn{normal}
\\An operator on an inner product space is called \boldit{normal} if it commutes with its adjoint. In other words, $T \in \L V$ is normal if $$TT^* = T^* T.$$

\header{$T$ is normal if and only if $\norm{Tv} = \norm{T^*v}$ for all $v$}\\
An operator $T \in \L V$ is normal if and only if $$\norm{Tv} = \norm{T^*v}$$ for all $v \in V$.

\header{For $T$ normal, $T$ and $T^*$ have the same eigenvectors}\\
Suppose $T \in \L V$ is normal and $v \in V$ is an eigenvector of $T$ with eigenvalue $\lambda$. Then $v$ is also an eigenvector of $T^*$ with eigenvalue $\bar{\lambda}$.

\header{Orthogonal eigenvectors for normal operators}\\
Suppose $T \in \L V$ is normal. Then the eigenvectors of $T$ corresponding to distinct eigenvalues are orthogonal.
\end{document} 