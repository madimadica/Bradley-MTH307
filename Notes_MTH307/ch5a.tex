\defn{invariant subspace}\\
Suppose $T \in \L{V}$. A subspace $U$ of $V$ is called \boldit{invariant} under $T$ if $u \in U$ implies $Tu \in U$.

\defn{eigenvalue}\\
Suppose $T \in \L{V}$. A number $\lambda \in \F$ is called an \boldit{eigenvalue} of $T$ if there exists $v \in V$ such that $v \neq 0$ and $Tv = \lambda v$.

\header{Equivalent conditions to be an eigenvalue}\\
Suppose $V$ is finite-dimensional, $T \in \L{V}$, and $\lambda \in \F$. Then TFAE:
\begin{enumerate}[label=(\alph*)]
    \item $\lambda$ is an eigenvalue of $T$.
    \item $T - \lambda I$ is not injective.
    \item $T - \lambda I$ is not surjective.
    \item $T - \lambda I$ is not invertible.
\end{enumerate}

\defn{eigenvector}\\
Suppose $T \in \L{V}$ and $\lambda \in \F$ is an eigenvalue of $T$. A vector $v \in V$ is called an \boldit{eigenvector} of $T$ corresponding to $\lambda$ if $v \neq 0$ and $Tv = \lambda v$.

\header{Linearly independent eigenvectors}\\
Let $T \in \L{V}$. Suppose $\lambda_1, \dots, \lambda_m$ are distinct eigenvalues of $T$ and $v_1, \dots, v_m$ are corresponding eigenvectors. Then $v_1, \dots, v_m$ is linearly independent.

\header{Number of eigenvalues}\\
Suppose $V$ is finite-dimensional. Then each operator on $V$ has at most $\dim V$ distinct eigenvalues.

\defn{$T|_U$ \textbf{and} $T/U$}\\
Suppose $T \in \L{V}$ and $U$ is a subspace of $V$ invariant under $T$. 
\begin{itemize}
    \item The \boldit{restriction operation} $T|_U \in \L{U}$ is defined by 
    $$T|_U(u) = Tu \; \forall \; u \in U.$$
    \item The \boldit{quotient operator} $T/U \in \L{V/U}$ is defined by
    $$(T/U)(v+U) = Tv + U \; \forall \; v \in V.$$
\end{itemize}