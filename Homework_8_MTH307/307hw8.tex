\documentclass[12pt]{article}
\usepackage[english]{babel}
\usepackage{array}
\usepackage{setspace}
\usepackage{graphicx}
\usepackage{sistyle} %\num{100000} for commas
\SIthousandsep{,}
\usepackage{fancyhdr}
\usepackage{listings} % For code listings, may break stuff
\usepackage{xcolor, diagbox, empheq, makecell, tcolorbox}
\usepackage[autostyle]{csquotes}
\usepackage{amssymb, amsthm, linguex, enumitem, amsmath}
\usepackage{tcolorbox} %dont know what this does
\usepackage[colorlinks=true, allcolors=blue]{hyperref}
\usepackage{lipsum}

\makeatletter   %% <- make @ usable in macro names
\newcommand*\notab[1]{%
  \begingroup   %% <- limit scope of the following changes
    \par        %% <- start a new paragraph
    \@totalleftmargin=0pt \linewidth=\columnwidth
    %% ^^ let other commands know that the margins have been reset
    \parshape 0
    %% ^^ reset the margins
    #1\par      %% <- insert #1 and end this paragraph
  \endgroup
}
\makeatother    %% <- revert @

\pagestyle{empty}

\textwidth 6.5in
\hoffset=-.65in
\textheight=9.5in
\voffset=-1.in

%Sets
\newcommand{\R}{\mathbb{R}}
\newcommand{\C}{\mathbb{C}}
\newcommand{\N}{\mathbb{N}}
\newcommand{\F}{\mathbb{F}}
\newcommand{\Cb}{\mathbf{C}}
\newcommand{\Fb}{\mathbf{F}}
\newcommand{\Rb}{\mathbf{R}}

%Misc

\newcommand{\pars}[1]{\left( {#1} \right) } %auto size parenthesis 
\newcommand{\brac}[1]{\left[ {#1} \right] } %auto size brackets around arg
\newcommand{\Brac}[1]{\bigg[ {#1} \bigg] } %auto size brackets around arg
\newcommand{\set}[1]{\left\{{#1}\right\}} %auto size curly braces around arg
\newcommand{\vbrac}[1]{\left\langle{#1}\right\rangle} %vector angle brackets
\newcommand{\conj}[1]{\overline{{#1}}}
\newcommand{\vconj}[1]{\overline{\vbrac{{#1}}}}
\newcommand{\ceiling}[1]{\left\lceil {#1} \right\rceil} %auto size ceiling around arg
\newcommand{\floor}[1]{\left\lfloor {#1} \right\rfloor} %auto size floor around arg


\newcommand{\inner}[1]{\left\langle{#1}\right\rangle} %auto size angle brackets<>
\newcommand{\norm}[1]{\left\| {#1} \right\|} % norm: ||v||
\newcommand{\abs}[1]{\left| {#1} \right|} % absolute value |v|
\newcommand{\ceil}[1]{\left\lceil {#1} \right\rceil} %auto size ceiling

\newcommand{\limn}{\lim_{n\to\infty}} %limit as n approaches infinity
\newcommand{\thru}[1]{{#1}_1, \dots, {#1}_n}
\newcommand{\sumthru}[1]{{#1}_1 + \cdots + {#1}_n}
\renewcommand{\over}[1]{\frac{1}{{#1}}}
\newcommand{\pfrac}[2]{\left( \frac{{#1}}{{#2}} \right) } %auto size parenthesis over fraction 
\newcommand{\pover}[1]{\left( \frac{1}{{#1}} \right) } %auto size parenthesis over fraction

%Boolean Algebra
\newcommand{\OR}{\,\lor\,}
\newcommand{\AND}{\,\land\,}

%Probability and Statistics
\newcommand{\xbar}{\bar{X}}
\newcommand{\ybar}{\bar{Y}}
\newcommand{\yn}{Y_1, \dots, Y_n}
\newcommand{\yx}{X_1, \dots, X_n}
\newcommand{\normDist}{N\left(\mu, \sigma^2\right)} %default normal distribution
\newcommand{\gammaDist}[2]{\operatorname{Gamma} \left( {#1},{#2} \right)}
\newcommand{\prob}[1]{P \left( {#1} \right) }
\newcommand{\E}[1]{E \left( {#1} \right) }
\newcommand{\Eb}[1]{E[ \,{#1}\, ]} %E bracket
\newcommand*{\V}[1]{V \left( {#1} \right) }
\newcommand{\Vb}[1]{V [ \,{#1}\, ] }
\newcommand{\that}{\hat{\theta}} %theta hat
\newcommand{\phat}{\hat{p}}
\newcommand{\psihat}{\hat{\psi}}
\newcommand{\Psihat}{\hat{\Psi}}

%Linear Algebra
\newcommand{\spanset}[1]{\operatorname{span}\left\{{#1} \right\}} % \spanset{v}  is  span{v}
\newcommand{\range}[1]{\operatorname{range}{#1}} %range
\renewcommand{\null}{\operatorname{null}} %null
\newcommand{\dimrange}[1]{\operatorname{dim}\operatorname{range}{#1}} % dimrange
\newcommand{\dimnull}[1]{\operatorname{dim}\operatorname{null}{#1}} % dimnull
\newcommand{\mat}[4]{\begin{bmatrix}{#1} & {#2}\\{#3}&{#4}\end{bmatrix}} % 2x2 matrix
\newcommand{\Mat}[9]{\begin{bmatrix}{#1} & {#2} & {#3}\\{#4}&{#5}&{#6}\\{#7}&{#8}&{#9}\end{bmatrix}} % 3x3 matrix
\newcommand{\vdouble}[2]{\begin{pmatrix}{#1}\\{#2}\end{pmatrix}} % 2 high vertical vector
\newcommand{\vtriple}[3]{\begin{pmatrix}{#1}\\{#2}\\{#3}\end{pmatrix}} %vertical vector parenthesis, 3 args
\renewcommand{\L}[1]{\mathcal{L}\left({#1}\right)} %Set of linear maps
\newcommand{\poly}[2]{\mathcal{P}_{#1}({#2})} %polynomial up to degree (arg)
\newcommand{\pf}{\mathcal{P}(\mathbf{F})} %set of all polynomials
\newcommand{\vfive}[5]{\begin{pmatrix}{#1}\\{#2}\\{#3}\\{#4}\\{#5}\end{pmatrix}} %vertical vector parenthesis, 5 args
\newcommand{\vfour}[4]{\begin{bmatrix}{#1}\\{#2}\\{#3}\\{#4}\end{bmatrix}} %vertical vector parenthesis, 5 args
\newcommand{\detx}[4]{\begin{vmatrix}{#1} & {#2}\\{#3}&{#4}\end{vmatrix}} % 2x2 determinant

\newcommand{\ev}[1]{\vec{\mathbf{e_{#1}}}}

\newcommand{\bu}{\vec{\mathbf{u}}}
\newcommand{\bv}{\vec{\mathbf{v}}}
\newcommand{\bw}{\vec{\mathbf{w}}}
\newcommand{\bzero}{\vec{\mathbf{0}}}

%colors
\definecolor{ggreen}{RGB}{0, 127, 0}
\definecolor{dgray}{RGB}{63,63,63}
\definecolor{neonorange}{RGB}{255,47,0}
\definecolor{mygray}{rgb}{0.5,0.5,0.5}
\newcommand{\red}[1]{\color{red}{#1}\color{black}}
\newcommand{\grn}[1]{\color{ggreen}{#1}\color{black}}
\newcommand{\blu}[1]{\color{blue}{#1}\color{black}}
\newcommand{\redx}[1]{\color{red}\not{#1}\color{black}}

\newcommand{\prt}[1]{ \sqrt{{#1}} }
\newcommand{\port}[1]{\left( \frac{1}{\sqrt{{#1}}} \right)}

\newcommand{\say}[1]{\textquotedblleft{#1}\textquotedblright} %quote the "argument"
\newcommand*\widefbox[1]{\fbox{\hspace{2em}#1\hspace{2em}}}
\newtcolorbox{mybox}[1][]{colback=white, sharp corners, #1}

%Line break spacings
\newcommand{\nl}{\vspace{0.1in}\noindent}
\newcommand{\nnl}{\vspace{0.2in}\noindent}
\newcommand{\nnnl}{\vspace{0.3in}\noindent}

% Code snippets
\newcommand*{\code}{\fontfamily{qcr}\selectfont}
\lstset{
    backgroundcolor=\color{white},
    basicstyle=\footnotesize,
    breakatwhitespace=false,         % sets if automatic breaks should only happen at whitespace
    breaklines=true,                 % sets automatic line breaking
    captionpos=b,                    % sets the caption-position to bottom
    commentstyle=\color{dgray},    % comment style
    deletekeywords={...},            % if you want to delete keywords from the given language
    escapeinside={(*@}{@*)},          % if you want to add LaTeX within your code
    extendedchars=true,              % lets you use non-ASCII characters; for 8-bits encodings only, does not work with UTF-8
    firstnumber=1,                % start line enumeration with line 1
    frame=single,	                   % adds a frame around the code
    keepspaces=true,                 % keeps spaces in text, useful for keeping indentation of code (possibly needs columns=flexible)
    keywordstyle=\color{neonorange},       % keyword style
    language=C++,                 % the language of the code
    morekeywords={*,...},            % if you want to add more keywords to the set
    numbers=left,                    % where to put the line-numbers; possible values are (none, left, right)
    numbersep=5pt,                   % how far the line-numbers are from the code
    numberstyle=\tiny\color{mygray}, % the style that is used for the line-numbers
    rulecolor=\color{black},         % if not set, the frame-color may be changed on line-breaks within not-black text (e.g. comments (green here))
    showspaces=false,                % show spaces everywhere adding particular underscores; it overrides 'showstringspaces'
    showstringspaces=false,          % underline spaces within strings only
    showtabs=false,                  % show tabs within strings adding particular underscores
    stringstyle=\color{purple},     % string literal style
    tabsize=4,	                   % sets default tabsize to 4 spaces
}

\lstdefinestyle{cpp}{language=C++,
    morekeywords={cout, cin, Comparable, T},numbers=none
}
%Examples:
%{\code while}
%
%{\code \begin{lstlisting}[language=C++]
%sum1 = 0;
%for (i = 1; i <= n; i *= 2)
%    for (j = 1; j <= n; j++)
%        sum1++;
%\end{lstlisting}}







\begin{document}

\pagestyle{fancy}
\fancyhf{}
\fancyhead[RO]{Matthew Wilder} %header top right
\fancyhead[LO]{MTH 307 - Homework \#8} %header top left
\fancyfoot[CO]{Page \thepage} %page center bottom

\noindent MATH 307 \\
Assignment \#8 \\  % 6.A, 6.B, 6.C
Due Friday, March 11, 2022

\nl For each problem, include the statement of the problem. Leave a blank line.  At the beginning of the next line, write \textbf{Solution} or \textbf{Proof} -- as appropriate.

\begin{enumerate}
\item Prove that
    \[
        16 \le (a+b+c+d)\left(\frac{1}{a} + \frac{1}{b} + \frac{1}{c} + \frac{1}{d} \right)
    \]
    for all positive numbers $a,b,c,d$.\\
    Hint: find two vectors having lots of square roots; compute an inner product and also use Cauchy-Schwarz.
    \begin{mybox}\textbf{Solution:} Direct proof.
    \begin{proof}
        Let scalars $a,b,c,d \in \R^+$ and $\bv, \bw \in \R^4$ such that $\bv := \brac{\sqrt{a}, \sqrt{b}, \sqrt{c}, \sqrt{d}}$ and $\displaystyle \bw := \brac{\over{\sqrt{a}},\over{\sqrt{b}}, \over{\sqrt{c}}, \over{\sqrt{d}}}$. By the Cauchy-Schwarz Inequality, $\abs{\inner{\bv,\bw}} \leq \norm{\bv} \norm{\bw}$.

        \noindent Simplifying the left hand side,
        \begin{align*}
            \abs{\inner{\bv,\bw}} &= \abs{\inner{\brac{\sqrt{a}, \sqrt{b}, \sqrt{c}, \sqrt{d}},\; \textstyle \brac{\over{\sqrt{a}},\over{\sqrt{b}}, \over{\sqrt{c}}, \over{\sqrt{d}}}}  }\\
            &= \abs{ \frac{\sqrt{a}}{\sqrt{a}} + \frac{\sqrt{b}}{\sqrt{b}} + \frac{\sqrt{c}}{\sqrt{c}} + \frac{\sqrt{d}}{\sqrt{d}} }\\
            &= 4.
        \end{align*}
        Simplifying the right hand side,
        \begin{align*}
            \norm{\bv} \norm{\bw} &=  \sqrt{\prt a^2 + \prt b^2 + \prt c^2 + \prt d^2 } \cdot \sqrt{ \port a^2 + \port b^2 + \port c^2 + \port d^2 } \\
            &= \sqrt{a + b + c + d} \cdot \sqrt{\over{a}+\over b + \over c + \over d}.
        \end{align*}

        \noindent Hence, by Cauchy-Schwarz, $\displaystyle 4 \leq   \sqrt{a + b + c + d} \cdot \sqrt{\over{a}+\over b + \over c + \over d}$. Squaring both sides of this inequality, we see that $\displaystyle 16 \leq (a + b + c + d) \cdot \pars{\over{a}+\over b + \over c + \over d}$.
    \end{proof}
\end{mybox}
\newpage
\item Prove or disprove: there is an inner product on $\mathbf{R}^2$ such that the associated norm is given by
    \[
        \|(x,y)\| = \max \{ |x|, |y| \}
    \]
    for all $(x,y) \in \mathbf{R}^2$.
\begin{mybox} \textbf{Solution:} False by counterexample.
\begin{proof}
    Let $\bv, \bw \in \R^2$ such that $\bv = \pars{2,2}$ and $\bw = \pars{2,-2}$. Computing the norms, we obtain $\norm{\bv} = \max\set{|2|,|2|} = 2$ and $\norm{\bw} = \max\set{|2|, |-2|} = 2$.

    \nl Taking the norm of the sum $\bv + \bw$,
    \begin{align*}\norm{\bv + \bw} &= \norm{(2,2) + (2,-2)} \\ &= \norm{(4,0)} \\ &= \max\set{|4|,|0|} \\ &= 4.
    \end{align*}
    Since $\bv \cdot \bw = (2,2) \cdot (2,-2) = 2\cdot 2 + 2 \cdot (-2) = 0$, $\bv$ and $\bw$ are orthogonal. Hence, by the Pythagorean Theorem, $\norm{\bv + \bw}^2 = \norm{\bv}^2 + \norm{\bw}^2$. However,\\
    $4^2 = 16 \neq 8 = 2^2 + 2^2$. Thus a contradiction to the Pythagorean Theorem.
\end{proof}
\end{mybox}

\item Suppose $V$ is a real inner product space.  Prove that
    $$
        \hspace{1in} \langle u,v \rangle = \frac{\|u+v \|^2 - \|u-v \|^2}{4} \qquad \text{for all } u, v \in V
    $$
    \begin{mybox} \textbf{Solution:} Direct proof. 
        \begin{proof} The following utilizes the fact that we are acting on a real vector space and the conjugates are trivial. As such, $\inner{\bu, \bv} = \inner{\bv,\bu}$ and $\inner{\bu, -\bv} = -\inner{\bu,\bv}$.
            \begin{align*}
                &\qquad \dfrac{\norm{\bu+\bv}^2-\norm{\bu-\bv}^2}{4} \\
                &= \over{4} \Brac{ 
                    \inner{\bu + \bv, \bu + \bv}
                - 
                    \inner{\bu-\bv, \bu-\bv}
                }\\
                &= \over{4}\Brac{
                    \norm{\bu} + \inner{\bu,\bv} + \inner{\bv,\bu} + \norm{\bv} - \Big(\norm{\bu} + \inner{\bu,-\bv} + \inner{-\bv,\bu} + \inner{-\bv,-\bv}\Big)
                }\\
                &= \over{4}\Brac{
                    \norm{\bu} + \inner{\bu,\bv} + \inner{\bv,\bu} + \norm{\bv} - \norm{\bu} - \inner{\bu,-\bv} - \inner{-\bv,\bu} - \inner{-\bv,-\bv}
                }\\
                &= \over{4}\Brac{
                    \norm{\bu} + \inner{\bu,\bv} + \inner{\bu,\bv} + \norm{\bv} - \norm{\bu} + \inner{\bu,\bv} + \inner{\bv,\bu} - \norm{\bv}
                }\\
                &= \over{4}\Brac{
                    \inner{\bu,\bv} + \inner{\bu,\bv} + \inner{\bu,\bv} + \inner{\bu,\bv}
                }\\
                &= \inner{\bu,\bv}
            \end{align*}
        \end{proof}
        \end{mybox}

\item Show that if $a_1, \ldots a_n \in \mathbf{R}$, then the square of the average of $a_1, \ldots , a_n$ is less than or equal to the average of $a_1^2, \ldots , a_n^2$.
\begin{mybox}\textbf{Solution:} Direct proof:
\begin{proof}
    Let $\bar{A} := \displaystyle \pars{ \over{n} \sum_{i=1}^n {a_i}}^2$ and $\bar{B} := \displaystyle \over{n} \sum_{i=1}^n {a_i^2}$. Will will show that $\bar{A} \leq \bar{B}$.
    
    \nnl Let $\bv, \bw \in \R^n$ such that $\bv = (\thru{a})$ and $\bw = (\thru{1})$. By the Cauchy-Schwarz Inequality we have $\abs{\inner{\bv,\bw}} \leq \norm{\bv} \norm{\bw}$. Computing the left hand side,
    \begin{align*}
        \abs{\inner{\bv,\bw}} &= \abs{\inner{ (\thru{a}), (1,\dots,1)}}\\
        &= \abs{ \sumthru{1a} }\\
        &= \abs{ \sumthru{a} }.
    \end{align*}
    
    \nnl Computing the right hand side,
    \begin{align*}
        \norm{\bv} \norm{\bw} &= \sqrt{a_1^2 + \cdots + a_n^2} \cdot \sqrt{1_1+\cdots+1_n}\\
        &= \sqrt{n \pars{a_1^2 + \cdots + a_n^2}}.
    \end{align*}
    
    \nnl Hence, $\abs{ \sumthru{a} } \leq \sqrt{n \pars{a_1^2 + \cdots + a_n^2}}$. Algebraically manipulating this inequality,
    \begin{align*}
        \abs{ \sumthru{a} } &\leq \sqrt{n \pars{a_1^2 + \cdots + a_n^2}}\\
        \pars{\abs{\sumthru{a}}}^2 &\leq n \pars{a_1^2 + \cdots + a_n^2} \tag{square both sides}\\
        \over{n^2}\pars{\abs{\sumthru{a}}}^2 &\leq \over{n}\pars{a_1^2 + \cdots + a_n^2}\tag{divide by $n^2$} \\
        \over{n^2}\pars{\sumthru{a}}^2 &\leq \over{n}\pars{a_1^2 + \cdots + a_n^2}\tag{property of absolute value} \\
        \pars{\frac{\sumthru{a}}{n}}^2 &\leq \pars{\frac{a_1^2 + \cdots + a_n^2}{n}}\tag{factor in $n$} \\
        \bar{A} &\leq \bar{B}.
    \end{align*} 
\end{proof}
\end{mybox}
\newpage
\item Convert $\mathcal{P}_2([0,1])$ into an inner product space by writing $\langle p,q \rangle = \int_0^1 p(x)\overline{q(x)} \; dx$ for $p,q \in \mathcal{P}_2([0,1])$. Find a complete orthonormal set in that space.

\nnl \textbf{Solution:} Direct computation via Gram Schmidt.

    \nl Let $\poly{2}{[0,1]}$ have the ordered basis $\set{1,x,x^2}$ denoted by $\set{v_1,v_2,v_3}$. Using the Gram Schmidt process, we will compute $e_1, e_2$ and $e_3$.

    \nnl For $e_1 = \dfrac{v_1}{\norm{v_1}}$, we compute as follows.
    \begin{align*}
        e_1 &= \frac{v_1}{\norm{v_1}}\\ &= \frac{1}{\norm{1}}\\
        &= \over{\inner{1,1}}\\
        &= \over{\int_0^1 1 \cdot 1 \, dx}\\
        &= \over{ x\big|_{x=0}^{x=1}}\\
        &= \over{1-0}\\
        &= 1.
    \end{align*}
    
    \newpage Next, let $\alpha = v_2-\inner{v_2,e_1}e_1$. Then $e_2 = \dfrac{v_2-\inner{v_2,e_1}e_1}{\norm{v_2-\inner{v_2,e_1}e_1}} = \dfrac{\alpha}{\norm{\alpha}}$. Computing $\alpha$,
    \begin{align*}
        \alpha &= v_2-\inner{v_2,e_1}e_1\\
            &= x-\inner{x,1}\cdot 1 \\
            &= x - 1 \int_0^1 1x \,dx\\
            &= x - \brac{\frac{x^2}{2}}_{x=0}^{x=1}\\
            &= x - \over{2}\brac{(1)^2 - (0)^2}\\
            &= x - \over{2}.
    \end{align*}
    And computing the norm,
    \begin{align*}
        \norm{\alpha} &= \norm{x - \over{2}}\\
            &= \sqrt{\inner{x - \over{2}, \; x - \over{2}}}\\
            &= \sqrt{\int_0^1 \pars{x - \over{2}}^2\,dx}\\
            &= \sqrt{\over{3}\brac{\pars{x - \over{2}}^3}_{x=0}^{x=1}}\\
            &= \sqrt{\over{3}\brac{\pars{1 - \over{2}}^3 - \pars{0 - \over{2}}^3}}\\
            &= \sqrt{\over{3}\brac{\over{8} - \pars{- \over{8}}}}\\
            &= \sqrt{\over{12}}\\
            &= \over{2\sqrt{3}}.
    \end{align*}
    And hence,
    \begin{align*}
        e_2 &= \frac{\alpha}{\norm{\alpha}}\\
        &= \frac{x - \over{2}}{\over{2\sqrt{3}}}\\
        &= 2\sqrt{3}\pars{x - \over{2}}.
    \end{align*}
    
    \nl Lastly, let $\beta = v_3 - \inner{v_3,e_1}e_1 - \inner{v_3}{e_2}e_2$. Then $e_3 = \dfrac{\beta}{\norm{\beta}}$. Computing $\beta$,
    \begin{align*}
        \beta &= v_3 - \inner{v_3,e_1}e_1 - \inner{v_3,e_2}e_2\\
        &= x^2 - \inner{x^2, 1} - \inner{x^2, 2\sqrt{3}\pars{x - \over{2}}}\cdot 2\sqrt{3}\pars{x - \over{2}}\\
        &= x^2 - \int_0^1 x^2 \, dx - 2\sqrt{3} \cdot 2\sqrt{3}\pars{x - \over{2}} \int_0^1 \pars{x^3 - \frac{x^2}{2}} \,dx\\
        &= x^2 - \int_0^1 x^2 \, dx - \pars{ 12x-6 } \int_0^1 \pars{x^3 - \frac{x^2}{2}} \,dx\\
        &= x^2 - \brac{\frac{x^3}{3}}_{x=0}^{x=1} - \pars{ 12x-6 } \brac{\frac{x^4}{4} - \frac{x^3}{6}}_{x=0}^{x=1}\\
        &= x^2 - \over{3} - \pars{ 12x-6 } \brac{\over{12}}\\
        &= x^2 - \over{3} - \pars{ x-\over{2} }\\
        &= x^2 - x + \over{6}.
    \end{align*}
    And
    \begin{align*}
        \norm{\beta} &= \sqrt{\inner{\beta, \beta}}\\
        &= \sqrt{\int_0^1 \pars{x^2 - x + \over{6}}^2\,dx}\\
        &= \sqrt{\int_0^1 \pars{x^4 - 2x^3 + \frac 43x^2 -\frac 13x + \over{36}}dx}\\
        &= \sqrt{\brac{\frac{x^5}{5} - \frac{x^4}{2} + \frac{4x^3}{9} - \frac{x^2}{6} + \frac{x}{36} }_{x=0}^{x=1}}\\
        &= \sqrt{\over 5 - \over 2 + \frac{4}{9} - \over{6} + \over{36}}\\
        &= \sqrt{\frac{36-90+80-30+5}{180}}\\
        &= \over{\sqrt{180}}\\
        &= \over{6\sqrt{5}}.
    \end{align*}
    Hence $\displaystyle
        e_3 = \frac{\beta}{\norm{\beta}} = \frac{x^2 - x + \over{6}}{(6\sqrt{5})^{-1}}
        = 6\sqrt{5}\pars{x^2 - x + \over{6}}.$
    Therefore the orthonormal basis of $\poly{2}{[0,1]}$ is
    $$\set{1,\quad 2\sqrt3 \pars{x-\over2}, \quad 6\sqrt{5}\pars{x^2 - x + \over{6}} }.$$

\newpage
\item Suppose $U$ is the subspace of $\mathbf{R}^4$ defined by
    \[
        U = \text{span }((1,2,3,-4), (-5,4,3,2)).
    \]
    Find an orthonormal basis of $U$ and an orthonormal basis of $U^\perp$.
    
    \nnl \textbf{Solution: } Applying Gram Schmidt 4 times. 
    
    \nl Before we do any Gram-Schmidt-ing we need to extend $U$ to a basis for $\R^4$. We'll choose $\hat{i}$ and $\hat{j}$ and verify that these are infact linear independent, thus forming a basis for $\R^4$. To do this, we will apply the definition of linearly independent and show that the zero vector is only obtainable from 0 coefficients.
    $$\vec{0} = a \vfour{1}{2}{3}{-4} + b \vfour{-5}{4}{3}{2} + c \underbrace{\vfour1000}_{\textstyle \hat{i}} + d \underbrace{\vfour0100}_{\textstyle \hat{j}}$$
    Therefore
    \begin{align*}
        a - 5b + c &= 0\\
        2a + 4b + d &= 0\\
        3a + 3b &= 0\\
        -4a + 2b &= 0.
    \end{align*}
    Which, the last 2 rows imply that $b = -a$ and $b = 2a$. Thus $2a = -a$ which is only possible if $a=0$. Similarly, since $b=-a=-0$, $b=0$. By the first row, $0 - 5(0) + c = 0$ implies $c=0$. Similarly, the second row $2(0) + 4(0) + d = 0$ implies $d=0$. Hence, $a = b = c = d = 0$ and thus the set is linearly independent and forms a basis of $\R^4$.

    \nl Denote the basis $\set{(1,2,3,-1),\;(-5,4,3,2),\;\hat{i},\;\hat{j} }$ by $\set{\bv_1,\bv_2,\bv_3,\bv_4}$

    \nnl For our first application,
    \begin{align*}
        \ev1 &= \dfrac{\bv_1}{\norm{\bv_1}} = \dfrac{\bv_1}{\sqrt{ \bv_1 \cdot \bv_1}} = \dfrac{(1,2,3,-4)}{\sqrt{(1,2,3,-4)\cdot (1,2,3,-4)}} = \dfrac{(1,2,3,-4)}{\sqrt{1^2+2^2+3^2+(-4)^2}} \\&= \over{\sqrt{30}}(1,2,3,-4).
    \end{align*}

    \nl For the second application, let $\bu_2$ denote $\bv_2 - \inner{\bv_2, \ev1} \ev1$.

    \nl We will first compute the value of $\bu_2$.
    \begin{align*}
        \bu_2 &= \bv_2 - \inner{\bv_2, \ev1} \ev1 \\&= \bv_2 -( \bv_2 \cdot \ev1 )\ev1\\
        &= \bv_2 - \over{\sqrt{30}}(-5\cdot 1 + 4 \cdot 2 + 3 \cdot 3 + 2 \cdot (-4))\ev1\\
        &= \bv_2 - \frac{4}{\sqrt{30}}\ev1\\
        &= (-5,4,3,2) - \frac{4}{\sqrt{30}}\pars{\over{\sqrt{30}}(1,2,3,-4)}\\
        &=  (-5,4,3,2) - \frac{4}{30}(1,2,3,-4)\\
        &= \pars{
            -\frac{77}{15},\;
            \frac{56}{15},\;
            \frac{39}{15},\;
            \frac{38}{15}
        }\\
        &= \over{15}\pars{-77,\; 56,\; 39,\; 38}
    \end{align*}
    Then
    \begin{align*}
        \ev2 &= \frac{\bu_2}{\norm{\bu_2}} = \frac{\bu_2}{\sqrt{\bu_2\cdot\bu_2}} = \bu_2 \cdot \frac{1}{\sqrt{\bu_2\cdot \bu_2}}\\
        &= \bu_2 \cdot \over{\sqrt{ \over{15}\pars{-77,\; 56,\; 39,\; 38} \cdot  \over{15}\pars{-77,\; 56,\; 39,\; 38}}}\\
        &= \bu_2 \cdot \frac{15}{\sqrt{\pars{-77,\; 56,\; 39,\; 38} \cdot \pars{-77,\; 56,\; 39,\; 38}}}\\
        &= \bu_2 \cdot \frac{15}{\sqrt{\num{12030}}}\\
        &=  \over{15}\pars{-77,\; 56,\; 39,\; 38} \cdot \frac{15}{\sqrt{\num{12030}}}\\
        &= \over{\sqrt{\num{12030}}}\pars{-77,\; 56,\; 39,\; 38}
    \end{align*}

    \nl For the third application, let $\bu_3$ denote $\bv_3 - \inner{\bv_3, \ev1}\ev1 - \inner{\bv_3, \ev2} \ev2$. Computing $\bu_3$,
    \begin{align*}
        \bu_3 &= \bv_3 - \inner{\bv_3, \ev1}\ev1 - \inner{\bv_3, \ev2} \ev2\\
        &= \bv_3 - \pars{\bv_3 \cdot \ev1}\ev1 - \pars{\bv_3\cdot \ev2} \ev2\\
        &= \bv_3 - \pars{1\cdot \over{\sqrt{30}} + 0 + 0 + 0}\ev1 - \pars{1 \cdot \frac{-77}{\sqrt{\num{12030}}}+0+0+0} \ev2\\
        &= \bv_3 - \pover{\sqrt{30}}\pover{\sqrt{30}}(1,2,3,-4) - \pfrac{-77}{\sqrt{\num{12030}}}\pover{\sqrt{\num{12030}}}\pars{-77,\; 56,\; 39,\; 38}\\
        &= \bv_3 - \over{30}\pars{1,2,3,-4} + \frac{77}{\num{12030}}(-77,56,39,38)\\
        &= (1,0,0,0) - \over{30}\pars{1,2,3,-4} + \frac{77}{\num{12030}}(-77,56,39,38)\\
        &= \frac{\num{12030}}{\num{12030}} \hat{i} - \frac{401}{\num{12030}}(1,2,3,-4) + \frac{77}{\num{12030}}(-77,56,39,38)\\
        &= \over{\num{12030}}(5700, 3510, 1800, 4530)\\
        &= \over{401}\pars{190, 117, 60, 151}
    \end{align*}
    Then
    \begin{align*}
        \ev3 &= \frac{\bu_3}{\norm{\bu_3}} = \frac{\bu_3}{\sqrt{\bu_3 \cdot \bu_3}} = \bu_3 \cdot \over{\sqrt{\bu_3 \cdot \bu_3}}
        \\ &= \bu_3 \over{\sqrt{ \over{401}\pars{190, 117, 60, 151} \cdot \over{401}\pars{190, 117, 60, 151} }}\\
        &= 401\bu_3 \cdot \over{\sqrt{\pars{190, 117, 60, 151}\cdot \pars{190, 117, 60, 151}}} \\
        &= 401\bu_3 \cdot \over{\sqrt{190^2 + 117^2 + 60^2 + 151^2}} \\
        &= 401\bu_3 \cdot \over{\sqrt{\num{76190}}} \\
        &= \over{401}\pars{190, 117, 60, 151} \cdot \frac{401}{\sqrt{\num{76190}}}
        \\ &= \over{\sqrt{\num{76190}}}\pars{190, 117, 60, 151}
    \end{align*}

    \newpage \noindent Lastly, for $\ev4$, let $\bu_4$ denote $\bv_4 - \inner{\bv_4, \ev1}\ev1  - \inner{\bv_4, \ev2}\ev2  - \inner{\bv_4, \ev3}\ev3$. Computing $\bu_4$,
    \begin{align*}
        \bu_4 &= \bv_4 - \inner{\bv_4, \ev1}\ev1  - \inner{\bv_4, \ev2}\ev2  - \inner{\bv_4, \ev3}\ev3\\
        &= \hat{j} - \inner{\hat{j}, \ev1}\ev1  - \inner{\hat{j}, \ev2}\ev2  - \inner{\hat{j}, \ev3}\ev3\\
        &= \hat{j} - (\hat{j} \cdot \ev1)\ev1 - (\hat{j} \cdot \ev2)\ev2 - (\hat{j}\cdot \ev3)\ev3\\
        &= \hat{j} - \pfrac{2}{\sqrt{30}}\ev1 - \pfrac{56}{\sqrt{\num{12030}}}\ev2 - \pfrac{117}{\sqrt{\num{76190}}}\ev3\\
        &= \hat{j} - \pfrac{2}{30} (1,2,3,-4)
        - \pfrac{56}{\num{12030}} \pars{-77,\; 56,\; 39,\; 38}
        - \pfrac{117}{\num{76190}}\pars{190, 117, 60, 151}
        \\
        &= \hat{j} - \pfrac{1}{15} (1,2,3,-4)
        - \pfrac{28}{6015} \pars{-77,\; 56,\; 39,\; 38}
        - \pfrac{117}{\num{76190}}\pars{190, 117, 60, 151} \\
        &= \frac{\num{228570}}{\num{228570}}\hat{j} - \pfrac{15238}{\num{228570}}(1,2,3,-4) - \pfrac{1444}{\num{228570}}\pars{-77,\; 56,\; 39,\; 38} \\ & \hspace{2.7in} - \pfrac{351}{\num{228570}}\pars{190, 117, 60, 151} \\
        &= \over{\num{228570}} (29260,\; 76173,\; -123090, \;-46921)
    \end{align*}
    Then computing $\ev4$,
    \begin{align*}
        \ev4 &= \frac{\bu_4}{\norm{\bu_4}} = \frac{\bu_4}{\sqrt{\bu_4 \cdot \bu_4}} = \bu_4 \cdot \frac{1}{\sqrt{\bu_4 \cdot \bu_4}}\\
        &= 228570\bu_4 \cdot \over{\sqrt{29260^2 + 76173^2 + (-123090)^2+(-46921)^2}}\\
        &= 228570\bu_4 \cdot \over{\sqrt{\num{24011201870}}}\\
        &= \frac{228570}{228570} (29260,\; 76173,\; -123090, \;-46921) \cdot \over{\sqrt{\num{24011201870}}}\\
        &=  \over{\sqrt{\num{24011201870}}} (29260,\; 76173,\; -123090, \;-46921)
    \end{align*}
    Hence our orthonormal basis of $U$ is 
    $$\set{ \over{\sqrt{30}}(1,2,3,-4), \quad  \over{\sqrt{\num{12030}}}\pars{-77,\; 56,\; 39,\; 38} }$$
    
    \nnl And our orthonormal basis of $U^{\perp}$ is
    $$\set{\over{\sqrt{\num{76190}}}\pars{190, 117, 60, 151}, \quad \over{\sqrt{\num{24011201870}}} (29260,\; 76173,\; -123090, \;-46921) }$$
\end{enumerate}
\end{document} 