\documentclass[12pt]{article}
%\usepackage[document]{ragged2e}
\usepackage{array, amssymb, amsthm, linguex, enumerate, amsmath, physics, enumitem, xcolor, graphicx, xparse}
\let\fg\undefined %remove linguex/siunitx naming clash
\usepackage[english]{babel}
\usepackage[letterpaper,top=2cm,bottom=2cm,left=3cm,right=3cm,marginparwidth=1.75cm]{geometry}
\usepackage[colorlinks=true, allcolors=blue]{hyperref}
\usepackage[group-separator={,}]{siunitx} %\num{12345} -> "12,345"
\usepackage{fancyhdr}
\usepackage{notomath}
\usepackage[T1]{fontenc}

%Number sets
\newcommand{\R}{\mathbb{R}}
\newcommand{\C}{\mathbb{C}}
\newcommand{\N}{\mathbb{N}}
\newcommand{\F}{\mathbb{F}}
\renewcommand{\Re}{\operatorname{Re}}
\renewcommand{\Im}{\operatorname{Im}}
\renewcommand{\L}[1]{\mathcal{L}\left({#1}\right)} %Linear Map

\newcommand{\pmp}{\,\pm\,} %add small extra space to \pm

\NewDocumentCommand{\ceil}{ s m }{% ceiling brackets
    \IfBooleanTF{#1}%
    {\lceil #2 \rceil}% starred: no-autosizing
    {\left\lceil #2 \right\rceil}% unstarred: autosizing
}

\NewDocumentCommand{\ceiling}{ s m }{% ceiling brackets
    \IfBooleanTF{#1}%
    {\lceil #2 \rceil}% starred: no-autosizing
    {\left\lceil #2 \right\rceil}% unstarred: autosizing
}

\NewDocumentCommand{\floor}{ s m }{% floor brackets
    \IfBooleanTF{#1}%
    {\lfloor #2 \rfloor}% starred: no-autosizing
    {\left\lfloor #2 \right\rfloor}% unstarred: autosizing
}

\NewDocumentCommand{\pars}{ s m }{% parenthesis
    \IfBooleanTF{#1}%
    {( #2 ) }% starred: no-autosizing
    {\left( #2 \right) }% unstarred: autosizing
}

\NewDocumentCommand{\inner}{ s m }{% inner product
    \IfBooleanTF{#1}%
    {\langle #2 \rangle}% starred: no-autosizing
    {\left\langle #2 \right\rangle}% unstarred: autosizing
}

\NewDocumentCommand{\innerconj}{ s m }{% inner product
    \IfBooleanTF{#1}%
    {\overline{\langle #2 \rangle}}% starred: no-autosizing
    {\overline{\left\langle #2 \right\rangle}}% unstarred: autosizing
}

\NewDocumentCommand{\brac}{ s m }{% brackets
    \IfBooleanTF{#1}%
    {[#2] }% starred: no-autosizing
    {\left[ #2 \right] }% unstarred: autosizing
}

%default latex bracket size naming
\newcommand{\biggbrac}[1]{\bigg[ {#1} \bigg] }
\newcommand{\bigbrac}[1]{\big[ {#1} \big] }
\newcommand{\Bigbrac}[1]{\Big[ {#1} \Big] }


\RenewDocumentCommand{\over}{ s m }{% fraction 1/arg
    \IfBooleanTF{#1}%
    {\dfrac{1}{#2}}% starred: dfrac
    {\frac{1}{#2}}% unstarred: normal frac
}

\NewDocumentCommand{\pover}{ s m }{% parenthesis around fraction (1/arg)
    \IfBooleanTF{#1}%
    {\left(\dfrac{1}{#2}\right)}% starred: dfrac
    {\left(\frac{1}{#2}\right)}% unstarred: normal frac
}

\NewDocumentCommand{\pfrac}{ s m m}{% parenthesis around fraction (arg1/arg2)
    \IfBooleanTF{#1}%
    {\left( \dfrac{{#2}}{{#3}} \right)}% starred: dfrac
    {\left( \frac{{#2}}{{#3}} \right)}% unstarred: normal frac
}


\newcommand{\Xbar}{\bar{X}}
\newcommand{\Ybar}{\bar{Y}}
\newcommand{\xbar}{\bar{x}}
\newcommand{\ybar}{\bar{y}}


\newcommand{\limn}{\lim_{n\to\infty}}

\newcommand{\gammaDist}[2]{\operatorname{Gamma} \left( {#1},{#2} \right)} %gamma distribution
\NewDocumentCommand{\normalDist}{s g g}{ %normal distibution
    \IfBooleanTF{#1} { % starred, no autosizing parenthesis
      \IfNoValueTF{#2}{ 
          N (\mu,\, \sigma^2 ) %\normalDist* "default" normal distribution N(\mu, \sigma^2)
        } {
            \IfNoValueTF{#3}{N (#2)}{} %\normalDist{arg} --> N(arg)
        }
      \IfNoValueTF{#3}{}{N ( #2, #3 )}  %\normalDist*{arg1}{arg2} --> N(arg1,arg2)
    }  % else (unstarred) autosize parenthesis
    {
        \IfNoValueTF{#2}{
            N \left(\mu,\, \sigma^2 \right) %\normalDist "default" normal distribution N(\mu, \sigma^2)
        } {
            \IfNoValueTF{#3}{N \left(#2\right)}{} %\normalDist{arg} --> N(arg)
        }
        \IfNoValueTF{#3}{}{N \left( #2, #3 \right)} %\normalDist{arg1}{arg2} --> N(arg1,arg2)
    }
}



%colors
\definecolor{ggreen}{RGB}{0, 127, 0}
\definecolor{dgray}{RGB}{63,63,63}
\definecolor{neonorange}{RGB}{255,47,0}
\definecolor{mygray}{rgb}{0.5,0.5,0.5}
\definecolor{eblue}{RGB}{0,74,127}
\newcommand{\red}[1]{\color{red}{#1}\color{black}}
\newcommand{\green}[1]{\color{ggreen}{#1}\color{black}}
\newcommand{\blue}[1]{\color{blue}{#1}\color{black}}
\newcommand{\setRed}{\color{red}}
\newcommand{\setBlack}{\color{black}}
\newcommand{\setBlue}{\color{blue}}
\newcommand{\setGreen}{\color{ggreen}}



\newcommand{\thru}[1]{{#1}_1, \dots, {#1}_n}
\newcommand{\sumThru}[1]{{#1}_1 + \cdots + {#1}_n}
\newcommand{\yn}{Y_1, \dots, Y_n} % Y_1, ..., Y_n
\newcommand{\xn}{X_1, \dots, X_n} % Y_1, ..., Y_n

%hats and tildes
\newcommand{\that}{\widehat{\theta}} % theta hat
\newcommand{\phat}{\widehat{p}} % p hat
\newcommand{\qhat}{\widehat{q}} % p hat
\newcommand{\psihat}{\widehat{\psi}} % psi hat
\newcommand{\Psihat}{\widehat{\Psi}} % Psi hat
\newcommand{\ptilde}{\widetilde{p}} % psi tilde
\newcommand{\Psitil}{\widetilde{\Psi}} % Psi tilde
\newcommand{\betah}{\widehat{\beta}} % beta hat

%2x2 matrix shortcuts
\newcommand{\detx}[4]{\begin{vmatrix}{#1} & {#2}\\{#3}&{#4}\end{vmatrix}} % 2x2 determinant
\newcommand{\bmat}[4]{\begin{bmatrix}{#1} & {#2}\\{#3}&{#4}\end{bmatrix}} % 2x2 matrix brackets
\renewcommand{\pmat}[4]{\begin{pmatrix}{#1} & {#2}\\{#3}&{#4}\end{pmatrix}} % 2x2 matrix parenthesis

%remove any enumerate/itemize indent temporarily
\makeatletter   %% <- make @ usable in macro names
\newcommand*\notab[1]{%
  \begingroup   %% <- limit scope of the following changes
    \par        %% <- start a new paragraph
    \@totalleftmargin=0pt \linewidth=\columnwidth
    %% ^^ let other commands know that the margins have been reset
    \parshape 0
    %% ^^ reset the margins
    #1\par      %% <- insert #1 and end this paragraph
  \endgroup
}
\makeatother    %% <- revert @


\newcommand{\dimrange}[1]{\operatorname{dim}\operatorname{range}{#1}} % dimrange
\newcommand{\dimnull}[1]{\operatorname{dim}\operatorname{null}{#1}} % dimnull
\newcommand{\range}[1]{\operatorname{range}{#1}} %range
\newcommand{\nullspace}{\operatorname{null}} %null

% polynomial notation
\NewDocumentCommand{\poly}{ s g g }{%
    \IfBooleanTF{#1} {
        \IfNoValueTF{#2} {
            \mathcal{P}(\mathbb{R})
        } {
            \mathcal{P}_{#2}(\mathbb{R})
        }
    } {
        \IfNoValueTF{#3} {
            {\mathcal{P}(#2)}
        } { %else
            {\mathcal{P}_{#2}(#3)}
        }
    }
}

\NewDocumentCommand{\bias}{ s m }{% bias(arg)
    \IfBooleanTF{#1}%
    {\operatorname{bias}(#2)}% starred: no autosizing
    {\operatorname{bias}\left(#2\right)}% unstarred: autosizing
}

\NewDocumentCommand{\MSE}{ s m }{% MSE(arg)
    \IfBooleanTF{#1}%
    {\operatorname{MSE}(#2)}% starred: no autosizing
    {\operatorname{MSE}\left(#2\right)}% unstarred: autosizing
}

\NewDocumentCommand{\Var}{ s m }{% variance with parenthesis V(arg)
    \IfBooleanTF{#1}%
    {\operatorname{Var}(#2)}% starred: no autosizing
    {\operatorname{Var}\left(#2\right)}% unstarred: autosizing
}

\NewDocumentCommand{\Varb}{ s m }{% variance with brackets V[arg]
    \IfBooleanTF{#1}%
    {\operatorname{Var}[\,#2\,]}% starred: no autosizing
    {\operatorname{Var}\left[\,#2\,\right]}% unstarred: has autosizing
}

\NewDocumentCommand{\Vb}{ s m }{% another renaming of variance with brackets V[arg]
    \IfBooleanTF{#1}%
    {\operatorname{Var}[\,#2\,]}% starred: no autosizing
    {\operatorname{Var}\left[\,#2\,\right]}% unstarred: has autosizing
}

\NewDocumentCommand{\E}{ s m }{% expectation with parenthesis E(arg)
    \IfBooleanTF{#1}%
    {\operatorname{E}(#2)}% starred: no autosizing
    {\operatorname{E}\left(#2\right)}% unstarred: has autosizing
}

\NewDocumentCommand{\Eb}{ s m }{% expectation with brackets E[arg]
    \IfBooleanTF{#1}%
    {\operatorname{E}[#2]}% starred: no autosizing
    {\operatorname{E}\left[#2\right]}% unstarred: has autosizing
}

\RenewDocumentCommand{\P}{ s m }{% probability with parenthesis Pr(arg)
    \IfBooleanTF{#1}%
    {\Pr (#2) }% starred: no autosizing
    {\Pr \left( #2 \right) }% unstarred: has autosizing
}

\NewDocumentCommand{\prob}{ s m }{% probability with parenthesis Pr(arg)
    \IfBooleanTF{#1}%
    {\Pr (#2) }% starred: no autosizing
    {\Pr \left( #2 \right) }% unstarred: has autosizing
}

\NewDocumentCommand{\eff}{ s m }{% efficiency with parenthesis eff(arg)
    \IfBooleanTF{#1}%
    {\operatorname{eff}(#2)}% starred: no autosizing
    {\operatorname{eff}\left(#2\right)}% unstarred: has autosizing
}

%vertical vector of up to 8 elements
\NewDocumentCommand\vvec{s m g g g g g g g}{%
    \IfBooleanTF{#1} {
        \begin{bmatrix}% if starred use brackets
            \IfNoValueTF{#2}{}{#2}
            \IfNoValueTF{#3}{}{\\#3}
            \IfNoValueTF{#4}{}{\\#4}
            \IfNoValueTF{#5}{}{\\#5}
            \IfNoValueTF{#6}{}{\\#6}
            \IfNoValueTF{#7}{}{\\#7}
            \IfNoValueTF{#8}{}{\\#8}
        \end{bmatrix}
    }  % else (unstarred) use parethesis
    {
        \begin{pmatrix}%
            \IfNoValueTF{#2}{}{#2}
            \IfNoValueTF{#3}{}{\\#3}
            \IfNoValueTF{#4}{}{\\#4}
            \IfNoValueTF{#5}{}{\\#5}
            \IfNoValueTF{#6}{}{\\#6}
            \IfNoValueTF{#7}{}{\\#7}
            \IfNoValueTF{#8}{}{\\#8}
        \end{pmatrix}
    }
}
\def\Cov{\operatorname{Cov}} %Covariance
\def\df{\text{df}} %degrees of freedom

\NewDocumentCommand{\example}{ s g }{% Example header
    \IfBooleanTF{#1}%
    {\vspace{0.1in}}% starred: 0.1in
    {\vspace{0.2in}}% unstarred: 0.2in
    \IfNoValueTF{#2} {\noindent\textbf{\color{eblue} Example: }}{\noindent\textbf{\color{eblue} Example (#2): }}
}
\NewDocumentCommand{\disc}{ s }{% Discussion header
    \IfBooleanTF{#1}%
    {\vspace{0.1in}\noindent\textbf{Discussion: } }% starred: 0.1in
    {\vspace{0.2in}\noindent\textbf{Discussion: } }% unstarred: 0.2in
}
\NewDocumentCommand{\defn}{ s }{% Definition header
    \IfBooleanTF{#1}%
    {\vspace{0.1in}\noindent\textbf{\color{neonorange} Definition: } }% starred: 0.1in
    {\vspace{0.2in}\noindent\textbf{\color{neonorange} Definition: } }% unstarred: 0.2in
}
\NewDocumentCommand{\reason}{ s }{% Reason header
    \IfBooleanTF{#1}%
    {\vspace{0.1in}\noindent\textbf{Reason:} }% starred: 0.1in
    {\vspace{0.2in}\noindent\textbf{Reason:} }% unstarred: 0.2in
}
\NewDocumentCommand{\recall}{ s }{% Recall header
    \IfBooleanTF{#1}%
    {\vspace{0.1in}\noindent\textit{Recall:} }% starred: 0.1in
    {\vspace{0.2in}\noindent\textit{Recall:} }% unstarred: 0.2in
}
\NewDocumentCommand{\remark}{ s }{% Remark header
    \IfBooleanTF{#1}%
    {\vspace{0.1in}\noindent\textit{Remark:} }% starred: 0.1in
    {\vspace{0.2in}\noindent\textit{Remark:} }% unstarred: 0.2in
}

\NewDocumentCommand{\soln}{ s }{% Remark header
    \IfBooleanTF{#1}%
    {\vspace{0.1in}\noindent\textbf{Solution: } }% starred: 0.1in
    {\vspace{0.2in}\noindent\textbf{Solution: } }% unstarred: 0.2in
}

\newcommand{\proj}[2]{\operatorname{proj}_{{#1}}{#2}} %projection
\newcommand{\wideand}{\qquad \text{and} \qquad}

\newcommand{\bu}[1]{\textbf{\underline{{#1}}} } %bold underline
\newcommand{\boldit}[1]{\textbf{\textit{{#1}}} } %bold italix

% put actual quotation marks "around something"
\newcommand{\say}[1]{\textquotedblleft{#1}\textquotedblright}

% max{arg} and min{arg}
\renewcommand{\max}[1]{\operatorname{max}\left\{ #1 \right\}}
\renewcommand{\min}[1]{\operatorname{min}\left\{ #1 \right\}}

%Create a new vspace line no indent
\newcommand{\nl}{\vspace{0.1in}\noindent}
\newcommand{\nnl}{\vspace{0.2in}\noindent}
\newcommand{\nnnl}{\vspace{0.3in}\noindent}
\textwidth=7.02in
\hoffset=-.45in
\begin{document}

\pagestyle{fancy}
\fancyhf{}
\fancyhead[RO]{Matthew Wilder} %header top right
\fancyhead[LO]{MTH 307 - Homework \#12} %header top left
\fancyfoot[CO]{Page \thepage} %page center bottom

\noindent MATH 307 \\
Assignment \#12 \\  % 6.A, 6.B, 6.C
Due Friday, April 22$^{\text{nd}}$, 2022

\nl For each problem, include the statement of the problem. Leave a blank line.  At the beginning of the next line, write \textbf{Solution} or \textbf{Proof} -- as appropriate.

\begin{enumerate}
  \item Suppose $T$ is a positive operator on $V$. Prove that $T$ is invertible if and only if $\langle Tv,v \rangle >0$ for every $v \in V$ with $v \ne 0$.

\begin{proof} ($\Longrightarrow$)\\
    Suppose $T$ is invertible. By the positivity hypothesis, there exists $R \in \L{V}$ such that $R^2 = T$. Hence, 
    \begin{align*}
        \inner{Tv, v}
        &= \inner{R^2v, v} \tag{Substitution}\\
        &= \inner{Rv, R^*v} \tag{Taking the adjoint}\\
        &= \inner{Rv, Rv} \tag{Self-adjoint square roots}\\
    \end{align*}
    Which, by the definiteness property of inner products, $\inner{Rv, Rv} = 0$ if and only if $Rv = 0$.
    
    \nl By positivity on $T$, then $R^2 = T$. Multiplying on the right by $T^{-1}$, then $R^2T^{-1} = TT^{-1} = I$. Therefore $R(RT^{-1}) = I$. Hence $R$ is invertible and $R^{-1} = RT^{-1}$. 
    
    \nl Thus, by invertibility $\nullspace R = \{0\}$. Since $v \neq 0$ by hypothesis, $Rv \neq 0$.  Therefore $\inner{Rv,Rv} > 0$ for $v \neq 0$ and hence $\inner{Tv,v} >0$.
\end{proof}

\begin{proof} ($\Longleftarrow$)\\
    Suppose $\inner{Tv,v} > 0$ for $v \neq 0$. This implies that $Tv \neq 0$ and $v \neq 0$ since $\inner{0,u} = 0 = \inner{u,0}$ for any $u \in V$. Thus $Tv \neq 0$ for all $v \neq 0$. Hence $T$ is injective, and thus invertible.
\end{proof}\newpage
  \item Suppose $T \in \mathcal{L}(V)$, for an inner product space $V$. For $u , v \in V$, define the function of two variables $\langle u , v \rangle_T$ by
    \[
        \langle u , v \rangle_T = \langle Tu , v \rangle.
    \]
Prove that $\langle \cdot , \cdot \rangle_T$ is an inner product on $V$ if and only if $T$ is an invertible positive operator (with respect to the original inner product $\langle \cdot , \cdot \rangle \; $).

\begin{proof} ($\Longrightarrow$)\\
    Suppose $\inner{\cdot, \cdot}_T$ is an inner product, we will show that $T$ is an invertible positive operator (with respect to the original $\inner{\cdot, \cdot}$). By positivity of inner products, $\inner{Tv,v} = \inner{v,v}_T \geq 0$, so $\inner{Tv,v} \geq 0$. For self-adjoint, 
    \begin{align*}
        \inner{Tu,v} &= \inner{u,v}_T\\
        &= \innerconj{v,u}_T\\
        &= \innerconj{Tv,u}\\
        &= \inner{u,Tv}.
    \end{align*}
    Therefore $T$ is self-adjoint and hence $T$ is a positive operator.
    
    \nl For invertibility, $\inner{v,v}_T = 0$ if and only if $v=0$, so $\inner{v,v}_T = \inner{Tv,v} = 0$ if and only if $v = 0^\dagger  $. Suppose $Tv = 0$ for some $v \neq 0$. Then $\inner{Tv,v} = \inner{0,v} = 0$. Hence, a contradiction to $^\dagger $. Thus $Tv = 0$ if and only if $v = 0$ and $\nullspace T = \{0\}$. Therefore $T$ is injective and hence invertible. 
\end{proof}

\begin{proof} ($\Longleftarrow$)\\
    Suppose $T$ is an invertible positive operator (with respect to the original $\inner{\cdot, \cdot}$). We will show that $\inner{\cdot, \cdot}_T$ is an inner product. That is, positivity, definiteness, additivity, homogeneity, and symmetry.
    %
    \notab{\textbf{positivity}}
    Since $T$ is positive we know that $\inner{Tv, v} \geq 0$. Since $\inner{v,v}_T = \inner{Tv,v}$ then $\inner{v,v}_T \geq 0$.
    %
    \notab{\textbf{definiteness}}
    By $T$'s positivity, $R^2 = T$ for a positive square root $R$ of $T$. Then $\inner{Tv,v} = \inner{R^2v, v} = \inner{Rv, Rv}$. As shown in \#1, since $T$ is positive and invertible then $R$ is also invertible and hence injective. So $Rv = 0$ if and only if $v = 0$ and hence $\inner{Rv,Rv} = \inner{Tv, v} = \inner{v,v}_T = 0$ if and only if $v = 0$. 
    %
    \notab{\textbf{additivity in the first slot}}
    Directly, $\inner{u+v,w}_T = \inner{T(u+v), w} = \inner{Tu,w} + \inner{Tv,w} = \inner{u,w}_T + \inner{v,w}_T$.
    %
    \notab{\textbf{homogeneity in first slot}}
    Directly, $\inner{\lambda u, v}_T = \inner{\lambda Tu,v} = \lambda \inner{Tu,v} = \lambda \inner{u,v}_T$.
    %
    \notab{\textbf{conjugate symmetry}}
    Directly with $T$ self-adjoint, $\inner{u,v}_T = \inner{Tu,v} = \innerconj{v,Tu} = \innerconj{Tv, u} = \innerconj{v,u}_T$.
\end{proof}\newpage
  \item Suppose $S \in \mathcal{L}(V)$. Prove that the following are equivalent:
\begin{enumerate}
    \item S is an isometry;
    \item $\langle S^*u , S^*v \rangle = \langle u , v \rangle$ for all $u, v \in V$;
    \item $S^*e_1 , \ldots S^*e_m$ is an orthonormal list for every orthonormal list of vectors $e_1 , \ldots , e_m$ in $V$;
    \item $S^*e_1 , \ldots S^*e_n$ is an orthonormal basis for some orthonormal basis $e_1 , \ldots , e_n$ of $V$.
    \end{enumerate}
    
    \nl We will prove $(a) \Rightarrow (b) \Rightarrow (c) \Rightarrow (d) \Rightarrow (a)$.
\begin{proof} (a $\implies$ b)\\
   Suppose $S$ is an isometry. Then taking the adjoint of the right hand side, $\inner{S^*u, S^*v} = \inner{SS^*u, v}$. Since $S$ is an isometry, then $SS^* = I$. Hence, $\inner{S^*u, S^*v} = \inner{u,v}$.
\end{proof}
\begin{proof} (b $\implies$ c)\\
    Suppose $\inner{S^*u, S^*v} = \inner{u,v}$ for all $u,v \in V$. Since $e_1, \dots, e_m$ is an orthonormal list then $\inner{e_j, e_k} = 0$ for $j \neq k$ and 1 for $j = k$. By (b), $\inner{S^*e_j, S^*e_k} = \inner{e_j, e_k}$, hence $S^*e_1, \dots, S^*e_m$ is also an orthonormal list in $V$ by equality.
\end{proof}
\begin{proof} (c $\implies$ d)\\
    Suppose $S^*e_1 , \ldots S^*e_m$ is an orthonormal list for every orthonormal list of vectors $e_1 , \ldots , e_m$ in $V$. Then we can extend $e_1, \dots, e_m$ to a basis of $V$ by the Gram Schmidt procedure with and apply (c) with $m = n = \dim V$.
\end{proof}
\begin{proof} (d $\implies$ a)\\
    Suppose $S^*e_1 , \ldots S^*e_n$ is an orthonormal basis for some orthonormal basis $e_1 , \ldots , e_n$ of $V$. Then $\inner{SS^*e_j, e_k} = \inner{S^*e_j, S^*e_k} = \inner{e_j, e_k}$, which is orthonormal. Every $u, v \in V$ can be written as a unique linear combination of $e_1, \dots, e_n$. Hence, $\inner{SS^*u, v} = \inner{u,v}$ and $SS^* = I$, a condition for an isometry.
\end{proof}\newpage
  \item Suppose $T_1 , T_2$ are normal operators on $\mathbf{F}^3$ and both operators have 2 , 5 , 7 as eigenvalues. Prove that there exists an isometry $S \in \mathcal{L}(\mathbf{F}^3)$ such that $T_1 = S^*T_2S$.

\begin{proof} $ $
    
    \nl By Theorem 7.22, since $T_1$ are normal, then the eigenvectors of $T_1$ corresponding to distinct eigenvalues are orthogonal. Since there are 3 distinct eigenvalues (namely 2, 5, and 7) and $\dim V = 3$, then we have an orthonormal basis for $\F^3$ consisting of the eigenvectors corresponding to distinct eigenvalues from $T_1$. Similarly $T_2$ has an orthonormal basis of eigenvectors corresponding to 2, 5, and 7.

    \nl Let $B_1 := \{e_1, e_2, e_3\}$ be an orthonormal basis for $\F^3$ corresponding to $T_1$'s eigenvectors. Similarly, let $B_2 := \{f_1, f_2, f_3\}$ be an orthonormal basis for $\F^3$ corresponding to $T_2$'s eigenvectors. For $S \in \L{\F^3}$, define $Se_i = f_i$ for $i = 1, 2, 3$. Then for the orthonormal basis $e_1, e_2, e_3$ of $\F^3$, $Se_1, \dots, Se_n = f_1, \dots, f_n$ is an orthonormal basis. Hence (d) $\Rightarrow$ (a) of 7.42 shows $S$ is an isometry. 

    \nl Therefore $S^*$ is an isometry and $S^* = S^{-1}$. Thus $S^*f_i = S^{-1}f_i = e_i$. Then, recall that $T_1 e_i = \lambda_i e_i$ and $T_2 f_i = \lambda_i f_i$. Then
    \begin{align*}
        T_1e_i &= \lambda_i e_i \tag{eigenvalues for eigenvectors}\\
        &= \lambda_i (S^* f_i) \tag{substitution}\\
        &= S^* (\lambda_i f_i) \tag{linearity}\\
        &= S^* (T_2 f_i) \tag{substitution}\\
        &= S^* (T_2 S e_i) \tag{substitution}
    \end{align*}
    Hence $T_1 e_i = S^* T_2 e_i$. Since $e_i$ forms a [orthonormal] basis $B_1$ for $\F^3$, then for any vector $v \in \F^3$ we have $v = a_1e_1 + a_2e_2 + a_3e_3$. So by linearity, 
    \begin{align*}
        T_1 v & = T_1 (a_1e_1 + a_2e_2 + a_3e_3) \tag{substitution of $v$}\\
        &= a_1(T_1 e_1) + a_2(T_1 e_2) + a_3(T_1 e_3) \tag{linearity}\\
        &= a_1(S^*T_2S e_1) + a_2(S^*T_2S e_2) + a_3(S^*T_2S e_3) \tag{substitution of $T_1 e_i = S^*T_2Se_i$}\\
        &= S^*T_2S(a_1e_1 + a_2 e_2 + a_3 e_3) \tag{linearity}\\
        &= S^*T_2S v \tag{substitution of $v$}
    \end{align*}
    Hence $T_1 = S^*T_2S$ and $S$ is an isometry.
\end{proof}\newpage
  \item Fix $u,  x \in V$ with $u \ne 0$. Define $T \in \mathcal{L}(V)$ by $Tv = \langle v , u \rangle x$ for every $v \in V$.  Prove that
    \[
      \sqrt{T^*T}v = \frac{\| x \|}{\| u \|} \langle v , u \rangle u
    \]
for every $v \in V$.

\begin{proof} $ $

  \nl First computing $T^*$ we have 
  \begin{align*}
    \inner{Tv, w} &= \inner{ \inner{v,u}x, \; w} \tag{substitution with given $Tv = \inner{v,u}x$}\\
    &= \inner{v,u}\inner{x,w} \tag{homogeneity in the first slot}\\
    &= \inner{v,\;\innerconj{x,w} u} \tag{second slot conjugate homogeneity}\\
    &= \inner{v,\; \inner{w,x}u} \tag{conjugate symmetry}\\
    &= \inner{v, T^*w}. \tag{take adjoint}
  \end{align*}
  Therefore $T^*w = \inner{w,x}u.$ Thus 
  \begin{align*}
    T^*Tv &= T^*\inner{v,u}x \tag{substitution}\\
    &= \inner{ \inner{v,u}x,x}u \tag{definition of $T^*$}\\
    &= \inner{v,u} \inner{x,x}u \tag{homogeneity in first slot}\\
    &= \inner{v,u} \norm{x}^2 u \\
    &= \inner{v,u} \norm{x}^2 u \frac{\inner{u,u}}{\norm{u}^2} \tag{fancy 1}\\
    &= \pfrac{\norm{x}}{\norm{u}}^2 \inner{v,u} \inner{u,u}u
  \end{align*}
  $$\text{*not finished*}$$
\end{proof}\newpage
  \item Give an example of $T \in \mathcal{L}(\mathbf{C}^2)$ such that 0 is the only eigenvalue of $T$ and the singular values of $T$ are 5 , 0.

\soln Define $T(x,y) = (5y, 0)$. Then using the standard basis
$$\mathcal M(T) = \begin{bmatrix}
    0 & 5 \\ 0 & 0
\end{bmatrix} \wideand \mathcal M(T) = \begin{bmatrix}
    0 & 0 \\ 5 & 0
\end{bmatrix}.$$
Therefore
$$T^*T = \begin{bmatrix}
    0 & 0 \\ 0 & 25
\end{bmatrix} \wideand \sqrt{T^*T} = \begin{bmatrix}
    0 & 0 \\ 0 & 5
\end{bmatrix}.$$
Because $T$ is an upper triangular matrix, the diagonal gives us its eigenvalues. Hence 0 is $T$'s eigenvalue with multiplicity 2. Then by diagonal matrix properties, the eigenvalues of $\sqrt{T^*T}$ are 5 and 0. Therefore the singular values of $T$ are 5, 0.\vspace{1.5in}
  \item Suppose $T \in \mathcal{L}(V)$ and $s$ is a singular value of $T$. Prove that there exists a vector $v \in V$ such that $\| v \| = 1$ and $\| Tv \| = s$.

\begin{proof} $ $

    \nl Let $s_1, \dots, s_n$ be the singular values of $T$. Let $s = s_1$. Then by SVD, there exists orthonormal bases $e_1, \dots, e_n$ and $f_1, \dots, f_n$ such that 
    $$Tv = s_1 \inner{v, e_1}f_1 + \cdots + s_n \inner{v,e_n}f_n$$ for every $v \in V$. If we choose $v = e_1$ then its clear that $\norm{v} = 1$ by normalized vector properties. Further, 
    \begin{align*}
        Tv &=  s_1 \inner{v, e_1}f_1 + \cdots + s_n \inner{v,e_n}f_n\\
        &=  s_1 \underbrace{ \inner{e_1, e_1} }_{\textstyle = 1} f_1 + s_2 \underbrace{\inner{e_1, e_2}}_{\textstyle = 0} f_2 + \underbrace{\cdots}_{ \textstyle = 0} + s_n \underbrace{\inner{e_1,e_n}}_{\textstyle = 0}f_n\\
        &= s_1 f_1 \\ &= s f_1.
    \end{align*}
    Then $\norm{Tv}^2 = \norm{sf_1}^2 = \inner{sf_1, sf_1} = \abs{s}^2 \inner{f_1, f_1} = \abs{s^2}$ since $f_1$ is normalized. Thus $\norm{Tv} = \abs{s}$. But singular values are non-negative by definition, hence $\abs{s} = s$ and therefore $\norm{Tv} = s$ for $v = e_1$. 
\end{proof}\vspace{2.5in}
  \item Suppose $T \in \mathcal{L}(\mathbf{C}^2)$ is defined by $T(x,y) = (-4y,x)$. Find the singular values of $T$.

\soln The transformation matrix for $T$ with respect to the standard basis is 
$$\mathcal{M}(T) = \begin{bmatrix}
    0 & -4 \\ 1 & 0
\end{bmatrix} \wideand \mathcal{M}(T^*) = \begin{bmatrix}
    0 & 1 \\ -4 & 0
\end{bmatrix}.$$
Then
$$T^*T = \begin{bmatrix}
    1 & 0 \\ 0 & 16
\end{bmatrix} \wideand \sqrt{T^*T} = \begin{bmatrix}
    1 & 0 \\ 0 & 4
\end{bmatrix}.
$$
As such, the eigenvalues of $\sqrt{T^*T}$ are the singular values by definition. Since the eigenvalues of a diagonal matrix are the diagonals, then the singular values are 4, 1.
\end{enumerate}
\end{document} 