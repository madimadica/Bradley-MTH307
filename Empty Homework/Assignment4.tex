\documentclass[12pt]{article}
\usepackage{setspace}
\usepackage{graphicx}
\usepackage{amsmath}
\usepackage{amssymb}
%\usepackage{mathabx}
\pagestyle{empty}

\textwidth 6.5in
\hoffset=-.65in
\textheight=9.5in
\voffset=-1.in

\begin{document}

\noindent MATH 307 \\
Assignment \#4 \\  %3.B, 3.C
Due Friday, February 11, 2022 \\
**Revised 07Feb2022 to correct typo in problem 5. Fixed a \LaTeX\ typo at the beginning of the first sentence to state that $D$ is a linear transformation from $\mathcal{P}_3(\mathbf{R})$ to $\mathcal{P}_2(\mathbf{R})$. And corrected parts (a) and (b) of \#6 changing $D$ to $A$.\\

For each problem, include the statement of the problem. Leave a blank line.  At the beginning of the next line, write \textbf{Solution} or \textbf{Proof} -- as appropriate.

\begin{enumerate}
\item \begin{enumerate}
    \item Find linear map $T : \mathbf{R}^4 \to \mathbf{R}^4$ so that range $T = \text{null } T$.
    \item Show that there is no linear map $T : \mathbf{R}^5 \to \mathbf{R}^5$ so that range $T = \text{null } T$.
    \end{enumerate}

\item Find a $4 \times 4$ matrix $M$ so that the range of $M$ is spanned by $(1,0,1,0)$ and $(0,1,0,1)$.

\item \begin{enumerate}
        \item Give an example of a linear map on a three-dimensional space with a two-dimensional range.
        \item Give an example of a linear map on a three-dimensional space with a two-dimensional null-space.
    \end{enumerate}

\item Let $T : V \to V$ be a linear map with a one-dimensional range.  Prove that $T^2 = cT$ for some scalar $c$.  (This means that $T(Tv) = cTv$ for all $v \in V$.)

\item Let $D \in \mathcal{L}(\mathcal{P}_3(\mathbf{R}), \mathcal{P}_2(\mathbf{R}))$ denote the differentiation map $Dp = p'$.  Example 3.34 gives the matrix of $D$ with respect to the usual bases for $\mathcal{P}_3(\mathbf{R})$ and $\mathcal{P}_2(\mathbf{R}$.  \\
    Find two new bases for $\mathcal{P}_3(\mathbf{R})$ and $\mathcal{P}_2(\mathbf{R}$ so that the matrix for $D$ with respect to these bases is
    \[
    \left(
      \begin{array}{cccc}
        1 & 0 & 0 & 0 \\
        0 & 1 & 0 & 0 \\
        0 & 0 & 1 & 0 \\
      \end{array}
    \right).
    \]


\item The general operation of finding an antiderivative is not a linear map because of the ``$+C$'' which means that any function has infinitely many antiderivatives.  Let's define a linear map from $\mathcal{P}_3(\mathbf{R})$ to $\mathcal{P}_4(\mathbf{R})$ that avoids ambiguity.  Let $A(a_0 + a_1x + a_2x^2 + a_3 x^3) = a_0x + (a_1/2)x^2 + (a_2/3)x^3 + (a_3/4) x^4$.
    \begin{enumerate}
    \item Find the matrix of $A$ with respect to the standard bases for $\mathcal{P}_3(\mathbf{R})$ and $\mathcal{P}_4(\mathbf{R})$.
    \item Find new bases for $\mathcal{P}_3(\mathbf{R})$ and $\mathcal{P}_4(\mathbf{R})$ so that the matrix for $A$ with respect to the new bases is
        \[
    \left(
      \begin{array}{cccc}
        1 & 0 & 0 & 0 \\
        0 & 1 & 0 & 0 \\
        0 & 0 & 1 & 0 \\
        0 & 0 & 0 & 1 \\
        0 & 0 & 0 & 0 \\
      \end{array}
    \right).
    \]
    \end{enumerate}


\end{enumerate}

\end{document} 