\documentclass[12pt]{article}
\usepackage{setspace}
\usepackage{graphicx}
\usepackage{amsmath}
\usepackage{amssymb}
%\usepackage{mathabx}
\pagestyle{empty}

\textwidth 6.5in
\hoffset=-.65in
\textheight=9.5in
\voffset=-1.in

\begin{document}

% Invariant Subspaces --- only one section; test on Wednesday

\noindent MATH 307 \\
Assignment \#6 \\   % 5.A
Due Friday, February 25, 2022 \\
Revision: 2/21/2022:  Fixed typo in problem \#3:  $U+V$ should be $U+W$.

For each problem, include the statement of the problem. Leave a blank line.  At the beginning of the next line, write \textbf{Solution} or \textbf{Proof} -- as appropriate.

\begin{enumerate}
\item Label the following statements as being true or false.
Provide some justification from the text for your label.
    \begin{enumerate}
        \item Every linear operator on an $n$-dimensional vector space has $n$ distinct eigenvalues.
        \item If a linear operator on a vector space over $\mathbf{R}$ has one eigenvector, then it has an infinite number of eigenvectors.
        \item There exists a square matrix with no eigenvectors.
        \item Eigenvalues must be nonzero scalars.
        \item Any two eigenvectors are linearly independent.
        \item The sum of two eigenvalues of a linear operator $T$ is also an eigenvalue of $T$.
        \item Linear operators on infinite-dimensional vectors spaces never have eigenvalues.
        \item The sum of two eigenvectors of an operator $T$ is always an eigenvector of $T$.
    \end{enumerate}

\item Consider the operator
        $T = \left(
               \begin{array}{cc}
                 0 & 0  \\
                 0 & 1 \\
               \end{array}
             \right)$
        acting on $\mathbf{R}^2$.  How many subspaces are there that are invariant under $T$?

\item If $U$ and $W$ are invariant subspaces for $T \in \mathcal{L}(V)$ then $U+W$ is invariant for $T$.

\item In $\mathbf{R}^2$, let $T$ be the reflection across the line $y=x$.
    \begin{enumerate}
    \item Write the matrix $A$ that represents $T$ relative to the standard basis.
    \item Determine two invariant subspaces $\mathcal{M}$ and $\mathcal{N}$ for $T$ such that $\mathbf{R}^2 = \mathcal{M} \oplus \mathcal{N}$ where neither $\mathcal{M}$ nor $\mathcal{N}$ is the zero subspace.
    \item Write a basis $\{ u_1, u_2 \}$ for $\mathbf{R}^2$ so that $\mathcal{M} = \text{span } (u_1)$ and  $\mathcal{N} = \text{span } (u_2)$.
    \item Write the matrix $B$ that represents $T$ relative to the basis $\{ u_1, u_2 \}$.
    \end{enumerate}

\item \begin{enumerate}
        \item Let $V = \mathbf{R}^2$. Find eigenvalues and eigenvectors for the linear operator $T$ defined by $T(x,y) = (2y , x)$.
        \item Let $V = \mathbf{R}^2$. Find eigenvalues and eigenvectors for the linear operator $T$ defined by $T(x,y) = (-2y , x)$.
        \end{enumerate}

\item Let $T \in \mathcal{L}(V)$ and $S \in \mathcal{L}(V)$ be invertible.
    \begin{enumerate}
    \item Show that $T$ and $S^{-1}TS$ have the same eigenvalues.
    \item What is the relationship between the eigenvectors of $T$ and the eigenvectors of $S^{-1}TS$.
    \end{enumerate}

\item Show that the operator $T \in \mathcal{L}(\mathbf{C}^\infty)$ defined by
    \[
    T(z_1,z_2, \ldots) = (0,z_1,z_2, \ldots)
    \]
    has no eigenvalues.

\item Suppose $T \in \mathcal{L}(V)$ and there exists nonzero vectors $v$ and $w$ so that
    \[
        Tv = 3w \hspace{.5in} \text{and} \hspace{.5in} Tw=3v.
    \]
    Prove that 3 or $-3$ is an eigenvalue of $T$.

\end{enumerate}

\end{document} 