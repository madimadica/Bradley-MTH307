\documentclass[12pt]{article}
\usepackage{setspace}
\usepackage{graphicx}
\usepackage{amsmath}
\usepackage{amssymb}
%\usepackage{mathabx}
\pagestyle{empty}

\textwidth 6.5in
\hoffset=-.65in
\textheight=9.5in
\voffset=-1.in

\begin{document}

\noindent MATH 307 \\
Assignment \#7 \\  % 5.B, 6.A
Due Friday, March 4, 2022 \\

For each problem, include the statement of the problem. Leave a blank line.  At the beginning of the next line, write \textbf{Solution} or \textbf{Proof} -- as appropriate.

\begin{enumerate}
\item Suppose $T \in \mathcal{L}(V)$ and $T^2 = I$ and $-1$ is not an eigenvalue of $T$.  Prove that $T = I$.

\item Suppose $P \in \mathcal{L}(V)$ and $P^2=P$.  Prove that $V = \text{null }P \oplus \text{range } P$.

% \item Suppose $T \in \mathcal{L}(V)$.  Prove that 9 is an eigenvalue of $T^2$ if and only if 3 or $-3$ is an eigenvalue of $T$.

\item Suppose $T \in \mathcal{L}(V)$ and $v$ is an eigenvector of $T$ with eigenvalue $\lambda$.  Suppose $p \in \mathcal{P}(\mathbf{R})$.  Prove that $p(T)v = p(\lambda)v$.

\item Suppose $W$ is a complex vector space and $T \in \mathcal{L}(W)$ has no eigenvalues.  Prove that every subspace of $W$ invariant under $T$ is either $\{0\}$ or infinite-dimensional.

\item Suppose $V$ is a finite-dimensional complex vector space and $T \in \mathcal{L}(V)$.  Define a function $f: \mathbf{C} \to \mathbf{R}$ by
    \[
        f(\lambda) = \dim \text{range}(T- \lambda I).
    \]
    Prove that $f$ is not a continuous function.

\item Suppose $T \in \mathcal{L}(V)$ has a diagonal matrix $A$ with respect to some basis of $V$ and that $\lambda \in \mathbf{F}$.  Prove that $\lambda$ appears on the diagonal of $A$ precisely $\dim E(\lambda , T)$ times.

\item Show that the function that takes $((x_1,x_2) , (y_1,y_2)) \in \mathbf{R}^2 \times \mathbf{R}^2$ to $|x_1y_1| + |x_2y_2|$ is not an inner product on $\mathbf{R}$.

\item Suppose $T \in \mathcal{L}(V)$ is such that $\|Tv\| \le \|v\|$ for every $v \in V$.  Prove that $T - \sqrt{2} I$ is invertible.

\item Suppose $\|u| = \|v\| = 1$ and $\langle u , v \rangle = 1$.  Prove that $u = v$.


\end{enumerate}

\end{document} 