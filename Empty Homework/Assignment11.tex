\documentclass[12pt]{article}
\usepackage{setspace}
\usepackage{graphicx}
\usepackage{amsmath}
\usepackage{amssymb}
%\usepackage{mathabx}
\pagestyle{empty}

\textwidth 6.5in
\hoffset=-.65in
\textheight=9.5in
\voffset=-1.in

\begin{document}

\noindent MATH 307 \\
Assignment \#11 \\
Due Friday, April 15, 2022 \\

For each problem, include the statement of the problem. Leave a blank line.  At the beginning of the next line, write \textbf{Solution} or \textbf{Proof} -- as appropriate.

\begin{enumerate}
\item \begin{enumerate}
        \item Show that
        $A=\left( \begin{array} {c c c} 1 & 1 & 1 \\ 1 & 1 & 1 \\1 & 1 & 1 \end{array} \right)$ is positive.

        \item Find all $\alpha$ such that
        $A=\left( \begin{array} {c c c} \alpha & 1 & 1 \\ 1 & 0 & 0 \\1 & 0 & 0 \end{array} \right)$ is positive.

        \item Show that even though all its entries are positive, the matrix
        $A=\left( \begin{array} {c c c} 2 & 2  \\ 2 & 1  \end{array} \right)$
        is not positive.

        \item Find an example of a positive matrix some of whose entries are negative.
        \end{enumerate}

\item If $T$ is a positive and invertible operator, is $T^{-1}$ positive?

\item Consider the three statements:
    \begin{enumerate}
        \item $T$ is self-adjoint
        \item $T$ is an isometry
        \item $T^2=I$ (such a $T$ is called an \emph{involution})
    \end{enumerate}
    Prove that if an operator has any two of the properties, then it has the third one as well.

\item Prove or give a counterexample:  If $T \in \mathcal{L}(V)$ and there exists an orthonormal basis $e_1, \ldots , e_n$ of  $V$ such that $\|Te_i\| = 1$ for each $e_i$, then $T$ is an isometry.

\item Suppose $T \in \mathcal{L}(V)$.  Prove that there exists an isometry $S \in \mathcal{L}(V)$ such that
    \[
        T = \sqrt{TT^*}\;S.
    \]


\item Find the singular values of the differentiation operator $D \in \mathcal{L}(\mathcal{P}_2(\mathbf{R}))$ defined by $Dp = p'$, where the inner product is $\langle p , q \rangle = \int_{-1}^1 p(x)q(x)\;dx$.\\
    Remark: It might be helpful to compute the matrix for $D$ with respect to the basis $1, x, x^2$ to find eigenvalues (easy) and then compute the matrix for $D$ again using an \emph{orthonormal basis} for $\mathcal{P}_2(\mathbf{R})$ to compute the singular values.  Use some technology for the integrations.

\item Define $T \in \mathcal{L}(\mathbf{F}^3)$ by $T(z_1,z_2,z_3) = ( 4z_2 , 5z_3 , z_1 )$. Find (explicitly) an isometry $S \in \mathcal{L}(\mathbf{F}^3)$ such that $T = S\;\sqrt{T^*T}$.

\item Suppose $T \in \mathcal{L}(V)$ is self-adjoint.  Prove that the singular values of $T$ equal the absolute values of the eigenvalues of $T$, repeated appropriately.


\end{enumerate}

\end{document} 