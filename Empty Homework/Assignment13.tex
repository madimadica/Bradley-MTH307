\documentclass[12pt]{article}
\usepackage{setspace}
\usepackage{graphicx}
\usepackage{amsmath}
\usepackage{amssymb}
%\usepackage{mathabx}
\pagestyle{empty}

\textwidth 6.5in
\hoffset=-.65in
\textheight=9.5in
\voffset=-1.in

\begin{document}

\noindent MATH 307 \\
Assignment \#13 \\
Due Monday, May 2, 2022 \\ CORRECTION: Due date is Monday 5/2; Problem \#5 inserted ``null" in the equation for V. April 26.\\

For each problem, include the statement of the problem. Leave a blank line.  At the beginning of the next line, write \textbf{Solution} or \textbf{Proof} -- as appropriate.

\begin{enumerate}
\item Define $T \in \mathcal{L}(\mathbf{C}^2)$ by $T(w,z) =  (0,w)$.  Find all generalized eigenvectors of $T$.

\item Define $T \in\mathcal{L}(\mathbf{C}^2)$ by $T(w,z) = (z,-w)$. Find the generalized eigenspaces corresponding to the distinct eigenvalues of $T$. (Note Example 5.8 is an analogous transformation.)

\item Suppose $T \in \mathcal{L}(V)$ and $\alpha , \beta \in \mathbf{F}$ with $\alpha \ne \beta$.  Prove that $G(\alpha, T) \cap G(\beta,T) = \{ 0 \}$.

\item Suppose that $T \in \mathcal{L}(\mathbf{C}^3)$ is defined by $T(z_1,z_2,z_3) = (z_2,z_3,0)$. Prove that $T$ has no square root.  More precisely, prove that there does not exist $S \in \mathcal{L}(\mathbf{C}^3)$ such that $S^2 = T$.

\item Suppose that $T \in \mathcal{L}(V)$ is not nilpotent.  Let $n = \dim V$.  Show that \\ $V = \text{null }T^{n-1} \oplus \text{range }T^{n-1}$.

\item Suppose $T \in \mathcal{L}(V)$. Suppose $S \in \mathcal{L}(V)$ is invertible.  Prove that $T$ and $S^{-1}TS$ have the same eigenvalues with the same multiplicities.

\item Suppose $V$ is a complex vector space and $T \in \mathcal{L}(V)$.  Prove that $V$ has a basis consisting of eigenvectors of $T$ if and only if every generalized eigenvector of $T$ is an eigenvector of $T$.

\item Define $N \in \mathcal{L}(\mathbf{F}^5)$ by
    \[
       N(x_1,x_2,x_3,x_4,x_5) = (2x_2,3x_3,-x_4,4x_5,0).
    \]
    Find a square root of $I+N$.

\item Suppose $\mathbf{F} = \mathbf{C}$ and $T \in \mathcal{L}(V)$.  Prove that there exists $D, N \in  \mathcal{L}(V)$ such that $T = D + N$, the operator $D$ is diagonalizable, $N$ is nilpotent, and $DN = ND$.
\end{enumerate}

\end{document} 