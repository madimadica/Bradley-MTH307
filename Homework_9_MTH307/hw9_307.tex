\documentclass[12pt]{article}
\usepackage[english]{babel}
\usepackage{array}
\usepackage{setspace}
\usepackage{graphicx}
\usepackage{sistyle} %\num{100000} for commas
\SIthousandsep{,}
\usepackage{fancyhdr}
\usepackage{listings} % For code listings, may break stuff
\usepackage{xcolor, diagbox, empheq, makecell, tcolorbox}
\usepackage[autostyle]{csquotes}
\usepackage{amssymb, amsthm, linguex, enumitem, amsmath}
\usepackage{tcolorbox} %dont know what this does
\usepackage[colorlinks=true, allcolors=blue]{hyperref}
\usepackage{lipsum}

\usepackage{notomath}
\usepackage[T1]{fontenc}

\makeatletter   %% <- make @ usable in macro names
\newcommand*\notab[1]{%
  \begingroup   %% <- limit scope of the following changes
    \par        %% <- start a new paragraph
    \@totalleftmargin=0pt \linewidth=\columnwidth
    %% ^^ let other commands know that the margins have been reset
    \parshape 0
    %% ^^ reset the margins
    #1\par      %% <- insert #1 and end this paragraph
  \endgroup
}
\makeatother    %% <- revert @

\pagestyle{empty}

\textwidth 6.5in
\hoffset=-.65in
\textheight=9.5in
\voffset=-1.in

%Sets
\newcommand{\R}{\mathbb{R}}
\newcommand{\C}{\mathbb{C}}
\newcommand{\N}{\mathbb{N}}
\newcommand{\F}{\mathbb{F}}
\newcommand{\Cb}{\mathbf{C}}
\newcommand{\Fb}{\mathbf{F}}
\newcommand{\Rb}{\mathbf{R}}

\renewcommand{\Re}{\operatorname{Re}}
\renewcommand{\Im}{\operatorname{Im}}
%Misc

\newcommand{\pars}[1]{\left( {#1} \right) } %auto size parenthesis 
\newcommand{\brac}[1]{\left[ {#1} \right] } %auto size brackets around arg
\newcommand{\Brac}[1]{\bigg[ {#1} \bigg] } %auto size brackets around arg
\newcommand{\set}[1]{\left\{{#1}\right\}} %auto size curly braces around arg
\newcommand{\vbrac}[1]{\left\langle{#1}\right\rangle} %vector angle brackets
\newcommand{\conj}[1]{\overline{{#1}}}
\newcommand{\vconj}[1]{\overline{\vbrac{{#1}}}}
\newcommand{\ceiling}[1]{\left\lceil {#1} \right\rceil} %auto size ceiling around arg
\newcommand{\floor}[1]{\left\lfloor {#1} \right\rfloor} %auto size floor around arg


\newcommand{\inner}[1]{\left\langle{#1}\right\rangle} %auto size angle brackets<>
\newcommand{\innerc}[1]{\overline{\left\langle{#1}\right\rangle}} %auto size angle brackets<>
\newcommand{\norm}[1]{\left\| {#1} \right\|} % norm: ||v||
\newcommand{\abs}[1]{\left| {#1} \right|} % absolute value |v|
\newcommand{\ceil}[1]{\left\lceil {#1} \right\rceil} %auto size ceiling

\newcommand{\limn}{\lim_{n\to\infty}} %limit as n approaches infinity
\newcommand{\thru}[1]{{#1}_1, \dots, {#1}_n}
\newcommand{\sumthru}[1]{{#1}_1 + \cdots + {#1}_n}
\renewcommand{\over}[1]{\frac{1}{{#1}}}
\newcommand{\pfrac}[2]{\left( \frac{{#1}}{{#2}} \right) } %auto size parenthesis over fraction 
\newcommand{\pover}[1]{\left( \frac{1}{{#1}} \right) } %auto size parenthesis over fraction

%Boolean Algebra
\newcommand{\OR}{\,\lor\,}
\newcommand{\AND}{\,\land\,}

%Probability and Statistics
\newcommand{\xbar}{\bar{X}}
\newcommand{\ybar}{\bar{Y}}
\newcommand{\yn}{Y_1, \dots, Y_n}
\newcommand{\yx}{X_1, \dots, X_n}
\newcommand{\normDist}{N\left(\mu, \sigma^2\right)} %default normal distribution
\newcommand{\gammaDist}[2]{\operatorname{Gamma} \left( {#1},{#2} \right)}
\newcommand{\prob}[1]{P \left( {#1} \right) }
\newcommand{\E}[1]{E \left( {#1} \right) }
\newcommand{\Eb}[1]{E[ \,{#1}\, ]} %E bracket
\newcommand*{\V}[1]{V \left( {#1} \right) }
\newcommand{\Vb}[1]{V [ \,{#1}\, ] }
\newcommand{\that}{\hat{\theta}} %theta hat
\newcommand{\phat}{\hat{p}}
\newcommand{\psihat}{\hat{\psi}}
\newcommand{\Psihat}{\hat{\Psi}}

%Linear Algebra
\newcommand{\spanset}[1]{\operatorname{span}\left\{{#1} \right\}} % \spanset{v}  is  span{v}
\newcommand{\range}[1]{\operatorname{range}{#1}} %range
\renewcommand{\null}{\operatorname{null}} %null
\newcommand{\dimrange}[1]{\operatorname{dim}\operatorname{range}{#1}} % dimrange
\newcommand{\dimnull}[1]{\operatorname{dim}\operatorname{null}{#1}} % dimnull
\newcommand{\mat}[4]{\begin{bmatrix}{#1} & {#2}\\{#3}&{#4}\end{bmatrix}} % 2x2 matrix
\newcommand{\Mat}[9]{\begin{bmatrix}{#1} & {#2} & {#3}\\{#4}&{#5}&{#6}\\{#7}&{#8}&{#9}\end{bmatrix}} % 3x3 matrix
\newcommand{\vdouble}[2]{\begin{pmatrix}{#1}\\{#2}\end{pmatrix}} % 2 high vertical vector
\newcommand{\vtriple}[3]{\begin{pmatrix}{#1}\\{#2}\\{#3}\end{pmatrix}} %vertical vector parenthesis, 3 args
\renewcommand{\L}[1]{\mathcal{L}\left({#1}\right)} %Set of linear maps
\newcommand{\poly}[2]{\mathcal{P}_{#1}({#2})} %polynomial up to degree (arg)
\newcommand{\pf}{\mathcal{P}(\mathbf{F})} %set of all polynomials
\newcommand{\vfive}[5]{\begin{pmatrix}{#1}\\{#2}\\{#3}\\{#4}\\{#5}\end{pmatrix}} %vertical vector parenthesis, 5 args
\newcommand{\vfour}[4]{\begin{bmatrix}{#1}\\{#2}\\{#3}\\{#4}\end{bmatrix}} %vertical vector parenthesis, 5 args
\newcommand{\detx}[4]{\begin{vmatrix}{#1} & {#2}\\{#3}&{#4}\end{vmatrix}} % 2x2 determinant

\newcommand{\ev}[1]{\vec{\mathbf{e_{#1}}}}

\newcommand{\bu}{\vec{\mathbf{u}}}
\newcommand{\bv}{\vec{\mathbf{v}}}
\newcommand{\bw}{\vec{\mathbf{w}}}
\newcommand{\bzero}{\vec{\mathbf{0}}}

\newcommand{\per}[1]{{#1}^{\perp}}

%colors
\definecolor{ggreen}{RGB}{0, 127, 0}
\definecolor{dgray}{RGB}{63,63,63}
\definecolor{neonorange}{RGB}{255,47,0}
\definecolor{mygray}{rgb}{0.5,0.5,0.5}
\newcommand{\red}[1]{\color{red}{#1}\color{black}}
\newcommand{\grn}[1]{\color{ggreen}{#1}\color{black}}
\newcommand{\blu}[1]{\color{blue}{#1}\color{black}}
\newcommand{\redx}[1]{\color{red}\not{#1}\color{black}}

\newcommand{\prt}[1]{ \sqrt{{#1}} }
\newcommand{\port}[1]{\left( \frac{1}{\sqrt{{#1}}} \right)}

\newcommand{\say}[1]{\textquotedblleft{#1}\textquotedblright} %quote the "argument"
\newcommand*\widefbox[1]{\fbox{\hspace{2em}#1\hspace{2em}}}
\newtcolorbox{mybox}[1][]{colback=white, sharp corners, #1}

%Line break spacings
\newcommand{\nl}{\vspace{0.1in}\noindent}
\newcommand{\nnl}{\vspace{0.2in}\noindent}
\newcommand{\nnnl}{\vspace{0.3in}\noindent}

% Code snippets
\newcommand*{\code}{\fontfamily{qcr}\selectfont}
\lstset{
    backgroundcolor=\color{white},
    basicstyle=\footnotesize,
    breakatwhitespace=false,         % sets if automatic breaks should only happen at whitespace
    breaklines=true,                 % sets automatic line breaking
    captionpos=b,                    % sets the caption-position to bottom
    commentstyle=\color{dgray},    % comment style
    deletekeywords={...},            % if you want to delete keywords from the given language
    escapeinside={(*@}{@*)},          % if you want to add LaTeX within your code
    extendedchars=true,              % lets you use non-ASCII characters; for 8-bits encodings only, does not work with UTF-8
    firstnumber=1,                % start line enumeration with line 1
    frame=single,	                   % adds a frame around the code
    keepspaces=true,                 % keeps spaces in text, useful for keeping indentation of code (possibly needs columns=flexible)
    keywordstyle=\color{neonorange},       % keyword style
    language=C++,                 % the language of the code
    morekeywords={*,...},            % if you want to add more keywords to the set
    numbers=left,                    % where to put the line-numbers; possible values are (none, left, right)
    numbersep=5pt,                   % how far the line-numbers are from the code
    numberstyle=\tiny\color{mygray}, % the style that is used for the line-numbers
    rulecolor=\color{black},         % if not set, the frame-color may be changed on line-breaks within not-black text (e.g. comments (green here))
    showspaces=false,                % show spaces everywhere adding particular underscores; it overrides 'showstringspaces'
    showstringspaces=false,          % underline spaces within strings only
    showtabs=false,                  % show tabs within strings adding particular underscores
    stringstyle=\color{purple},     % string literal style
    tabsize=4,	                   % sets default tabsize to 4 spaces
}

\lstdefinestyle{cpp}{language=C++,
    morekeywords={cout, cin, Comparable, T},numbers=none
}
%Examples:
%{\code while}
%
%{\code \begin{lstlisting}[language=C++]
%sum1 = 0;
%for (i = 1; i <= n; i *= 2)
%    for (j = 1; j <= n; j++)
%        sum1++;
%\end{lstlisting}}








\begin{document}

\pagestyle{fancy}
\fancyhf{}
\fancyhead[RO]{Matthew Wilder} %header top right
\fancyhead[LO]{MTH 307 - Homework \#9} %header top left
\fancyfoot[CO]{Page \thepage} %page center bottom

\noindent MATH 307 \\
Assignment \#9 \\  % 6.A, 6.B, 6.C
Due Friday, March 25, 2022

\nl For each problem, include the statement of the problem. Leave a blank line.  At the beginning of the next line, write \textbf{Solution} or \textbf{Proof} -- as appropriate.

\begin{enumerate}
\item Suppose $V$ is finite-dimensional and $P \in \mathcal{L}(V)$ is such that (1) $P^2 = P$ and (2) every vector in $\text{null } P$ is orthogonal to every vector in $\text{range }P$. Prove that there exists a subspace $U$ of $V$ such that $P = P_U$.\\
    Hint:  For $v \in V$, write $v = Pv + (v-Pv)$.

    \vspace{0.5in}

\begin{proof}  We will first show that $V = \range P \oplus \null P$. Let $u \in \range P$ and $w \in \null P$. Then we will show that every $v \in V$ can be expressed as $v = u + w$. Taking $P(v - Pv)$ we have $Pv - P^2v = Pv - Pv$ by hypothesis (1), which is clearly zero. Therefore $(v - Pv) \in \null P$. As for $Pv$, we know that $Pv \in \range P$ by definition of range. Therefore we can rewrite v as the sum of subsets
$$v = \underbrace{Pv}_{\in \; \range P} + \underbrace{(v-Pv)}_{\in \; \null P}.$$
To show it is a unique linear combination (direct sum), we need $\null P \cap \range P = \set0$. For every $x \in \null P \cap \range P$, we have $Px = 0$ and $x = Py$ for some $y \in V$. By hypothesis (1), $P^2y = Py = P(Py) = Px = 0$. Thus, $\null P \cap \range P = \set0$ and
$$v = \range P \oplus \null P.$$

Hence every $v \in V$ can be written as a unique linear combination $v = Pv + (v - Pv)$. By hypothesis (2), we have $\null P \subseteq \pars{\range P}^{\perp}$. For $U := \range P$ (a subspace), then
$$P_Uv = P_U(Pv + v - Pv) = P_U\underbrace{(Pv)}_{\in \; U} + P_U\underbrace{(v-Pv)}_{\in \; \per U} = Pv + 0 = Pv.$$ 
%Factoring leads to $P(Px-x) = 0$  
    %We need to write $v \in V$ as a sum of $z$ That is, the intersection is the zero vector and every $v \in V$ can be written as a linear combination of $\range P$ and $\null P$. 

    %By hypothesis (2) we have $\null P \subset \pars{\range P}^{\perp}$.
\end{proof}\newpage
\item Suppose $V$ is finite-dimensional, $T \in \mathcal{L}(V)$ and $U$ is a subspace of $V$.  Prove that $U$ is invariant under $T$ if and only if $P_UTP_U = TP_U$.

\vspace{0.5in}

\begin{proof} ($\Longrightarrow$) (If $U$ is invariant under $T$ then $P_UTP_U = TP_U$.)
\\ For any $v \in V$ we have $P_U v \in U$ by properties of $P_U$ (6.55 (d)). By our hypothesis $T(T_Uv) \in U$, therefore
$$P_U \underbrace{(TP_Uv)}_{\in \; U} = TP_Uv$$
by property 6.55 (b).
\end{proof}
\vspace{.5in}
\begin{proof} ($\Longleftarrow$) (If $P_UTP_U = TP_U$ then $U$ is invariant under $T$.)
\\For any vector $u \in U$ we have $Tu = v + w$ for some $v \in U$ and $w \in \per U$. Then we would first have
\begin{align*}
    P_UTP_U &= P_UTu\\
    &= P_U(v+w)\\
    &= \underbrace{P_Uv}_{\in \; U} + \underbrace{P_Uw}_{\in \; \per U}\\
    &= v + 0
    \\ &= v.
\end{align*}
Secondly, we would have $TP_Uu = Tu = v + w$. Therefore under our hypothesis, $v = v + w$, which implies $w = 0$. Then $Tu = v + 0 = v \in U$, hence $U$ is invariant under $T$.
\end{proof}

\nnl Hence $U$ is invariant under $T$ if and only if $P_UTP_U = TP_U$.\newpage
\item In $\mathbf{R}^4$, let
    \[
        U = \text{span}\left((0,0,1,1) , (1,2,1,1) \right).
    \]
    Find $u \in U$ such that $\|u - (1,3,5,4) \|$ is as small as possible.

    \vspace{0.5in}


    \nl \textbf{Solution: } Let $B$ be an orthonormal basis of $U$. Let $v_1, v_2 \in \R^4$ equal $v_1 = (0,0,1,1)$ and $v_2 = (1,2,1,1).$ Using the Gram-Schmidt procedure on $v_1$ and $v_2$,
    $$e_1 = \frac{1}{\norm{v_1}} v_1 = \over{\sqrt 2} (0,0,1,1)$$
    and
    \begin{align*}
    e_2 &= \frac{v_2 - \inner{v_2, e_1}e_1}{\norm{v_2 - \inner{v_2, e_1}e_1}} \\ &= \frac{v_2 - \over2 \inner{v_2,v_1}v_1 }{\norm{v_2 - \over2 \inner{v_2,v_1}v_1}}
    \\ &= \frac{v_2 - \over2(0 + 0 + 1 + 1) v_1 }{\norm{v_2 - \over2(0 + 0 + 1 + 1) v_1}}\\
    &= \frac{v_2 - v_1}{\norm{v_2 - v_1}}\\
    &= \frac{1}{\sqrt5}(1,2,0,0).
    \end{align*}
Then $B := \set{e_1, e_2}$. Then the closest point to $u \in U$ to $w := (1,3,5,4)$ is 
\begin{align*}
    u &= \inner{w, e_1}e_1 + \inner{w,e_2}e_2 \\ &= \frac{5+4}{2}(0,0,1,1) + \frac{1 + 6}{5} (1,2,0,0)\\
    &= \pars{\frac75,\; \frac{14}{5},\; \frac92,\; \frac92 }.
\end{align*}   \newpage
\item Assume $T \in \mathcal{L}(V)$ for a complex vector space $V$.  Prove that $T$ is self-adjoint if and only if all eigenvalues for $T$ are real.
\vspace{0.3in}

\begin{proof} ($\Longrightarrow$) (If $T$ is self-adjoint, then all the eigenvalues for $T$ are real.)
    
    \nl Let $\lambda$ be an eigenvalue of $T$ and $v \in V\backslash0$ such that $Tv = \lambda v$. Then 
    \begin{align*}
        \lambda \norm v^2 &= \inner{\lambda v, v}\\
        &= \inner{Tv, v}\\
        &= \inner{v, T^*v}\\
        &= \inner{v, Tv} \qquad \text{by hypothesis}\\
        &= \inner{v, \lambda v}\\
        &= \bar{\lambda} \inner{v,v}\\
        &= \bar{\lambda} \norm v^2.
    \end{align*}
    Therefore $\lambda = \bar{\lambda}$, hence $\lambda \in \R$.
\end{proof}

\vspace{0.3in}

\begin{proof} ($\Longleftarrow$) (If all the eigenvalues for $T$ are real, then $T$ is self-adjoint.)
    By the hypothesis, $\lambda = \bar{\lambda}$ for an eigenvalue $\lambda$. Let $v \in V$ be an eigenvector so that $Tv = \lambda v$. Then,
    \begin{align*}
        \inner{Tv, v} &= \inner{\lambda v, v}\\
        &= \lambda \inner{v,v} \\
        &= \inner{v, \bar{\lambda} v}\\
        &= \inner{v, \lambda v} \qquad \text{by hypothesis}\\
        &= \inner{v, Tv}.
    \end{align*}
    Therefore $\inner{Tv, v} = \inner{v, Tv}$ is true for eigenvector $v \in V$. We are not guaranteed anything else under these assumptions.
    
    \nl \textbf{False by counterexample:} Let 
    $$\mathcal{M}(T) := \mat1101 \quad\text{ and }\quad \mathcal{M}(T^*) = \mat1011$$
    The eigenvalues of $T$ are $\lambda = 1$ with multiplicity 2. However, $\mathcal{M}(T) \neq \mathcal{M}(T^*)$. Hence a counterexample.
\end{proof}


\newpage
\item If $T \in \mathcal{L}(V)$ is self-adjoint and if $T^2v = 0$, then $Tv=0$
\vspace{.2in}

\begin{proof} If we have $T^2v = 0$, then this is equivilent to $T(Tv) = 0$. Taking the inner product of $Tv$ with itself, $\inner{Tv, Tv}$, we will show this is zero. By our hypothesis, 
    \begin{align*}
        \inner{Tv, Tv} &= \inner{v, T^*Tv} \\ &= \inner{v, T^2 v} \\ &= \inner{v,0} \qquad \text{ by  hypothesis} \\ &= 0. 
    \end{align*}
    Hence $\inner{Tv, Tv} = 0$, which implies that $Tv = 0$.
\end{proof}\vspace{1.25in}
\item Suppose $T \in \mathcal{L}(V,W)$. Prove that
    \begin{enumerate}
    \item $T$ is injective if and only if $T^*$ is surjective.
    \begin{proof} ($\Longleftrightarrow$)
        \\By definition of injective, $\null T = \set 0$.
        Using 7.7 properties, 
        \begin{align*}
            & \null T = \set0 \tag{\text{hypothesis}}\\
            \iff & (\range T^*)^{\perp} = \set0 \tag{c}\\
            \iff & \range T^* = \set0^{\perp} = W \tag{\text{perp of both}}\\
            \iff & T^* \text{ is surjective.} \tag{\text{definition of surjective}}
        \end{align*}
    \end{proof}

    \item $T$ is surjective if and only if $T^*$ is injective.
    \begin{proof} ($\Longleftrightarrow$)
        \\By definition of injective, $\null T^* = \set 0$.
        Using 7.7 properties, 
        \begin{align*}
            & \null T^* = \set0 \tag{\text{hypothesis}}\\
            \iff & (\range T)^{\perp} = \set0 \tag{a}\\
            \iff & (\range T) = \set0^{\perp} = W \tag{\text{perp of both}}\\
            \iff & T \text{ is surjective.} \tag{\text{definition of surjective}}
         \end{align*}
    \end{proof}
\end{enumerate}\newpage
\item Suppose $S,T \in \mathcal{L}(V)$ are self-adjoint.  Prove that $ST$ is self-adjoint if and only if $ST = TS$.

\vspace{.1in}

\begin{proof} ($\Longrightarrow$) (If $ST$ is self-adjoint then $ST=TS$.)
\begin{align*}
    ST &= (ST)^* \tag{\text{$ST$ self adjoint hypothesis}}\\
    &= T^* S^* \tag{\text{property e}} \\
    &= TS. \tag{$T = T^*$ \text{ and } $S=S^*$ \text{ hypothesis}}
\end{align*}
Hence if $ST$ is self adjoint then $ST = TS$.
\end{proof}

\vspace{0.3in}
\begin{proof} ($\Longleftarrow$) (If $ST = TS$ then $ST$ is self-adjoint.)
    \begin{align*}
        (ST)^* &= T^*S^* \tag{property e}\\
        &= TS \tag{$T=T^*$ and $S=S^*$ hypothesis}\\
        &= ST. \tag{$ST = TS$ hypothesis}
    \end{align*}
    Hence if $ST = TS$ then $(ST)^*=ST$. In other words, $ST$ is self-adjoint.
    \end{proof}
    \newpage
\item Suppose $P \in \mathcal{L}(V)$ is such that $P^2 = P$.  Prove that there is a subspace $U$ of $V$ such that $P = P_U$ if and only if $P$ is self-adjoint.
\vspace{0.3in}

\begin{proof} ($\Longrightarrow$) (If there is a subspace $U$ of $V$ such that $P=P_U$ then $P$ is self-adjoint)\\


\end{proof}

\vspace{0.3in}

\begin{proof} ($\Longleftarrow$) (If $P$ is self-adjoint then there exists a subspace $U$ of $V$ such that $P = P_U$)\\
Since $P$ is self-adjoint under the hypothesis, $V = \range P + \null P$. By the logic of problem \#1, $V = \range P \oplus \null P$. Let $U := \range P$ and $v \in V.$ Then for some $u \in U$ and some $w \in \per U$, $v = u +w$. We have $Pw = 0$ by null space definition. So
\begin{align*}
    P_Uv &= P_U(u + w)\\
    &= P_Uu + P_Uw\\
    &= u + 0\\
    &= u\\
    &= Pu + 0\\
    &= Pu + Pw\\
    &= Pv.
\end{align*}
\end{proof}
\end{enumerate}
\end{document} 