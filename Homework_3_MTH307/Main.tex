\documentclass[12pt]{article}
\usepackage{setspace}
\usepackage{graphicx}
\usepackage{amsmath}
\usepackage{amssymb}
\usepackage{fancyhdr}
\usepackage{setspace}
\usepackage{graphicx}
\usepackage{amsmath}
\usepackage{amssymb}
\usepackage{fancyhdr}
\usepackage{xcolor}
\usepackage{diagbox}
\usepackage{empheq}
\usepackage{makecell}
\usepackage[autostyle]{csquotes}
\usepackage{amssymb, amsthm, linguex, enumerate, amsmath}
\usepackage{amsmath}
\usepackage{graphicx}
\usepackage{enumitem}
\usepackage{xcolor}
\usepackage[colorlinks=true, allcolors=blue]{hyperref}
\usepackage{fancyhdr}
\usepackage{tcolorbox}
%\usepackage{mathabx}
\pagestyle{empty}
\usepackage{amssymb, amsthm, linguex, enumerate, amsmath}
\textwidth 6.5in
\hoffset=-.65in
\textheight=9.5in
\voffset=-1.in
\newcommand{\pf}{\mathcal{P}(\mathbf{F})}
\newcommand{\R}{\mathbb{R}}
\newcommand{\C}{\mathbb{C}}
\newcommand{\Cb}{\mathbb{C}}
\newcommand{\F}{\mathbb{F}}
\newcommand{\Fb}{\mathbf{F}}
\newcommand{\Rb}{\mathbf{R}}
\newcommand{\x}{x \in \mathbb{R}}
\newcommand{\nat}{n \in \mathbb{N}}
\newcommand{\dx}{\frac{d}{dx}}
\newcommand{\dxof}[1]{\frac{d}{dx} \left( {#1} \right) }
\renewcommand{\vector}[1]{\left\langle{#1}\right\rangle}
\newcommand{\pars}[1]{\left( {#1} \right) }
\newcommand{\brac}[1]{\left[ {#1} \right] }
\newcommand{\limit}[3]{\lim_{{#1}\to {#2}} {#3}}
\newcommand{\xo}{x_0}
\newcommand{\iid}{independent identically distributed }
\newcommand{\dble}{differentiable }
\newcommand{\xbar}{\bar{X}}
\newcommand{\ybar}{\bar{Y}}
\newcommand{\OR}{\,\lor\,}
\newcommand{\AND}{\,\land\,}
\newcommand{\rv}{random variable }
\newcommand{\distn}{distribution }
\newcommand{\cont}{continuous }
\newcommand{\then}{\,\Rightarrow\,}
\renewcommand{\emptyset}{\{\,\}}
\newcommand{\disc}{discontinuous }
\newcommand{\forx}{\qquad \text{for all } x}
\newcommand{\cbi}{closed bounded interval }
\newcommand{\seq}[1]{\{{#1}\}}
\newcommand{\limn}{\lim_{n\to\infty}}
\renewcommand{\f}[1]{f^{({#1})}}
\renewcommand{\sup}[1]{\text{sup}\left\{ {#1} \right\}}
\renewcommand{\inf}[1]{\text{inf}\left\{ {#1} \right\}}
\newcommand{\N}[2]{N \left( {#1},{#2} \right)}
\newcommand{\gammaDist}[2]{\gamma \left( {#1},{#2} \right)}
\newcommand{\Ndef}{N\left(\mu, \sigma^2\right)} %default normal
\newcommand{\thru}[1]{{#1}_1, \dots, {#1}_n}
\newcommand{\yn}{Y_1, \dots, Y_n}
\renewcommand{\max}[1]{\text{max}\left\{ {#1} \right\}}
\renewcommand{\min}[1]{\text{min}\left\{ {#1} \right\}}
\renewcommand{\over}[1]{\frac{1}{{#1}}}
\usepackage{setspace}
\usepackage{graphicx}
\usepackage{amsmath}
\usepackage{amssymb}
\newcommand*\widefbox[1]{\fbox{\hspace{2em}#1\hspace{2em}}}
%\usepackage{mathabx}
\newtcolorbox{mybox}[1][]{colback=white, sharp corners, #1}
\textwidth 6.5in
\hoffset=-.65in
\textheight=9.5in
\voffset=-1.in
\newcommand{\T}[2]{T\big({#1}\big)({#2})}
\newcommand{\func}[2]{\big({#1}\big)({#2})}
\newcommand{\dimn}[1]{\operatorname{dim}({#1})}
\begin{document}

\pagestyle{fancy}
\fancyhf{}
\fancyhead[LE,RO]{Matthew Wilder}
\fancyhead[RE,LO]{Assignment 3 - MTH 307}
\fancyfoot[LE,CO]{Page \thepage}

\noindent \textbf{Matthew Wilder}\\MATH 307 - Spring 2022 \\
Assignment \#3 \\
Due Friday, 02-04-22, 16:00 CST \\

For each problem, include the statement of the problem. Leave a blank line.  At the beginning of the next line, write \textbf{Solution} or \textbf{Proof} -- as appropriate.

\begin{enumerate}
\item  Suppose that the vectors $v_1, v_2, v_3, v_4$ are a basis for $V$.  Show that the vectors \\
    $v_1-v_2, v_1+v_2, v_3+v_4, v_4$ also form a basis for $V$.
\begin{mybox}
    \textbf{Solution:}
    Let $S \equiv \operatorname{span}\{(v_1-v_2), (v_1+v_2), (v_3+v_4), v_4\}$ We need to show that $\{v_1, v_2, v_3, v_4\} \in S$
    $$\over{2} \Big[ \underbrace{(v_1-v_2)}_{\in\,S} + \underbrace{(v_1 + v_2)}_{\in\,S}\Big] = \over{2}\brac{2v_1} = v_1$$
        $$\text{Therefore } \, v_1 \in S$$
        \vspace{0.3in}
    $$\underbrace{(v_1+v_2)}_{\in\,S} + \underbrace{(-v_1)}_{\in\, S} = v_2$$
                $$\text{Therefore } \, v_2 \in S$$\vspace{0.3in}
        $$\underbrace{1v_4}_{\in\,S} = v_4$$$$\text{Therefore }\, v_4 \in S$$\vspace{0.3in}
        $$\underbrace{(v_3 + v_4)}_{\in\,S} + \underbrace{(-v_4)}_{\in\, S} = v_3$$
        $$\text{Therefore }\, v_3 \in S$$\vspace{0.2in}\\
        Since $v_1, v_2, v_3, v_4 \in S$ and is a basis, then $S \equiv V$. Theorem 2.31 says every spanning list contains a basis, and Theorem 2.35 says that all bases must have the same length. Since the length of the basis $\{v_1, v_2, v_3, v_4\}$ is 4, and the length of spanning list $\{(v_1-v_2), (v_1+v_2), (v_3+v_4), v_4\}$ is 4, that spanning list must also form a basis for $V$.
\end{mybox}
\newpage
\item   Suppose that the vectors $v_1, v_2, v_3, v_4$ is a basis for $V$.  Let $U$ be a subspace of $V$.  Assume $v_1, v_2 \in U$ but neither $v_3$ nor $v_4$ are in $U$.  Is $v_1, v_2$ a basis for $U$?  Justify.
\begin{mybox}
    \textbf{Solution}. No. $v_1, v_2$ will not always form a basis for $U$
    \begin{proof}
        Let $V \equiv \R^4$ and $v_1, v_2, v_3, v_4$ be defined as follows,
        \begin{align}
            v_1 &= (1, 0, 0, 0) \notag\\
            v_2 &= (0, 1, 0, 0) \notag\\
            v_3 &= (0, 0, 1, 0) \notag\\
            v_4 &= (0, 0, 0, 1) \notag
        \end{align}
        
        This is the trivial basis, since for some $(w, x, y, z) \in \R^4$
        $$\begin{pmatrix} w \\x \\y\\z \end{pmatrix} = 
    a \underbrace{
        \begin{pmatrix}
            1 \\ 0 \\ 0 \\ 0 \\
        \end{pmatrix}
    }_{v_1}
    + b
    \underbrace{
        \begin{pmatrix}
            0 \\ 1 \\ 0 \\ 0 \\
        \end{pmatrix}
    }_{v_2}
    + c
    \underbrace{
        \begin{pmatrix}
            0 \\ 0 \\ 1 \\ 0 \\
        \end{pmatrix}
    }_{v_3} 
    + d
    \underbrace{
        \begin{pmatrix}
            0 \\ 0 \\ 0 \\ 1 \\
        \end{pmatrix}
    }_{v_4} 
    $$
    $$a = w \qquad b = x \qquad c = y \qquad d = z$$
    %
    $$\begin{pmatrix} w \\x \\y\\z \end{pmatrix} = 
    w \underbrace{
        \begin{pmatrix}
            1 \\ 0 \\ 0 \\ 0 \\
        \end{pmatrix}
    }_{v_1}
    + x
    \underbrace{
        \begin{pmatrix}
            0 \\ 1 \\ 0 \\ 0 \\
        \end{pmatrix}
    }_{v_2}
    + y
    \underbrace{
        \begin{pmatrix}
            0 \\ 0 \\ 1 \\ 0 \\
        \end{pmatrix}
    }_{v_3} 
    + z
    \underbrace{
        \begin{pmatrix}
            0 \\ 0 \\ 0 \\ 1 \\
        \end{pmatrix}
    }_{v_4} 
    $$
    
        Then we let $U \equiv \{(u_1, u_2, u_3, u_4) \mid u_3 = u_4\}$. Thus, $v_1 \in U$ and $v_2 \in U$ holds since $0 = 0$. But $v_3 \not \in U$ since $1 \neq 0$ and similarly $v_4 \not \in U$ since $0 \neq 1$. $v_1$ and $v_2$ do not span $U$ since, for example, no such $av_1 + bv_2 = (1,1,1,1) \in U$, which violates the definition of a basis being a \textit{list of vectors in V that is linearly independent and \underline{spans} V}. Thus $v_1, v_2$ is \textbf{not} a basis for $U$.
    \end{proof}
\end{mybox}
\newpage
\item Let $v_1 = (-1,1,2) \in \mathbf{R}^3$.  Construct two bases for $\mathbf{R}^3$:  $\{v_1, v_2,v_3\}$ and $\{v_1, v_2', v_3'\}$ so that $\{v_2, v_3, v_3'\}$ is also a basis.\vspace{0.5in}\\
    \textbf{Solution:} Let $B_1$ and $B_2$ be bases of $\R^3$ such that
    $$B_1 = \{v_1, v_2, v_3\} = 
    \left\{\begin{pmatrix}
        -1 \\ 1 \\ 2 \\ 
    \end{pmatrix}, 
    \begin{pmatrix}
        1 \\ 0 \\ 0 \\ 
    \end{pmatrix}, 
    \begin{pmatrix}
        0 \\ 1 \\ 0 \\ 
    \end{pmatrix} 
    \right\}$$
    
    $$B_2 = \{v_1, v_2', v_3'\} = 
    \left\{\begin{pmatrix}
        -1 \\ 1 \\ 2 \\ 
    \end{pmatrix}, 
    \begin{pmatrix}
        1 \\ 0 \\ 0 \\ 
    \end{pmatrix}, 
    \begin{pmatrix}
        0 \\ 0 \\ 1 \\ 
    \end{pmatrix} 
    \right\}$$
    Then, from the problem statement, let $B_3$ be a third basis for $\R^3$ such that
    $$B_3 = \{v_2, v_3, v_3'\} = 
    \left\{\begin{pmatrix}
        1 \\ 0 \\ 0 \\ 
    \end{pmatrix}, 
    \begin{pmatrix}
        0 \\ 1 \\ 0 \\ 
    \end{pmatrix}, 
    \begin{pmatrix}
        0 \\ 0 \\ 1 \\ 
    \end{pmatrix} 
    \right\}$$
    
    Then, $\operatorname{span}B_1 = \R^3$ since
    $$\begin{pmatrix}
            x \\ y \\ z \\ 
        \end{pmatrix} = a \underbrace{
        \begin{pmatrix}
            -1 \\ 1 \\ 2 \\ 
        \end{pmatrix}
    }_{v_1}
    + b
    \underbrace{
        \begin{pmatrix}
            1 \\ 0 \\ 0 \\ 
        \end{pmatrix}
    }_{v_2}
    + c
    \underbrace{
        \begin{pmatrix}
            0 \\ 1 \\ 0 \\ 
        \end{pmatrix}
    }_{v_3} 
    \in \operatorname{span}B_1
    $$
    Which implies that 
    \begin{align}x &= -a + b\notag\\ y &= a + c \notag\\ z &= 2a\notag\end{align} 
     Solving for $a,b,$ and $c$ in terms of $x, y,$ and $z$. We get
    \begin{align}a &= \frac{z}{2} \notag\\ b &= x + \frac{z}{2} \notag\\ c &= y - \frac{z}{2}\notag\end{align}
    Therefore, substituting back, we get
    $$\begin{pmatrix}
            x \\ y \\ z \\ 
        \end{pmatrix} = \frac{z}{2} \underbrace{
        \begin{pmatrix}
            -1 \\ 1 \\ 2 \\ 
        \end{pmatrix}
    }_{v_1}
    + \frac{2x+z}{2}
    \underbrace{
        \begin{pmatrix}
            1 \\ 0 \\ 0 \\ 
        \end{pmatrix}
    }_{v_2}
    + \frac{2y-z}{2}
    \underbrace{
        \begin{pmatrix}
            0 \\ 1 \\ 0 \\ 
        \end{pmatrix}
    }_{v_3} 
    \in \operatorname{span}B_1
    $$
    Since $\operatorname{span}B_1 = \R^3$, the length of $B_1$ is 3, and $\operatorname{dim} \R^3 = 3$, then by Theorem 2.42, $B_1$ is a basis of $\R^3$ since $3=3$.
    \vspace{0.5in}\\
    
    
    
    Next, $\operatorname{span}B_2 = \R^3$ since
    $$\begin{pmatrix}
            x \\ y \\ z \\ 
        \end{pmatrix} = a \underbrace{
        \begin{pmatrix}
            -1 \\ 1 \\ 2 \\ 
        \end{pmatrix}
    }_{v_1}
    + b
    \underbrace{
        \begin{pmatrix}
            1 \\ 0 \\ 0 \\ 
        \end{pmatrix}
    }_{v_2'}
    + c
    \underbrace{
        \begin{pmatrix}
            0 \\ 0 \\ 1 \\ 
        \end{pmatrix}
    }_{v_3'} 
    \in \operatorname{span}B_2
    $$
    Which implies that \begin{align}x &= -a + b\notag\\ y &= a \notag\\ z &= 2a + c\notag\end{align} Solving for $a,b,$ and $c$ in terms of $x, y,$ and $z$. We get
    \begin{align}a &= y\notag\\ b &= x+y \notag\\ c &= z - 2y\notag\end{align}
    Therefore,
    $$\begin{pmatrix}
            x \\ y \\ z \\ 
        \end{pmatrix} = y \underbrace{
        \begin{pmatrix}
            -1 \\ 1 \\ 2 \\ 
        \end{pmatrix}
    }_{v_1}
    + (x+y)
    \underbrace{
        \begin{pmatrix}
            1 \\ 0 \\ 0 \\ 
        \end{pmatrix}
    }_{v_2'}
    + (z-2y)
    \underbrace{
        \begin{pmatrix}
            0 \\ 0 \\ 1 \\ 
        \end{pmatrix}
    }_{v_3'} \in \operatorname{span}B_2$$
    Because $\operatorname{span}B_2 = \R^3$, and the length of $B_2$ equals $\operatorname{dim}\R^3$, by Theorem 2.42, $B_2$ is a basis of $\R^3$
    \vspace{1in}\\\begin{center}
        (continued to next page)
    \end{center}
    \newpage
    Lastly, $\operatorname{span}B_3 = \R^3$ since
    $$\begin{pmatrix}
            x \\ y \\ z \\ 
        \end{pmatrix} = a \underbrace{
        \begin{pmatrix}
            1 \\ 0 \\ 0 \\ 
        \end{pmatrix}
    }_{v_2}
    + b
    \underbrace{
        \begin{pmatrix}
            0 \\ 1 \\ 0 \\ 
        \end{pmatrix}
    }_{v_3}
    + c
    \underbrace{
        \begin{pmatrix}
            0 \\ 0 \\ 1 \\ 
        \end{pmatrix}
    }_{v_3'} 
    \in \operatorname{span}B_3
    $$
    Which implies that \begin{align}x &= a\notag\\ y &= b \notag\\ z &= c\notag\end{align}
    Substituting in for $a,b$ and $c$ we get,
    $$\begin{pmatrix}
            x \\ y \\ z \\ 
        \end{pmatrix} = x \underbrace{
        \begin{pmatrix}
            1 \\ 0 \\ 0 \\ 
        \end{pmatrix}
    }_{v_2}
    + y
    \underbrace{
        \begin{pmatrix}
            0 \\ 1 \\ 0 \\ 
        \end{pmatrix}
    }_{v_3}
    + z
    \underbrace{
        \begin{pmatrix}
            0 \\ 0 \\ 1 \\ 
        \end{pmatrix}
    }_{v_3'} \in \operatorname{span}B_3$$
    Because $\operatorname{span}B_3 = \R^3$, and the length of $B_3$ equals $\operatorname{dim}\R^3$, by Theorem 2.42, $B_3$ is a basis of $\R^3$
    \vspace{0.5in}

\newpage
\item
    \begin{enumerate}
        \item Under what conditions on the scalar $\xi$ do the vectors $(1,1,1)$ and $(1, \xi , \xi^2 )$ form a basis for $\mathbf{R}^3$?
        \begin{mybox}
            \textbf{Solution:} Under no condition does $\xi$ form a basis for the two vectors. If we assumed that we choose a $\xi$ such that $\{(1,1,1)\,, (1,\xi, \xi^2)\}$ is linearly independent (a requirement for a basis), then by Theorem 2.39 the length of the list must equal the dimension of $V$. In this instance, $$\operatorname{dim}\R^3 = 3 \neq 2 = \big|\{(1,1,1)\,, (1,\xi, \xi^2)\}\big|$$ Hence, no $\xi$ will form a basis of $\R^3$ since the length would be less than the dimension of $\R^3$ anyways.
        \end{mybox}
        \vspace{2in}
        \item Under what conditions on the scalar $\xi$ do the vectors\\ $(0,1,\xi)$, $(\xi , 0,1 )$, and $(\xi,1,1+\xi)$  form a basis for $\mathbf{R}^3$?
        \begin{mybox}
            \textbf{Solution:} Let $v_1 = (0,1,\xi)$, $v_2 = (\xi , 0,1 )$, and $v_3 = (\xi,1,1+\xi)$. Under no condition does $\xi$ form a basis because for every $\xi \in \R$, $v_3$ is a linear combination of $v_1 + v_2$. 
            $$\underbrace{(0,1,\xi)}_{v_1} + \underbrace{(\xi,0,1)}_{v_2} = \underbrace{(\xi, 1, 1 + \xi)}_{v_3}$$
            Hence, the set of vectors is linearly \textbf{dependent}, and by definition of a basis, cannot form a basis.
        \end{mybox}
    \end{enumerate}
\newpage
\item Let $V = \mathcal{P}(\mathbf{R})$ be the vector space of all polynomials with real coefficients.  If $p$ is any polynomial, let $Tp$ be the polynomial defined by $(Tp)(x) = p(x+1) - p(x)$.  Show that $T$ is a linear transformation.
\begin{mybox}
    \textbf{Solution: } \\First we will show additivity,
    \begin{align}
        \T{f+g}{x} &= \func{f+g}{x+1} - \func{f+g}{x} \notag\\
        &= f(x+1) + g(x+1) - \Big(f(x) + g(x) \Big)\notag\\
        &= f(x+1) + g(x+1) - f(x) - g(x) \notag\\
        &= f(x+1) - f(x) + g(x+1) - g(x) \notag
    \end{align}
    \begin{align}
        {Tf(x)} + {Tg(x)} &= {f(x+1) - f(x)} + {g(x+1) - g(x)} \notag
    \end{align}
    Since $\T{f+g}{x} = {Tf(x)} + {Tg(x)}$, addition is preserved.\vspace{0.5in}\\
    Now we will show homogeneity,
    \begin{align}
        \T{\lambda f}{x} &= \func{\lambda f}{x+1} - \func{\lambda f}{x} \notag\\
        &= \lambda f(x+1) - \lambda f(x) \notag
    \end{align}
        \begin{align}
        \lambda \pars{ Tf(x)} &= \lambda \pars{ f(x+1) - f(x) } \notag \\
        &= \lambda f(x+1) - \lambda f(x) \notag
    \end{align}
    Since $\T{\lambda f}{x} = \lambda Tf(x)$, homogeneity is preserved.
    \\\\
    These two conditions together satisfy the definition of a linear map. Therefore $T$ is a linear transformation.
\end{mybox}
\newpage
\item Let $V = \mathcal{P}_4(\mathbf{R})$, the vector space of polynomials of degree at most four. \\ Let $U = \{p \in V : p(1)=p(3)\}$
    \begin{enumerate}
        \item Find a basis of $U$.
        \begin{mybox}
            \textbf{Solution: } Since $p(1)=p(3)$, we can start by finding $p(x) : p(1)=p(3)=0$ and then adding the constant function $p(x) = c$ to remove the zero constraint and return to $U$. Let $S \subset U$ such that $S \equiv \{p(x) \in V : p(1) = p(3) = 0\}$
            $$(x-1)(x-3) \in  S \in U$$
            $$x(x-1)(x-3) \in  S \in U$$
            $$x^2(x-1)(x-3) \in S\in U$$
            $$1 \in U$$
            We can show that the list $\{1,\, (x-1)(x-3),\, x(x-1)(x-3),\, x^2(x-1)(x-3)\}$ is linearly independent since
            $$a + b(x-1)(x-3) + cx(x-1)(x-3) + dx^2(x-1)(x-3) = 0$$
            Since there is no degree 4 term on the RHS, $d=0$, similarly since there are no degree 3 terms on the RHS, $c = 0$, no degree 2 terms on the RHS implies $b = 0$, and thus we are left with $a = 0$. So the only wait to obtain the zero polynomial is trivially, thus the list is linearly independent. This means that
            $\operatorname{dim}U \geq 4$. But since $U \subseteq P$, $\operatorname{dim} U \leq \operatorname{dim}V = 5$ (by 2.38). Suppose that $\operatorname{dim}U = 5$, then $x \in U$, which is a contradiction. Therefore $\operatorname{dim}U \leq 4$. But by the linear independence of the former list of elements, $\operatorname{dim}U \geq 4$. Hence $4 \leq \operatorname{dim}U \leq 4$, and thus $\operatorname{dim}U = 4$. Then by Theorem 2.39 $\{1,\, (x-1)(x-3),\, x(x-1)(x-3),\, x^2(x-1)(x-3)\}$ is a basis for $U$ since it is linearly independent and its length equals $\operatorname{dim}U$.
        \end{mybox}
        \vspace{1.25in}
        \item Extend the basis in (a) to a basis of $V$.
        \begin{mybox}
            \textbf{Solution: } Using the explanation from part (a) that said $x \not \in U$. Thus we can add $x$ to the list and maintain linear independence.\\
            Let $L \equiv \{1,\, x,\, (x-1)(x-3),\, x(x-1)(x-3),\, x^2(x-1)(x-3)\}$
            Then $|L| = 5$, and $\operatorname{dim}V = 5$. Since $L$ is linearly independent, and $|L| = \operatorname{dim}V$, then $L$ is a basis by Theorem 2.39.
        \end{mybox}
        \newpage
        \item Find a subspace $W$ of $V$ so that $V=U \oplus W$.
        \begin{mybox}
            Let $W = \operatorname{span}(x)$, then by Theorem 2.43,
            $$\dimn{U+W} = \operatorname{dim}U + \operatorname{dim}W - \dimn{U\cap W}$$
            By part (a), $\operatorname{dim}U = 4$. We can easily see that $\operatorname{dim}W = 1$. $U \cap W = \{0\}$ and $\dimn{0} = 0$. Therefore, substituting back in we see that
            $$\dimn{U+W} = 4 + 1 - 0 = 5$$
            But because $U \cap W = \{0\}$, $U \oplus W = U + W$ (by 1.45). \\Hence, $\dimn{U \oplus W} = \dimn{U + W} = 5$. And we know that $\operatorname{dim}V = 5$. Therefore, $$\dimn{U \oplus W} = \dim V$$
            And since $U \oplus W \subseteq V$,$$U \oplus W = V$$
        \end{mybox}
    \end{enumerate}
\newpage
\item Suppose $a,b,c \in \mathbf{R}$.  Define $T: \mathcal{P}(\mathbf{R}) \to \mathbf{R}^3$ by
    \[
    Tp = \left( 2p(5)-5p'(1) + a p(1)p(3)\; , \;\int_1^4 x^2p(x)\;dx + b e^{p(0)}\; , \; p(2)+c\right).
    \]
    Show that $T$ is linear if and only if $a=b=c=0$.
    \begin{proof} ($\Longrightarrow$) If $T$ is linear, then $a = b = c = 0$.
    \begin{align}
        T(p+q) &= \begin{pmatrix}
            2(p+q)(5)-5(p+q)'(1) + a(p+q)(1)\cdot (p+q)(3) \\
            \int_1^4 x^2 (p+q)(x)\,dx + be^{(p+q)(0)} \\
            (p+q)(2) + c \\ 
        \end{pmatrix}
        \notag\\
        &= \begin{pmatrix}
            2p(5) + 2q(5) -5p'(1) -5q'(1) + a\brac{\big(p(1)+q(1)\big)\big(p(3)+q(3)\big)} \\
            \int_1^4 x^2 p(x)+x^2q(x)\,dx + be^{p(0)+q(0)} \\
            p(2)+q(2) + c \\ 
        \end{pmatrix} \notag\\
        &= \begin{pmatrix}
            2p(5) + 2q(5) -5p'(1) -5q'(1) + a\brac{\big(p(1)+q(1)\big)\big(p(3)+q(3)\big)} \\
            \int_1^4 x^2 p(x)\,dx + \int_1^4 x^2q(x)\,dx + be^{p(0)+q(0)} \\
            p(2)+q(2) + c \\
        \end{pmatrix} \notag
    \end{align}
    \begin{align}
        Tp + Tq &= \begin{pmatrix}
            2p(5)-5p'(1) + ap(1)p(3) \\
            \int_1^4 x^2 p(x)\,dx + be^{p(0)} \\
            p(2) + c \\ 
        \end{pmatrix} + 
        \begin{pmatrix}
            2q(5)-5q'(1) + aq(1)q(3) \\
            \int_1^4 x^2 q(x)\,dx + be^{q(0)} \\
            q(2) + c \\ 
        \end{pmatrix}
        \notag\\
        &= \begin{pmatrix}
            2p(5) + 2q(5) -5p'(1) -5q'(1) + ap(1)p(3) + aq(1)q(3) \\
            \int_1^4 x^2 p(x)\,dx + \int_1^4 x^2 q(x)\,dx + be^{p(0)} + be^{q(0)}\\
            p(2) + c + q(2) + c \\ 
        \end{pmatrix}\notag\\
        &= \begin{pmatrix}
            2p(5) + 2q(5) -5p'(1) -5q'(1) + ap(1)p(3) + aq(1)q(3) \\
            \int_1^4 x^2 p(x)\,dx + \int_1^4 x^2 q(x)\,dx + be^{p(0)} + be^{q(0)}\\
            p(2) + c + q(2) + c \\ 
        \end{pmatrix}\notag\\
        &= \begin{pmatrix}
            2p(5) + 2q(5) -5p'(1) -5q'(1) + ap(1)p(3) + aq(1)q(3) \\
            \int_1^4 x^2 p(x)\,dx + \int_1^4 x^2 q(x)\,dx + 2be^{p(0)}\\
            p(2) + q(2) + 2c \\ 
        \end{pmatrix}\notag
    \end{align}
    In order for additivity to hold, each component must be equal. For the 1st component,
    $$2p(5) + 2q(5) -5p'(1) -5q'(1) + a\brac{\big(p(1)+q(1)\big)\big(p(3)+q(3)\big)}$$$$= 2p(5) + 2q(5) -5p'(1) -5q'(1) + ap(1)p(3) + aq(1)q(3)$$
    Simplifying, we get
    $$a\brac{\big(p(1)+q(1)\big)\big(p(3)+q(3)\big)} = ap(1)p(3) + aq(1)q(3)$$
    Distributing the left hand side we get
    $$ap(1)p(3) + aq(1)q(3) + ap(1)q(3) + aq(1)p(3) = ap(1)p(3) + aq(1)q(3)$$
    Simlifying further we reach,
    $$ap(1)q(3) + aq(1)p(3) = 0$$
    Thus, for the first component to hold linearity under addition, $a$ must be 0.
    \vspace{0.3in}\\
    For the second component, linearity implies that
    $$\int_1^4 x^2 p(x)\,dx + \int_1^4 x^2q(x)\,dx + be^{p(0)+q(0)} = \int_1^4 x^2 p(x)\,dx + \int_1^4 x^2 q(x)\,dx + 2be^{p(0)}$$
    Simplifying out the integrals leaves us with
    $$be^{p(0)+q(0)} = 2be^{p(0)+q(0)}$$
    Which implies $b = 0$ for linearity.
    \vspace{0.3in}\\
    For the third component, linearity implies that
    $$p(2) + q(2) + c = p(2) + q(2) + 2c$$
    Simplifying for
    $$c = 2c$$
    Implies that $c = 0$ to preserve linearity.\\\\
    Therefore, in order for addition to be linear, $a = b = c = 0$. It suffices to show just addition since it restricts all 3 variables to one value.
    \end{proof}
    \begin{proof} ($\Longleftarrow$) If $a=b=c=0$ then $T$ is linear.
    That means that
    $$Tp = \left( 2p(5)-5p'(1)\,,\,\int_1^4 x^2p(x)\,dx\,,\, p(2)\right).$$
    It has been shown in the forward direction that this holds for additivity when \\$a = b = c = 0$. Thus, we only need to show the remaining property of homogeneity.
    \begin{align}
        T(\lambda p) &= \left( 2\lambda p(5)-5\lambda p'(1)\,,\,\int_1^4 x^2\lambda p(x)\,dx\,,\, \lambda p(2)\right) \notag\\
        &= \left( 2\lambda p(5)-5\lambda p'(1)\,,\,\lambda\int_1^4 x^2 p(x)\,dx\,,\, \lambda p(2)\right) \notag
    \end{align}
    \begin{align}
        \lambda \pars{ Tp } &= \lambda \pars{ 2p(5) - 5p'(1)\,,\,\int_1^4 x^2p(x)\,dx\,,\,p(2) } \notag \\
        &=  \pars{ 2\lambda p(5) - 5\lambda p'(1)\,,\,\lambda \int_1^4 x^2p(x)\,dx\,,\,\lambda p(2) } \notag
    \end{align}
    Hence, $T(\lambda p) = \lambda(Tp)$ and therefore $T$ is linear by definition.
    \end{proof}
    It has been shown that $T$ is linear if and only if $a = b = c = 0$
    \newpage
\item \begin{enumerate}
        \item Find an example of a function $\varphi : \mathbf{R}^2 \to \mathbf{R}$ that is homogeneous but not additive (and hence not linear).
        \begin{mybox}
        \textbf{Solution: } Let $\phi : \R^2 \to R$ by defined as,
        $$\phi(x,y) = \sqrt{x^3 + y^3}$$
        Then it is homogeneous since
        \begin{align}
            \phi(\lambda(x,y) ) &= \phi(\lambda x, \lambda y) \notag\\
            &= \sqrt[3]{(\lambda x)^3 + (\lambda y)^3} \notag\\
            &= \sqrt[3]{\lambda^3 x^3 + \lambda^3 y^3} \notag\\
            &= \sqrt[3]{\lambda^3(x^3+y^3)} \notag\\
            &= \lambda \sqrt[3]{x^3+y^3} \notag\\
            &= \lambda \phi(x,y)\notag
        \end{align}
        However, it is not additive because if we take $v_1 = (x_1, y_1) = (1,0)$ and $v_2 = (x_2, y_2) = (0,1)$ then
        \begin{align}
            \phi(v_1 + v_2) &= \phi((1,0) + (0,1)) \notag\\
            &= \phi(1,1) \notag\\
            &= \sqrt[3]{1^3+1^3} \notag\\
            &= \sqrt[3]{2}\notag
        \end{align}
        And
        \begin{align}
           \phi(v_1) + \phi(v_2) &= \phi(1,0) + \phi(0,1) \notag\\
           &= \sqrt[3]{1^3 + 0^3} + \sqrt[3]{0^3 + 1^3} \notag\\
           &= \sqrt[3]{1} + \sqrt[3]{1} \notag\\
           &= 2\notag
        \end{align}
        Therefore $\phi(v_1+v_2) \neq \phi(v_1) + \phi(v_2)$. Hence $\phi$ is not additive.
        \end{mybox}
        \newpage
        \item Find an example of a function $\varphi : \mathbf{C}^2 \to \mathbf{C}$ that is additive but not homogeneous (and hence not linear).
        \begin{mybox}
        \textbf{Solution: }
        $$\phi : \C^2 \to C$$
        $$(a+bi\,,\;c+di) \to ai+b$$
        This is additive because if we 
        \\Let $x = (a_1+b_1i\,,\; c_1+d_1i)$ and
        \\Let $y = (a_2+b_2i\,,\; c_2+d_2i)$
        \begin{align}
            \phi(x+y) &= \phi((a_1+a_2)+(b_1+b_2)i\,,\; (c_1+c_2)+(d_1+d_2)i)\notag\\
            &= (b_1+b_2) + (a_1+a_2)i \notag\\
            &= b_1 + a_1i + b_2 + a_2i \notag\\
            &= \phi(x) + \phi(y)\notag
        \end{align}
        But it is not homogeneous because if we let $\lambda = i \in \C$, $a = 1$, $b = c = d = 0$, then
        $$\phi(\lambda (1, 0)) = \phi(i(1,\,0)) = \phi(i,\, 0) = 1$$
        But 
        $$\lambda(\phi(1,0)) = i(\phi(1,0)) = i(i) = -1$$
        Therefore $\lambda(\phi(v)) \neq \phi(\lambda v)$, hence $\phi$ is not homogeneous.
        \end{mybox}
    \end{enumerate}

\end{enumerate}
\end{document} 