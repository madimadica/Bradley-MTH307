\documentclass[12pt]{article}
\usepackage[english]{babel}
\usepackage{array}
\usepackage{setspace}
\usepackage{graphicx}
\usepackage{sistyle} %\num{100000} for commas
\SIthousandsep{,}
\usepackage{fancyhdr}
\usepackage{listings} % For code listings, may break stuff
\usepackage{xcolor, diagbox, empheq, makecell, tcolorbox}
\usepackage[autostyle]{csquotes}
\usepackage{amssymb, amsthm, linguex, enumitem, amsmath}
\usepackage{tcolorbox} %dont know what this does
\usepackage[colorlinks=true, allcolors=blue]{hyperref}
\usepackage{lipsum}

\makeatletter   %% <- make @ usable in macro names
\newcommand*\notab[1]{%
  \begingroup   %% <- limit scope of the following changes
    \par        %% <- start a new paragraph
    \@totalleftmargin=0pt \linewidth=\columnwidth
    %% ^^ let other commands know that the margins have been reset
    \parshape 0
    %% ^^ reset the margins
    #1\par      %% <- insert #1 and end this paragraph
  \endgroup
}
\makeatother    %% <- revert @

\pagestyle{empty}

\textwidth 6.5in
\hoffset=-.65in
\textheight=9.5in
\voffset=-1.in

%Sets
\newcommand{\R}{\mathbb{R}}
\newcommand{\C}{\mathbb{C}}
\newcommand{\N}{\mathbb{N}}
\newcommand{\F}{\mathbb{F}}
\newcommand{\Z}{\mathbb{Z}}
\newcommand{\Cb}{\mathbf{C}}
\newcommand{\Fb}{\mathbf{F}}
\newcommand{\Rb}{\mathbf{R}}

%Misc

\newcommand{\pars}[1]{\left( {#1} \right) } %auto size parenthesis
\newcommand{\brac}[1]{\left[ {#1} \right] } %auto size brackets around arg
\newcommand{\set}[1]{\left\{{#1}\right\}} %auto size curly braces around arg
\newcommand{\vbrac}[1]{\left\langle{#1}\right\rangle} %vector angle brackets
\newcommand{\inner}[1]{\left\langle{#1}\right\rangle} %vector angle brackets
\newcommand{\conj}[1]{\overline{{#1}}} %conjugate bar
\newcommand{\vconj}[1]{\overline{\vbrac{{#1}}}}
\newcommand{\ceil}[1]{\left\lceil {#1} \right\rceil} %auto size ceiling around arg
\newcommand{\floor}[1]{\left\lfloor {#1} \right\rfloor} %auto size floor around arg

\newcommand{\limn}{\lim_{n\to\infty}} %limit as n approaches infinity
\newcommand{\thru}[1]{{#1}_1, \dots, {#1}_n}
\newcommand{\sumthru}[1]{{#1}_1 + \cdots + {#1}_n}
\renewcommand{\over}[1]{\frac{1}{{#1}}}
\newcommand{\pfrac}[2]{\left( \frac{{#1}}{{#2}} \right) } %auto size parenthesis over fraction
\newcommand{\pover}[1]{\left( \frac{1}{{#1}} \right) } %auto size parenthesis over fraction

%Boolean Algebra
\newcommand{\OR}{\,\lor\,}
\newcommand{\AND}{\,\land\,}

%Probability and Statistics
\newcommand{\xbar}{\bar{X}}
\newcommand{\ybar}{\bar{Y}}
\newcommand{\yn}{Y_1, \dots, Y_n}
\newcommand{\yx}{X_1, \dots, X_n}
\newcommand{\normDist}{N\left(\mu, \sigma^2\right)} %default normal distribution
\newcommand{\gammaDist}[2]{\operatorname{Gamma} \left( {#1},{#2} \right)}
\newcommand{\norm}[1]{\left\| {#1} \right\|}
\newcommand{\abs}[1]{\left| {#1} \right|}
\newcommand{\prob}[1]{P \left( {#1} \right) }
\newcommand{\E}[1]{E \left( {#1} \right) }
\newcommand{\Eb}[1]{E[ \,{#1}\, ]} %E bracket
\newcommand*{\V}[1]{V \left( {#1} \right) }
\newcommand{\Vb}[1]{V [ \,{#1}\, ] }
\newcommand{\that}{\hat{\theta}} %theta hat
\newcommand{\phat}{\hat{p}}
\newcommand{\psihat}{\hat{\psi}}
\newcommand{\Psihat}{\hat{\Psi}}


\newcommand{\dimrange}[1]{\operatorname{dim}\operatorname{range}{#1}} % dimrange
\newcommand{\dimnull}[1]{\operatorname{dim}\operatorname{null}{#1}} % dimnull

\newcommand{\vdub}[2]{\begin{pmatrix}{#1}\\{#2}\end{pmatrix}}

%Linear Algebra
\newcommand{\poly}[1]{\mathcal{P}_{#1}(\mathbf{R})} %polynomial up to degree (arg)
\newcommand{\pf}{\mathcal{P}(\mathbf{F})} %set of all polynomials
\renewcommand{\L}[1]{\mathcal{L}\left({#1}\right)}%Set of linear maps
\newcommand{\vdouble}[2]{\begin{pmatrix}{#1}\\{#2}\end{pmatrix}} % 2 high vertical
\newcommand{\vtriple}[3]{\begin{pmatrix}{#1}\\{#2}\\{#3}\end{pmatrix}} %vertical vector parenthesis, 3 args
\newcommand{\vquad}[4]{\begin{pmatrix}{#1}\\{#2}\\{#3}\\{#4}\end{pmatrix}} %vertical vector parenthesis, 4 args
\newcommand{\vquin}[5]{\begin{pmatrix}{#1}\\{#2}\\{#3}\\{#4}\\{#5}\end{pmatrix}} %vertical vector parenthesis, 5 args
\newcommand{\vvector}[3]{\begin{bmatrix}{#1}\\{#2}\\{#3}\end{bmatrix}} %vertical vector braces, 3 args
\newcommand{\dimn}[1]{\operatorname{dim}\,{#1}}
\newcommand{\rank}[1]{\operatorname{dim}\operatorname{range}{#1}}
\newcommand{\nullity}[1]{\operatorname{dim}\operatorname{null}{#1}}
\newcommand{\range}[1]{\operatorname{range}{#1}}
\newcommand{\NULL}[1]{\operatorname{null}{#1}}
\renewcommand{\null}{\operatorname{null}}
\newcommand{\mat}[4]{\begin{bmatrix}{#1} & {#2}\\{#3}&{#4}\end{bmatrix}} % 2x2 matrix
\newcommand{\Mat}[9]{\begin{bmatrix}{#1} & {#2} & {#3}\\{#4}&{#5}&{#6}\\{#7}&{#8}&{#9}\end{bmatrix}} % 3x3 matrix
\newcommand{\detx}[4]{\begin{vmatrix}{#1} & {#2}\\{#3}&{#4}\end{vmatrix}} % 2x2 determinant
\newcommand{\nf}{\infty}
%colors
\definecolor{ggreen}{RGB}{0, 127, 0}
\definecolor{dgray}{RGB}{63,63,63}
\definecolor{neonorange}{RGB}{255,47,0}
\definecolor{mygray}{rgb}{0.5,0.5,0.5}
\newcommand{\red}[1]{\color{red}{#1}\color{black}}
\newcommand{\grn}[1]{\color{ggreen}{#1}\color{black}}
\newcommand{\blu}[1]{\color{blue}{#1}\color{black}}
\newcommand{\redx}[1]{\color{red}\not{#1}\color{black}}

\newcommand{\say}[1]{\textquotedblleft{#1}\textquotedblright} %quote the "argument"
\newcommand*\widefbox[1]{\fbox{\hspace{2em}#1\hspace{2em}}}
\newtcolorbox{mybox}[1][]{colback=white, sharp corners, #1}

%Line break spacings
\newcommand{\nl}{\vspace{0.1in}\noindent}
\newcommand{\nnl}{\vspace{0.2in}\noindent}
\newcommand{\nnnl}{\vspace{0.3in}\noindent}

% Code snippets
\newcommand*{\code}{\fontfamily{qcr}\selectfont}
\lstset{
    backgroundcolor=\color{white},
    basicstyle=\footnotesize,
    breakatwhitespace=false,         % sets if automatic breaks should only happen at whitespace
    breaklines=true,                 % sets automatic line breaking
    captionpos=b,                    % sets the caption-position to bottom
    commentstyle=\color{dgray},    % comment style
    deletekeywords={...},            % if you want to delete keywords from the given language
    escapeinside={(*@}{@*)},          % if you want to add LaTeX within your code
    extendedchars=true,              % lets you use non-ASCII characters; for 8-bits encodings only, does not work with UTF-8
    firstnumber=1,                % start line enumeration with line 1
    frame=single,	                   % adds a frame around the code
    keepspaces=true,                 % keeps spaces in text, useful for keeping indentation of code (possibly needs columns=flexible)
    keywordstyle=\color{neonorange},       % keyword style
    language=C++,                 % the language of the code
    morekeywords={*,...},            % if you want to add more keywords to the set
    numbers=left,                    % where to put the line-numbers; possible values are (none, left, right)
    numbersep=5pt,                   % how far the line-numbers are from the code
    numberstyle=\tiny\color{mygray}, % the style that is used for the line-numbers
    rulecolor=\color{black},         % if not set, the frame-color may be changed on line-breaks within not-black text (e.g. comments (green here))
    showspaces=false,                % show spaces everywhere adding particular underscores; it overrides 'showstringspaces'
    showstringspaces=false,          % underline spaces within strings only
    showtabs=false,                  % show tabs within strings adding particular underscores
    stringstyle=\color{purple},     % string literal style
    tabsize=4,	                   % sets default tabsize to 4 spaces
}

\lstdefinestyle{cpp}{language=C++,
    morekeywords={cout, cin, Comparable, T},numbers=none
}
%Examples:
%{\code while}
%
%{\code \begin{lstlisting}[language=C++]
%sum1 = 0;
%for (i = 1; i <= n; i *= 2)
%    for (j = 1; j <= n; j++)
%        sum1++;
%\end{lstlisting}}







\begin{document}

\pagestyle{fancy}
\fancyhf{}
\fancyhead[RO]{Matthew Wilder} %header top right
\fancyhead[LO]{MTH 307 - Homework \#7} %header top left
\fancyfoot[CO]{Page \thepage} %page center bottom

\noindent MTH 307 - Spring 2022
\\Assignment \#7
\\Due: Friday, March 4, 2022 (4pm)

\nl For each problem, include the statement of the problem. Leave a blank line.  At the beginning of the next line, write \textbf{Solution} or \textbf{Proof} -- as appropriate.

\begin{enumerate}
    \item Suppose $T \in \mathcal{L}(V)$ and $T^2 = I$ and $-1$ is not an eigenvalue of $T$.  Prove that $T = I$.
\begin{mybox}
    \begin{proof}
        Since we are given $T^2 = I$, we can subtract $I$ from both sides to obtain $T^2 - I = 0$. Further, we can split this into $$(T+I)((T-I)v) = 0 \quad \text{and} \quad (T-I)((T+I)v)=0.$$ If we choose any $v \in V$ where $v \neq \vec{0}$ and $w \in V$ such that $w = (T-I)v \neq 0$, then $(T+I)w = 0$ by the first equation. Simplifying, this would imply that $Tw = -Iw$ and $-1$ to be an eigenvalue. This contradicts our assumption that $-1$ is \textit{not} an eigenvalue. Hence, $w$ \textit{must} equal 0 to remove this eigenvalue. Therefore $w = 0 = (T-I)v$ then $(T-I)v = 0$. Simplifying this we obtain $Tv = Iv$ for all $v \neq 0$. If $v = 0$ then $T\vec{0} = I\vec{0}$ is certainly true. Hence $Tv = Iv$ for all $v \in V$. By function equivalency we know $T=I$ when $-1$ is \textit{not} an eigenvalue.
    \end{proof}
\end{mybox}
    \vspace{.5in}
    \item Suppose $P \in \mathcal{L}(V)$ and $P^2=P$.  Prove that $V = \text{null }P \oplus \text{range } P$.
\begin{mybox}
    \begin{proof}
        Using the given notion that $P^2 = P$ we can simplify as follows:
            $$
                  P^2 = P
                \iff P^2 - P = 0
                \iff  P(P-I) = 0.$$
            Hence for all $v \in V$, $P(P-I)v = 0$. This implies that \\$\NULL P \supset \{(P-I)v : v \in V\}$. Thus a typical element $v \in V$ can be rewritten as
            $$v = \underbrace{Pv}_{\text{range}P} - \underbrace{(P-I)v}_{\text{null}P}.$$
            Hence $V = \range P + \NULL P$. We need the intersection $\NULL P \cap \range P$, and to do this we need $\NULL P$'s equality. Let $v \in \NULL P$. Then $Pv = 0$ implies that (by the above equation) $v = Pv - (P-I)v = (P-I)(-v)$. Hence $$\NULL P = \{(P-I)v : v \in V\}.$$
            As such, the intersection of $\NULL{P}$ and $\range{P}$ is when $Pv = (P-I)v = Pv - Iv$. Therefore $0 = -Iv$, implying that $\NULL P \cap \range P = \{\vec{0}\}$. We can apply the definition of a Direct Sum to get that $V = \NULL P \oplus \range P$.
    \end{proof}
\end{mybox}

    \item Suppose $T \in \mathcal{L}(V)$ and $v$ is an eigenvector of $T$ with eigenvalue $\lambda$.  Suppose $p \in \mathcal{P}(\mathbf{R})$.  Prove that $p(T)v = p(\lambda)v$.
\begin{mybox}
    \begin{proof}
        Let $p$ be of the form $p(z) = a_0 + a_1 z + \cdots + a_m z^m$. Then we have
        \begin{align*}
            p(T)v &= \pars{a_0 T^0 + a_1 T^1 + \cdots + a_m T^m}v & p(T)\text{ definition} \\
            &= a_0 T^0v + a_1 T^1v + \cdots + a_m T^mv & \text{linearity of } T\\
            &= a_0 \lambda^0 v + a_1 \lambda^1v + \cdots + a_m \lambda^mv & Tv = \lambda v\\
            &= ( a_0 + a_1 \lambda + \cdots + a_m \lambda^m )v & \text{linearity}\\
            &= p(\lambda)v & \text{definition of } p
        \end{align*}
    \end{proof}
\end{mybox}
    \vspace{0.6in}
    \item Suppose $W$ is a complex vector space and $T \in \mathcal{L}(W)$ has no eigenvalues.  Prove that every subspace of $W$ invariant under $T$ is either $\{0\}$ or infinite-dimensional.
\begin{mybox}
    \begin{proof}
        Suppose $W$ is non-zero, finite dimentional, then by Theorem 5.21 it has an eigenvalue. Hence $T|_U$ has some eigenvalue $\lambda$ and consequently $T$ has that eigenvalue $\lambda$. This is a contradiction to the assumption that $T$ has no eigenvalues. Therefore $W$ must be $\{0\}$ or infinite dimentional to satisfy $T$'s assumption.
    \end{proof}
\end{mybox}
    \vspace{0.6in}
    \item Suppose $V$ is a finite-dimensional complex vector space and $T \in \mathcal{L}(V)$.  Define a function $f: \mathbf{C} \to \mathbf{R}$ by
    \[
        f(\lambda) = \dim \text{range}(T- \lambda I).
    \]
    Prove that $f$ is not a continuous function.
    \begin{mybox}
        \begin{proof}
            Let $\dim V = n$. \\Because we are in a finite dimensional \textit{complex} vector space we are guaranteed the existance of some eigenvalue $\lambda_0$ for $T$ (when $T$ is not the zero transformation). Since $T-\lambda_0 I$ is not surjective, then $f(\lambda_0) = \rank T \leq n-1$ by the rank-nullity theorem (FTLM). If $\lambda_1$ is not an eigenvalue, then $T - \lambda_1 I$ implies $T = \lambda_1 I$. Since $\lambda_1$ is not an eigenvalue $(T-\lambda_1 I)$ is surjective. Hence $f(\lambda_1) = n$ by rank-nullity. We will assume that the output of $\dim$ is a nonnegative integer ($\operatorname{dim} = 0, 1, 2, \dots$). Since we have $f = n$ and $f \leq n-1$ for 2 different values, $f$ is discontinuous. (The only way to be continuous is if $f$ is constant.)  %where $\dim (T-\lambda_0 I) = \dim \{0\} = 0$, then $$\rank (T-\lambda_0 I) = \dim V - \nullity (T-\lambda_0 I) = \dim V = n.$$ Hence $f(\lambda_0) = n$. 
    \end{proof}
    \end{mybox}
    \newpage
    \item Suppose $T \in \mathcal{L}(V)$ has a diagonal matrix $A$ with respect to some basis of $V$ and that $\lambda \in \mathbf{F}$.  Prove that $\lambda$ appears on the diagonal of $A$ precisely $\dim E(\lambda , T)$ times.
\begin{mybox}
    \begin{proof}
        Let $m = \dim V$ and $\{v_1, \dots, v_m\}$ be a basis for $V$. Then denote the diagonal entries of $A$ by $\lambda_1, \dots, \lambda_n$. 

        \nl A basis for the eigenspace is every $v_i \in V$ such that $(T-\lambda_i I)v_i = \vec{0}$ with $i \in [1,m]$. This is true only when $\lambda_i = \lambda$. We can build a basis for $E(\lambda, T)$ by appending $v_i$ each time $\lambda_i = \lambda$. Since the dimention is the number of elements in a basis, $\dim E(\lambda,T)$ is exactly equal to the number of times $\lambda = \lambda_i$; the number of times $\lambda$ appeared on the diagonal of $A$.
    \end{proof}
\end{mybox}
    \vspace{0.6in}
    \item Show that the function that takes $((x_1,x_2) , (y_1,y_2)) \in \mathbf{R}^2 \times \mathbf{R}^2$ to $|x_1y_1| + |x_2y_2|$ is not an inner product on $\mathbf{R}$.
\begin{mybox}
    \begin{proof}
        Let $\phi$ be defined as follows,
        \begin{center}
            $\phi : \R^2 \times \R^2 \to \R$ \\
            $\phi\big( (x_1, x_2), (y_1, y_2) \big) = \abs{x_1y_1} + \abs{x_2y_2}$
        \end{center}
        Suppose $\phi$ is an inner product. Then definition of an inner product has homogeneity in the first slot, $\vbrac{\lambda u, v} = \lambda \vbrac{u,v}$. Leting $u = v = e_1$ and $\lambda = -1$, then
        $$\phi(-e_1, e_1) = \abs{-1 \cdot 1} + \abs{0 \cdot 0} = 1.$$
        and 
        $$-\phi(e_1, e_1) = -\pars{\abs{1 \cdot 1} + \abs{0\cdot0}} = -1.$$
        Since $1 \neq -1$ we have a contradiction to the assumption that $\phi$ is an inner product. Therefore $\phi$ is not an inner product by counterexample.
\end{proof}
\end{mybox}
    \vspace{0.6in}
    \item Suppose $T \in \mathcal{L}(V)$ is such that $\|Tv\| \le \|v\|$ for every $v \in V$.  Prove that $T - \sqrt{2} I$ is invertible.
\begin{mybox}
    \begin{proof}
        Suppose that $\lambda = \sqrt2$ is an eigenvalue. Let $u \in V^*$. Then $Tu = \lambda u = \sqrt 2 u$ and $\norm{Tu} = \norm{\sqrt{2} u} = \sqrt{2} \norm{u}$. This contadicts the assumption that \\$\norm{Tv} \le \norm{v}$, hence $\sqrt2$ is not an eigenvalue. Consequestly, $(T-\sqrt{2}I)$ is invertible.
    \end{proof}
\end{mybox}
    \newpage
    \item Suppose $\norm{u} = \norm{v} = 1$ and $\vbrac{u,v} = 1$.  Prove that $u = v$.
\begin{mybox}
    \begin{proof}
        By the Cauchy-Schwarz Inequality, because $\abs{\vbrac{u,v}} = \norm u \norm v = 1$, we get that $u = cv$ for some scalar $c \in \F$. Substituting $u = cv$ into $\vbrac{u,v}$,
        \begin{align*}
            1 &= \vbrac{u,v} \\ &= \vbrac{cv,v}\\
            &= c \vbrac{v,v}\\
            &= c \cdot 1 \\
            &= c.
        \end{align*}
        Therefore $c = 1$ and $u = 1 \cdot v$ implies $u = v$.
    \end{proof}
\end{mybox}
    \end{enumerate}
\end{document}
